\begin{subfigure}[t]{0.23\textwidth}
    \centering
    \begin{tikzpicture}
      % Draw the 4×5 grid and name it "M"
      \matrix (M) [gridmatrix] {
        % row 1
        |[graycell]| & |[graycell]| & |[graycell]| & |[graycell]| \\
        % row 2
        |[redcell]|  & |[graycell]| & |[graycell]| & |[graycell]| \\
        % row 3
        |[graycell]| & |[graycell]| & |[graycell]| & |[graycell]| \\
        % row 4
        |[graycell]| & |[graycell]| & |[graycell]| & |[graycell]| \\
        % row 5
        |[graycell]| & |[graycell]| & |[graycell]| & |[graycell]| \\
      };

      % Column labels: attach ABOVE the north anchor of row 1 cells
      \foreach \c [count=\i from 1] in {$h_1$,$h_2$,$h_3$,$h_4$} {
        \node[anchor=south] at (M-1-\i.north) {\c};
      }

      % Row labels: attach LEFT of the west anchor of column 1 cells
      \foreach \r [count=\j from 1] in {$p_1$,$p_2$,$p_3$,$p_4$,$p_5$} {
        \node[anchor=east] at (M-\j-1.west) {\r};
      }
    \end{tikzpicture}
    \captionsetup{width=0.8\textwidth}  % override for this one only
    \caption{The first unit learned is of the form $x_{2, 1}$}~\label{subfig:pigeonhole-a}  
\end{subfigure}
\begin{subfigure}[t]{0.23\textwidth}
    \centering
    \begin{tikzpicture}
        % Draw the 4×5 grid and name it "M"
        \matrix (M) [gridmatrix] {
            % row 1
            |[graycell]| & |[graycell]| & |[graycell]| & |[graycell]| \\
            % row 2
            |[redcell]|  & |[graycell]| & |[graycell]| & |[graycell]| \\
            % row 3
            |[redcell]| & |[graycell]| & |[graycell]| & |[graycell]| \\
            % row 4
            |[redcell]| & |[graycell]| & |[graycell]| & |[graycell]| \\
            % row 5
            |[graycell]| & |[graycell]| & |[graycell]| & |[graycell]| \\
        };

        % Column labels: attach ABOVE the north anchor of row 1 cells
        \foreach \c [count=\i from 1] in {$h_1$,$h_2$,$h_3$,$h_4$} {
            \node[anchor=south] at (M-1-\i.north) {\c};
        }

        % Row labels: attach LEFT of the west anchor of column 1 cells
        \foreach \r [count=\j from 1] in {$p_1$,$p_2$,$p_3$,$p_4$,$p_5$} {
            \node[anchor=east] at (M-\j-1.west) {\r};
        }
    \end{tikzpicture}
    \captionsetup{width=0.8\textwidth}  % override for this one only
    \caption{In a similar fashion we then learn the units $\overline{x}_{3, 1}$ and $\overline{x}_{4, 1}$}~\label{subfig:pigeonhole-b} 
\end{subfigure}
\begin{subfigure}[t]{0.23\textwidth}
    \centering
    \begin{tikzpicture}
    % Draw the 4×5 grid and name it "M"
    \matrix (M) [gridmatrix] {
        % row 1
        |[graycell]| & |[redcell]| & |[redcell]| & |[redcell]| \\
        % row 2
        |[redcell]|  & |[graycell]| & |[graycell]| & |[graycell]| \\
        % row 3
        |[redcell]| & |[graycell]| & |[graycell]| & |[graycell]| \\
        % row 4
        |[redcell]| & |[graycell]| & |[graycell]| & |[graycell]| \\
        % row 5
        |[graycell]| & |[graycell]| & |[graycell]| & |[graycell]| \\
    };

    % Column labels: attach ABOVE the north anchor of row 1 cells
    \foreach \c [count=\i from 1] in {$h_1$,$h_2$,$h_3$,$h_4$} {
        \node[anchor=south] at (M-1-\i.north) {\c};
    }

    % Row labels: attach LEFT of the west anchor of column 1 cells
    \foreach \r [count=\j from 1] in {$p_1$,$p_2$,$p_3$,$p_4$,$p_5$} {
        \node[anchor=east] at (M-\j-1.west) {\r};
    }
    \end{tikzpicture}
    \captionsetup{width=0.8\textwidth}  % override for this one only
    \caption{By the symmetry argument, we learn units $\overline{x}_{1, 2}$, $\overline{x}_{1, 3}$, and $\overline{x}_{1, 4}$}~\label{subfig:pigeonhole-c}
\end{subfigure}
\begin{subfigure}[t]{0.23\textwidth}
    \centering
    \begin{tikzpicture}
    % Draw the 4×5 grid and name it "M"
    \matrix (M) [gridmatrix] {
        % row 1
        |[greencell]| & |[redcell]| & |[redcell]| & |[redcell]| \\
        % row 2
        |[redcell]|  & |[graycell]| & |[graycell]| & |[graycell]| \\
        % row 3
        |[redcell]| & |[graycell]| & |[graycell]| & |[graycell]| \\
        % row 4
        |[redcell]| & |[graycell]| & |[graycell]| & |[graycell]| \\
        % row 5
        |[redcell]| & |[graycell]| & |[graycell]| & |[graycell]| \\
    };

    % Column labels: attach ABOVE the north anchor of row 1 cells
    \foreach \c [count=\i from 1] in {$h_1$,$h_2$,$h_3$,$h_4$} {
        \node[anchor=south] at (M-1-\i.north) {\c};
    }

    % Row labels: attach LEFT of the west anchor of column 1 cells
    \foreach \r [count=\j from 1] in {$p_1$,$p_2$,$p_3$,$p_4$,$p_5$} {
        \node[anchor=east] at (M-\j-1.west) {\r};
    }
    \end{tikzpicture}
    \captionsetup{width=0.8\textwidth}  % override for this one only
    \caption{Constraint (1) gives $x_{1, 1}$, then constraint (2) gives $\overline{x}_{5, 1}$, completing the reduction to \ph{3}}~\label{subfig:pigeonhole-d}
\end{subfigure}

