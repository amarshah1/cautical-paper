\begin{subfigure}[t]{0.23\textwidth}
    \centering
    \begin{tikzpicture}
      % Draw the 4×5 grid and name it "M"
      \matrix (M) [gridmatrix] {
        % row 1
        |[bluepluscell]| & |[graycell]| & |[graycell]| & |[graycell]| \\
        % row 2
        |[graycell]|  & |[bluepluscell]| & |[graycell]| & |[graycell]| \\
        % row 3
        |[graycell]| & |[graycell]| & |[graycell]| & |[graycell]| \\
        % row 4
        |[graycell]| & |[graycell]| & |[graycell]| & |[graycell]| \\
        % row 5
        |[graycell]| & |[graycell]| & |[graycell]| & |[graycell]| \\
      };

      % Column labels: attach ABOVE the north anchor of row 1 cells
      \foreach \c [count=\i from 1] in {$h_1$,$h_2$,$h_3$,$h_4$} {
        \node[anchor=south] at (M-1-\i.north) {\c};
      }

      % Row labels: attach LEFT of the west anchor of column 1 cells
      \foreach \r [count=\j from 1] in {$p_1$,$p_2$,$p_3$,$p_4$,$p_5$} {
        \node[anchor=east] at (M-\j-1.west) {\r};
      }
    \end{tikzpicture}
    \captionsetup{width=0.8\textwidth}  % override for this one only
    \caption{We initially decide on $x_{1, 1}$ and $x_{2, 2}$.}~\label{subfig:pigeonholeclause-a}  
\end{subfigure}
\begin{subfigure}[t]{0.23\textwidth}
    \centering
    \begin{tikzpicture}
        % Draw the 4×5 grid and name it "M"
        \matrix (M) [gridmatrix] {
        % row 1
        |[bluepluscell]| & |[yellowminuscell]| & |[graycell]| & |[graycell]| \\
        % row 2
        |[yellowminuscell]|  & |[bluepluscell]| & |[graycell]| & |[graycell]| \\
        % row 3
        |[yellowminuscell]| & |[yellowminuscell]| & |[graycell]| & |[graycell]| \\
        % row 4
        |[yellowminuscell]| & |[yellowminuscell]| & |[graycell]| & |[graycell]| \\
        % row 5
        |[yellowminuscell]| & |[yellowminuscell]| & |[graycell]| & |[graycell]| \\
        };

        % Column labels: attach ABOVE the north anchor of row 1 cells
        \foreach \c [count=\i from 1] in {$h_1$,$h_2$,$h_3$,$h_4$} {
            \node[anchor=south] at (M-1-\i.north) {\c};
        }

        % Row labels: attach LEFT of the west anchor of column 1 cells
        \foreach \r [count=\j from 1] in {$p_1$,$p_2$,$p_3$,$p_4$,$p_5$} {
            \node[anchor=east] at (M-\j-1.west) {\r};
        }
    \end{tikzpicture}
    \captionsetup{width=0.8\textwidth}  % override for this one only
    \caption{Then unit propagation gives us $\overline{x}_{2,1}, \ldots, \overline{x}_{5,1},$ $\overline{x}_{1, 2}, \overline{x}_{3, 2}, \ldots, \overline{x}_{5, 2}$.}~\label{subfig:pigeonholeclause-b} 
\end{subfigure}
\begin{subfigure}[t]{0.23\textwidth}
    \centering
    \begin{tikzpicture}
    % Draw the 4×5 grid and name it "M"
    \matrix (M) [gridmatrix] {
        % row 1
        |[orangepluscell]| & |[orangeminuscell]| & |[graycell]| & |[graycell]| \\
        % row 2
        |[orangeminuscell]|  & |[orangepluscell]| & |[graycell]| & |[graycell]| \\
        % row 3
        |[purpleminuscell]| & |[purpleminuscell]| & |[graycell]| & |[graycell]| \\
        % row 4
        |[purpleminuscell]| & |[purpleminuscell]| & |[graycell]| & |[graycell]| \\
        % row 5
        |[purpleminuscell]| & |[purpleminuscell]| & |[graycell]| & |[graycell]| \\
    };

    % Column labels: attach ABOVE the north anchor of row 1 cells
    \foreach \c [count=\i from 1] in {$h_1$,$h_2$,$h_3$,$h_4$} {
        \node[anchor=south] at (M-1-\i.north) {\c};
    }

    % Row labels: attach LEFT of the west anchor of column 1 cells
    \foreach \r [count=\j from 1] in {$p_1$,$p_2$,$p_3$,$p_4$,$p_5$} {
        \node[anchor=east] at (M-\j-1.west) {\r};
    }
    \end{tikzpicture}
    \captionsetup{width=0.8\textwidth}  % override for this one only
    \caption{Literals $\overline{x}_{3, 1}, \ldots, \overline{x}_{5, 1},$
    $\overline{x}_{3, 2}, \ldots, \overline{x}_{5, 2}$ are in $\alpha_c$ by
    their respective constraint (1). The rest are in
    $\alpha_a$.}~\label{subfig:pigeonholeclause-c}
\end{subfigure}
\begin{subfigure}[t]{0.23\textwidth}
    \centering
    \begin{tikzpicture}
    % Draw the 4×5 grid and name it "M"
    \matrix (M) [gridmatrix] {
        % row 1
        |[graycell]| & |[orangeminuscell]| & |[graycell]| & |[graycell]| \\
        % row 2
        |[orangeminuscell]|  & |[graycell]| & |[graycell]| & |[graycell]| \\
        % row 3
        |[graycell]| & |[graycell]| & |[graycell]| & |[graycell]| \\
        % row 4
        |[graycell]| & |[graycell]| & |[graycell]| & |[graycell]| \\
        % row 5
        |[graycell]| & |[graycell]| & |[graycell]| & |[graycell]| \\
    };

    % Column labels: attach ABOVE the north anchor of row 1 cells
    \foreach \c [count=\i from 1] in {$h_1$,$h_2$,$h_3$,$h_4$} {
        \node[anchor=south] at (M-1-\i.north) {\c};
    }

    % Row labels: attach LEFT of the west anchor of column 1 cells
    \foreach \r [count=\j from 1] in {$p_1$,$p_2$,$p_3$,$p_4$,$p_5$} {
        \node[anchor=east] at (M-\j-1.west) {\r};
    }
    \end{tikzpicture}
    \captionsetup{width=0.8\textwidth}  % override for this one only
    \caption{Finally, we shrink and learn the clause $\overline{x}_{1, 2} \lor \overline{x}_{2, 1}$.
    E.g., we can omit $\overline{x}_{3, 2}$ from the condition because of clause $\overline{x}_{1, 2} \lor \overline{x}_{3, 2}$.}~\label{subfig:pigeonholeclause-d}
\end{subfigure}

