% \begin{filecontents*}{\jobname.xmpdata}
%     \Title{Context Pruning for More Robust SMT-based Program Verification}
%     \Author{Yi Zhou\sep Jay Bosamiya\sep Jessica Li\sep Marijn \sep Bryan Parno}
%     \Publisher{TU Wien Academic Press}
%   \end{filecontents*}
  
  \PassOptionsToPackage{dvipsnames}{xcolor}
  \documentclass[year=25,pdfa]{fmcad}
  
  \usepackage{cite}
  \usepackage{amsmath,amssymb,amsfonts,amsthm}
  % \usepackage{algorithmic}
  \usepackage{graphicx}
  \usepackage{textcomp}
  \usepackage[dvipsnames]{xcolor}
  \usepackage{bussproofs}  % For natural deduction proofs with \inferrule
  \usepackage{mathpartir}
  \usepackage{orcidlink}

  
  % \usepackage[margin=1in]{geometry}    % Exert more control over the margins
  % \usepackage{mathptm}     % Puts math text into the postscript Times font and symbol font (else uses the metafont-generated characters)
  % \usepackage{amsmath}     % Provides various features to facilitate writing math formulas and to improve the typographical quality of their output
  \usepackage{float}       % Basic package for floating objects like figures and tables
  % \usepackage{psfrag}      % Used to insert text into figures
  % \usepackage{floatflt}    % Used to wrap text around figures
  % \usepackage{graphicx}    % Used to display graphic figures and such
  \usepackage{xspace}      % Intelligently adds space after a word via \xspace
  % \usepackage{color}       % Used to highlight comments
  % \usepackage{pifont}      % Provides the ding symbol used for comments
  % \usepackage{subfigure}   % Place two related figures side-by-side
  % \usepackage{url}         % Properly formats URLs
  % \usepackage{array}       % Add more space to the tables
  % \usepackage{algorithm}   % Defines algorithm float environment
  % \usepackage{algpseudocode}
  \usepackage{multirow}    % Use to create table with a column that spans multiple rows
  \usepackage{enumerate}   % Control the numbering for enumerated lists
  \usepackage[square,comma,numbers,sort&compress]{natbib}      % Offers more control over citation appearance and bib spacing
  % \usepackage{enumitem}
  \usepackage{diagbox}
  % \usepackage{makecell}
  \usepackage{booktabs}
  % \usepackage[T1]{fontenc}
  \usepackage[noend]{algpseudocode}
  % \usepackage{algpseudocode}
  % \usepackage{authblk}
  \usepackage{pifont}
  % \usepackage[switch]{lineno} % Line numbering for draft-mode
  % \usepackage{tikz}
  % \usetikzlibrary{shapes.geometric, arrows}
  % \usetikzlibrary{decorations.pathreplacing,calligraphy}
  \usepackage{tikz}
  \usetikzlibrary{matrix}

  % ------------------------------------------------------------------
  % Dot grid styles 
  \newcommand*\dotdiam{15pt}      % <— tweak this for bigger / smaller circles
  \newcommand*\dotgap {4pt}      % <— negative → dots touch; 0pt = tangent

  \tikzset{
    cell/.style   = {draw,circle,fill=gray!60,
                    minimum size=\dotdiam, inner sep=0pt, outer sep=0pt},
    redcell/.style= {cell,fill=red!60},
    greencell/.style  = {cell,fill=green!60},
    graycell/.style = {cell,fill=gray!40},
    gridmatrix/.style = {
        matrix of nodes,
        nodes   = {cell},
        column sep=\dotgap, row sep=\dotgap,
        % ampersand replacement=\&,                      % enable & inside matrix
        nodes in empty cells,                          % draw every entry
    },
  }
  \usepackage{subcaption}
  
% }
  \usepackage{outlines}
  \usepackage{listings}
  % \usepackage[caption=false]{subfig}
  \usepackage{soul}
  \usepackage{microtype}
  
  \def\BibTeX{{\rm B\kern-.05em{\sc i\kern-.025em b}\kern-.08em
      T\kern-.1667em\lower.7ex\hbox{E}\kern-.125emX}}
  
  % Used for displaying a sample figure. If possible, figure files should
  % be included in EPS format.
  %
  % If you use the hyperref package, please uncomment the following line
  % to display URLs in blue roman font according to Springer's eBook style:
  % \renewcommand\UrlFont{\color{blue}\rmfamily}
  
  
  \usepackage[linesnumbered,ruled,vlined]{algorithm2e}
  % \SetAlCapSkip{1em} % or 10pt, 12pt, etc.
  \usepackage{hyperref}
  \hypersetup{
  %\ifpdf
  %        pdftex,
  %\else
  %        dvipdf,
  %\fi
    %pdftex,              % using ps2pdf vs pdftex
    %ps2pdf,              % using ps2pdf vs pdftex
          %hypertex,
    colorlinks=true,     % color the words instead of use a colored box
    urlcolor=blue,       % \href{...}{...} external (URL)
  %  filecolor=blue,      % \href{...} local file
    linkcolor=blue,     % \ref{...} and \pageref{...}
    citecolor=blue,     % \cite{}
  % letterpaper=true,
    plainpages=false,
  %    plainpages          boolean         true
  %    Forces page anchors to be named by the arabic form of the page number,
  %    rather than the formatted form.
    breaklinks=true,
  %    breaklinks          boolean         false
  %    Allows link text to break across lines; since this cannot be accommodated in
  %    PDF, it is only set true by default if the pdftex driver is used. This makes
  %    links on multiple lines into different PDF links to the same target.
  % pagebackref=true,
  %     Adds ?backlink? text to the end of each item in the bibliography, as a list
  %     of section numbers. This can only work properly if there is a blank line
  %     after each \bibitem.
    bookmarksnumbered=true,
  %    bookmarksnumbered   boolean         false
  %    If Acrobat bookmarks are requested, include section numbers.
    bookmarksopen=true,
  %    bookmarksopen       boolean         false
  %    If Acrobat bookmarks are requested, show them with all the subtrees expanded.
    pdftitle={},
    pdfauthor={},
    pdfkeywords={},
  % pdfpagelabels=true,
    pdfpagemode=UseOutlines  % None, UseThumbs, UseOutlines, FullScreen
  }

  \renewcommand{\sectionautorefname}{Section}
  \renewcommand{\subsectionautorefname}{Subsection}
  \renewcommand{\algorithmautorefname}{Algorithm}


  
  \usepackage[capitalise]{cleveref}
%   \usepackage{stys/dafny}
%   \usepackage{stys/smtlib}
  % \usepackage{array}
  \usepackage{tabularx}
  
%   \input{00-common-preamble.tex}
  
  % \algrenewcommand\algorithmicindent{0.65em}%

  \newtheorem{definition}{Definition}
  \newtheorem{theorem}{Theorem}
  \newtheorem{lemma}{Lemma}
  \newtheorem{claim}{Claim}


  
  \newcommand{\unsat}{\texttt{unsat}\xspace}
  \newcommand{\sat}{\texttt{sat}\xspace}
  \newcommand{\tool}{CA\textsc{uti}C\textsc{al}\xspace}
  \newcommand{\pr}{\texttt{PR}\xspace}
  \newcommand{\er}{\texttt{ER}\xspace}
  \newcommand{\impunit}{\vdash_1}
  \newcommand{\lingeling}{\textsc{lingeling}\xspace}
  \newcommand{\cadical}{C\textsc{a}D\textsc{i}C\textsc{aL}\xspace}
  \newcommand{\sadical}{S\textsc{a}D\textsc{i}C\textsc{aL}\xspace}
  \newcommand{\glucoser}{G\textsc{lucos}ER\xspace}
  \newcommand{\prelearn}{PR\textsc{e}L\textsc{earn}\xspace}
  \newcommand{\ph}[1]{\ensuremath{\mathrm{PHP}(#1)}}



  
  \begin{document}
  \title{Learning Short Clauses by Conditional Autarkies}
  
  % \author{TODO: add authors and orcid \\
  \author{Amar Shah \orcidlink{0009-0008-8282-2142}, Twain Byrnes \orcidlink{0009-0000-7939-1195}, Joseph Reeves \orcidlink{0000-0002-4585-0565}, Marijn J.\ H.\ Heule \orcidlink{0000-0002-5587-8801} \\
  \emph{Carnegie Mellon University, Pittsburgh, PA, USA} \\
  \texttt{\{amarshah,binarynewts,jereeves,marijn\}@cmu.edu}}
  
  \maketitle
  
  \begin{abstract}
    % \input{texs/abstract}
    State-of-the-art Boolean satisfiability (SAT) solvers increasingly use techniques that go beyond resolution. These admit short proofs for benchmark families with exponentially large resolution proofs, such as the pigeonhole principle and mutilated chessboard. 
    %
    One of the strongest such techniques is \pr learning. However, existing \pr clause learning techniques require an NP-hard check. Hence they are computationally expensive and difficult to integrate into existing tools.
 
    % todo: I need to cite some paper that introduces pigeonhole and mutilated chessboard
    % Propagation redundant (\pr) clauses are a class of satisfiability-preserving clauses that can be learned by modern SAT solvers.  

    We propose a new method for learning \pr clauses based on conditional autarkies and implement it in the SAT solver \cadical. Our method is modular and learns \pr clauses in linear time. Additionally, we introduce a number of heuristics, including a clause-shrinking technique, to ensure that the learned clauses are useful.
    %
    % todo: add that the we added our thing into cadical
    We show that we are competitive with state-of-the-art \pr clause learning techniques, improving the performance on a large portion of SAT competition benchmarks.
    % todo : I would like a better way to talk about the heuristics
    % \keywords{Program Verification \and SMT \and Proof Stability.}
  \end{abstract}
    
  % \begin{IEEEkeywords}
  % component, formatting, style, styling, insert
  % \end{IEEEkeywords}
  
  \section{Introduction}~\label{sec:intro}


Boolean satisfiability (SAT) solving is a core tool in computer science with
applications in program
verification~\cite{BillionQueries,sat-hardwareverification,ic3,bmc},
planning~\cite{planning,planningassat}, cryptography~\cite{cryptominisat}, and
mathematics~\cite{chromaticnumber,pythagoreantriples,kellersconjecture,emptyhexagon}.
As its usage expands, so too does the need for more powerful and specialized
solving techniques.



% It is most commonly solved using the conflict-driven clause learning (CDCL)
% framework. This framework explores the state space of possible solutions,
% while learning clauses when it reaches a conflict, to aid future explorations.
One such class of techniques is propagation redundant (\pr) clause
learning~\cite{prclauses}. In this technique, a solver generates and adds
clauses which are satisfiability-preserving; that is, clauses whose conjunction
with the formula is satisfiable if and only if the original formula is
satisfiable. The solver aims to learn clauses that drastically shrink the
potential solution space, such as clauses that break symmetries.

%A classic problem involving symmetries is the pigeonhole problem, which asks if
%it is possible to fit $n+1$ pigeons into $n$ holes with at most one pigeon per
%hole. In this problem, one may learn a clause equivalent to the lemma:
%\emph{pigeon 1 is not in hole 1}. Adding this clause restricts the search space
%for the first pigeon (and thus the problem), but it preserves satisfiability,
%as hole 1 is symmetric with all holes.

% One can ensure that these clauses are satisfiable within the given proof
% system.

% for instance vivification and blocked clause addition. These techinques use
% other facts about the formula to learn clauses which are often useful for
% solving.

The power of this technique is tied to the strength of the underlying proof
system. A more powerful proof system can admit shorter proofs, which can be
found and checked quicker. Most SAT solvers rely on resolution-based proof
systems, which are ineffective for many hard instances. Consider the pigeonhole
principle (PHP), which asks if it is possible to fit $n+1$ pigeons into $n$
holes with at most one pigeon per hole. With a linear number of \pr reasoning
steps, one may learn a \pr clause equivalent to the lemma: \emph{pigeon 1 is not
in hole 1}. Adding this clause restricts the search space for the first pigeon
(and thus the problem), but it preserves satisfiability, as hole 1 is symmetric
with all holes. 

It is difficult to perform similar reasoning compactly with resolution, and in
fact, PHP is known to require exponentially large resolution
proofs~\cite{hakenpigeonhole}. Proof systems
based on \pr clauses, however, have short proofs for such problems, including a
cubic proof (in the number of pigeons) for PHP~\cite{prclauses}. PHP is especially
important as it occurs frequently as a subproblem for many different SAT benchmarks.

The theoretical power of \pr learning comes at a cost, and efficiently learning
useful \pr clauses is a challenge. State-of-the-art techniques such as
Satisfiability Driven Clause Learning (SDCL) rely on calling another SAT solver
to verify that a candidate clause is \pr \cite{sadical}. 
% This is both expensive and difficult to integrate into existing tools. 
Thus, in the worst case, the solver takes exponential time to learn a single \pr
clause. This makes integration into high-performance solvers difficult and
limits their practical impact. To the best of our knowledge, none of the most popular SAT solvers
such as \cadical~\cite{cadical}, \kissat~\cite{kissat},
\cryptoMiniSAT~\cite{cryptominisat}, or \lingeling~\cite{lingeling} support
\pr clause learning in their main branch.
% maybe cite cryptominisat instead of lingeling? todo : check crypto mini sat to
% see if it uses \pr clauses

% todo : should we explain what conditional autarkies are? or go more into depth
% about globally blocked clauses?
In this work, we propose a new, more efficient approach to learn \pr clauses
based on conditional autarkies. Intuitively, a conditional autarky provides a
way to force the value of certain variables given a set of assumptions, e.g., in
PHP if pigeons $3 $ to $ n+1$ are not in hole $1$ or hole $2$ (see \autoref{subfig:pigeonholeclause-c}
for a visual representation), then pigeon $1$
can be placed in hole $1$ and pigeon $2$ can be placed in hole $2$. Kiesl et.
al. proposed an algorithm for finding conditional autarkies in linear
time~\cite{conditionalautarkies}, using them to delete clauses. In our work, we
will use conditional autarkies to construct and add \pr clauses. 
%We make use of an algorithm proposed by Kiesl et. al. for finding a conditional
%autarky given a assignment in linear time. 
%
%Given a assignment to a SAT formula, our method divides the assignment
%into a \emph{conditional} part and an \emph{autarky} part. Using this
%distinction, we can learn a globally blocked clause (a type of \pr clause) in
%linear time.

While conditional autarkies present a means for bridging the theoretical gap in
identifying \pr clauses, the \pr clauses they produce are not immediately useful
in real SAT solving applications. There are two major practical limitations: (1)
larger \pr clauses may not meaningfully constrain the search space, and (2) some
smaller \pr clauses may be trivial and distract the solver. We solve (1) by
introducing a shrinking procedure to extract compact, useful \pr clauses, and
(2) by introducing a number of heuristics for filtering away trivial \pr
clauses. Returning to PHP, instead of the conditional autarky described above
which produces a clause with at least $2n$ literals, it is possible through
shrinking to learn a binary \pr clause either forbidding pigeon $1$ in hole $2$ or
pigeon $2$ in hole $1$.
% todo: would like to say more about what these heuristics are

% Integrating \pr clause learning into a SAT solver is still an active area of
% study

To summarize, we make the following contributions: 

\begin{enumerate} 
    \item We introduce a method for learning \pr clauses in linear time in the
    size of the formula. 
    \item We develop a number of shrinking and filtering heuristics, to learn
    concise and useful \pr clauses.% from conditional autarkies.. 
    \item We implement these techniques in a solver, \tool\footnote{\tool's code
    is available at \url{https://github.com/amarshah10/cautical}.} (a fork of
    the state-of-the-art SAT solver \cadical), and evaluate it on pigeonhole 
    principle benchmarks and a suite of benchmarks from the annual SAT competitions.
\end{enumerate}
  \section{Background}~\label{sec:background}

We begin with some SAT preliminaries. Variables $x_1, x_2, ...$ take values
\emph{true} ($\top$) or \emph{false} ($\bot$). A literal $l$ is a variable
$x$ or its negation $\overline{x}$. A clause $C$ is a disjunction of literals,
e.g., $C = x_1 \lor \overline{x}_4 \lor x_6$. A clause may also be represented by the set of its literals:
the prior clause $C = \{x_1, \overline{x}_4, x_6\}$. This motivates denoting that a literal
$l$ occurs in clause $C$ via $l \in C$. A conjunctive normal form (CNF)
formula $\phi$ is a conjunction of (disjunctive) clauses. In this paper,
all formulas $\phi$ are CNF.

We use $var(\phi)$ to represent the set of variables occurring in formula $\phi$.
A partial assignment $\alpha : V \rightarrow \{\top, \bot\}$ on $\phi$
maps variables $V \subseteq var(\phi)$ to true or false. If $V
= var(\phi)$, then $\alpha$ is a (total) assignment. We abuse notation by
denoting $\alpha(l) = \alpha(x)$ if $l = x$ and $\alpha(l) = \neg \alpha(x)$ if
$l = \overline{x}$.

A formula restricted to a partial assignment $\phi|_\alpha$ is the formula resulting from
mapping variables in the domain of $\alpha$ to their assignments. We can
simplify such a formula by removing \emph{literals} assigned false and
\emph{clauses} including a literal assigned true (and thus satisfied).
For example, with formula $\phi =
(x_1 \lor \overline{x}_4 \lor x_6) \land (x_2 \lor x_3) \land \overline{x}_5$
and assignment $\alpha$ mapping $x_1$ to $\bot$ and $x_2$ to $\top$,
we have $\phi|_\alpha = (\overline{x}_4 \lor x_6) \land \overline{x}_5$.

A clause $C = l_1 \lor \dots \lor l_m$ is \emph{blocked} by an assignment
that maps each $l_i \in C$ to false ($\bot$). An assignment $\alpha$
\emph{touches} a clause $C$ if there is a variable $x$ in the domain of $\alpha$
such that $x \in C$ or $\overline{x} \in C$. We say $\alpha$ \emph{satisfies}
$C$ if $\alpha(l) = \top$ and $l \in C$.

Assignment $\alpha$ \emph{satisfies} formula $\phi$ if it satisfies every
clause $C \in \phi$. If there exists such a satisfying assignment, then $\phi$ is
\emph{satisfiable}. The boolean satisfiability problem asks whether a given
formula $\phi$ is satisfiable.

\subsection{Redundant Clauses}~\label{subsec:redundant}

A clause $C$ is \emph{redundant} (or \emph{satisfiability-preserving}) with
respect to a formula $\phi$ if the formulas $\phi$ and $\phi \land C$ are
\emph{equisatisfiable}, i.e. $\phi$ is satisfiable if and only if $\phi \land C$
is satisfiable.

In a \emph{clausal proof system}, each step will add or remove a redundant
clause $C$. The step may contain extra information, such as a boolean witness,
justifying why $C$ is redundant. A list of redundant clauses ending with the
empty clause $\bot$ is a proof of unsatisfiability for formula $\phi$.

% If $\phi$ is unsatisfiable, then any clause $C$ will be redundant. However, when solving, we do not know whether $\phi$ is satisfiable a priori. Clausal proof systems can justify that a clause is redundant.

Adding redundant clauses may be helpful in certain cases, since adding a clause
will constrain the set of possible solutions. However, it could be harmful as
new clauses may negatively interact with solver heuristics.

% \subsection{Reasoning Steps}~\label{subsec:reasoning-steps}

\emph{Resolution} is the main proof step in a clausal proof system. It states that two clauses $C \lor x$ and $\overline{x} \lor D$ and returns a new clause $C \lor D$.

\begin{equation*}
    \inferrule*[Right=Resolution]{
        C \lor x \\ \overline{x} \lor D
    }{
        C \lor D
    }    
\end{equation*}

\emph{Unit propagation} is a core reasoning techniques in a SAT solver. Starting with empty partial assignment $\alpha$, if a formula $\phi$, contains a \emph{unit clause} $l$, i.e. a clause with only a single literal, then we set $\alpha(l) = \top$. Then, this unit is propagated, i.e. we consider the formula $\phi|_\alpha$ and continue this process until there are no unit clauses remaining. When unit propagation terminates with $\phi|_\alpha = \bot$, we say that it derives a conflict.
% todo: notation issue: I never defined \bot as the empty clause and I also am using \bot both as false and the empty clause.

Unit propagation can be thought of as a proof step. Indeed, it can be thought of as a mechanized application of resolution.

\begin{equation*}
    % \inferrule*[Right=Unit-Pos]{
    %     C \lor l \\ l
    % }{
    %     \bot
    % }
    % \qquad \qquad \qquad
    \inferrule*[Right=Unit-Prop]{
        C \lor l \\ \overline{l}
    }{
        C
    }
\end{equation*}

Given a formula $\phi$ and clause $C = l_1 \lor ... \lor l_k$, we can say $\phi \impunit C$, i.e. ``$\phi$ implies $C$ via unit propagation,'' if $\phi \land \overline{l}_1 \land \dots \land \overline{l}_k$ derives a conflict. Here, $C$ is a simple example of a redundant clause w.r.t. $\phi$. 

% This can be justified in almost any proof system for propositional logic. However, this is not very useful as it rules out







For two formulas, we say $\phi \impunit \psi$ if for every clause $C \in \psi$, $\phi \impunit C$. 
% For literals $l_1$ and $l_2$, we sometimes notate $l_1 \impunit^\phi l_2$ to mean $\phi \vdash_1 \overline{l_1} \land l_2$. Informally, we can think of this as saying ``$l_1$ implies $l_2$ via unit propagation on $\phi$,'' which means that .
% see definition here: https://www.cs.cmu.edu/~mheule/publications/prencode.pdf
% However, this is not very useful as it is quiet easy for the solver to figure this out. Instead, we must appeal to a more complicated notion of redundancy.

% The propagation redundant (\pr)

\subsection{Propagation Redundant}~\label{subsec:pr}

While resolution is complete for propositional logic, more powerful proof steps can yield shorter proofs and faster runtimes. One such step is based on \pr clauses.

\begin{definition}[Propagation Redundant (\pr) clauses]
    For formula $\phi$, we say that clause $C$ (blocked by $\beta$) is propagation redundant (\pr) w.r.t. $\phi$ if there exists an assignment $\omega$ known as the witness such that $F|_\beta \impunit F|_\omega$ and $\omega$ satisfies $C$
\end{definition}

Such a clause $C$ must be redundant. Say there is a satisfying assignment $\alpha$ for $\phi$. Then if $\alpha$ is not a satisfying assignment for $\phi \land C$, then it must be that $\beta \subseteq \alpha$, i.e. $\alpha$ extends $\beta$. However, since $F|_\beta \impunit F|_\omega$, it must be that any assignment that satisfies $F|_\beta$, will satisfy $F|_\omega$. 

Then we can define $\alpha' (x) = \omega(x)$ if $x \in \omega$ and $\alpha'(x) = \alpha(x)$ otherwise. Thus, $\alpha'$ satisfies $\phi$ and $\alpha'$ satisfies $C$.

% Intuitively, we can think of adding $C$ as the constraint that prunes all assignments that extend $\beta$. Since $F|_\beta \impunit F|_\omega$, it must be that any assignment that satisfies $F_\beta$, will satisfy $F_\omega$.

% Additionally, since $\omega$ satisfies $C$, removing the assignments that extend $\beta$, will not affect satisfiability.
% % todo : this needs ot be explained better.

% If $C$ is a \pr clause w.r.t $\phi$, then $\phi$ and $\phi \land C$ are equisatisfiable. Indeed, the clause defined in \autoref{thm:gbcequisat} is \pr with witness $\alpha_a$. 

However, checking if a clause is \pr is NP-complete~\cite{prclause}, so witnesses must be provided for proof checking. \pr clauses subsume many classes of redundant clauses, including resolution asymmetric tautologies (RATs)~\cite{rat}, blocked clauses~\cite{blockedclause}, set-blocked clauses~\cite{setblocked}, and globally-blocked clauses~\cite{conditionalautarkies}.

\subsection{Autarkies and Globally Blocked Clauses}~\label{subsec:autarkies}

% A key realization in prior work~\cite{}

\begin{definition}[Autarky]
    A nonempty assignment $\alpha$ is an autarky for a formula $\phi$ if every clause $C \in \phi$ touched by $\alpha$ is satisfied.
\end{definition}

% An autarky $\alpha_a = a_1, ..., a_m$ can be very useful as $a_1 \land ... \land a_m$ is an equisatisfiable clause. However, these can be difficult to find.

In plain words, an autarky is an assignment that satisfies every clause it touches. For example, if $\alpha$ was a satisfying assignment, it would be an autarky since it satisfies every clause.

\begin{definition}[Conditional Autarky]
    A nonempty assignment $\alpha = \alpha_c \sqcup \alpha_a$ (disjoint union) is a conditional autarky for a formula $\phi$ if $\alpha_a$ is an autarky for $\phi|_{\alpha_c}$.
\end{definition}

Specifically, we can think about searching for conditional autarkies by first looking for a partial assignment $\alpha_c$, and then finding an autarky $\alpha_a$ on the reduced formula $\phi|_{\alpha_c}$.

Conditional autarkies can be very useful for learning equisatisfiable clauses, for instance, if $\alpha_c = c_1, ..., c_n$ and $\alpha_a = a_1, ..., a_m$, we add clauses of the form:

\begin{equation*}
    [c_1 \land ... \land c_n] \rightarrow [a_1 \land ... \land a_m]
\end{equation*}

This results in $m$ different clauses as in the following theorem from Kiesl et al.~\cite{conditionalautarkies}:

\begin{theorem}~\label{thm:gbcequisat}
    Formula $\phi$ and $\phi \land \bigwedge_{1 \leq i \leq m} (\overline{c_1} \lor ... \overline{c_n} \lor a_i)$ are equisatisfiable.
\end{theorem}

This means that $\phi$ is satisfiable if and only if $\phi \land \bigwedge_{1 \leq i \leq m} (\overline{c_1} \lor ... \overline{c_n} \lor a_i)$ is satisfiable. Thus, the solver can add any of the $m$ clauses $\overline{c_1} \lor ... \overline{c_n} \lor a_i$ and preserve satisfiability. Each of these clauses is a \pr clause with witness $\alpha_a$. Each of these clauses is a \emph{globally blocked clause}, however, this is not too important for our purposes, so we will not discuss it further.
% Additionally, we can see that this is a \pr clause:

% todo: this theorem works for us because of the algorithm we use to prove this, but it is not true in the general case.

% \begin{theorem}~\label{thm:totalassignmnet}
%     For $1 \leq i \leq m$, the clause $C = \overline{c_1} \lor ... \overline{c_n} \lor a_i$ is a \pr clause w.r.t $\phi$ with witness $\alpha_a$
% \end{theorem}

% \begin{proof}
%    Since $a_i \in \alpha_a$, $\alpha_a$ satisfies $C$.
   
%    Define the assignment that blocks $C$ as $\beta = c_1, ..., c_n, \overline{a_i}$. Now we want to show that: $F|_\beta \impunit F|_{\alpha_a}$. Take a clause $C \in F|_{\alpha_a}$. We know that
% \end{proof}




\subsection{Related Work}~\label{subsec:relatedwork}

The pigeonhole problem asks if $n+1$ pigeons can fit into $n$ holes. The
mutilated chessboard problem asks if a $2n \times 2n$ board with two diagonally
opposite corners removed can be tiled with $2 \times 1$ rectangular dominoes.
Both are unsatisfiable and it is easy for a human to see why. However, it has
been shown there are no polynomial-sized resolution proofs for either
problem~\cite{hakenpigeonhole,mutilatedchessboard-exponential}.
% todo: need to clarify if I should say that there are not polynomial-sized resolution proofs for these problems or if there are no sub-exponential-sized resolution proofs for these problems.

% todo: need to be consistent about whether I am using terminology "pigeonhole principle" or "pigeonhole problem"

% todo: I need to cite some paper that introduces pigeonhole and mutilated chessboard


 While extended resolution (\er) provided $O(n^4)$ for the pigeonhole problems, the proof system involved introducing new variables~\cite{er}. In general, the search space for new variables is infinite and thus tools like \glucoser based on \er did not scale well~\cite{glucoser}.

The \pr proof system remedies this by producing $O(n^3)$ proofs for the pigeonhole formula~\cite{prclauses} without learning any new variables. Later, this was shown to produce short proofs for mutilated chessboard problems~\cite{mutilatedchessboard-pr}. 
% todo we use O(n^_) where n is formula, but also where n is the number of variables.

Solvers that implement the \pr proof system typically use the satisfaction-driven clause learning (SDCL) framework~\cite{sdcl}, which extends conflict-driven clause learning (CDCL)~\cite{cdcl}. 
% Here, if unit propagation does not lead to a conflict, then they attempt to learn a \pr clause instead.
% \pr clause learning was first introduced in an extension of the solver \lingeling~\cite{prclause}.
After propagating an assignment, they would check if the clause $C$ that was blocked by this assignment was \pr. This was done by creating a new SAT formula called the \emph{positive reduct}. If the positive reduct was satisfiable, then $C$ is a \pr clause and it was added. This was implemented in an extension of the solver \lingeling and was shown to scale well on pigeonhole benchmarks.

Later, two new variants of the positive reduct were proposed for more aggressive pruning of the search space \cite{sadical}. This allowed SDCL to solve other difficult problems such as Tseitin formulas~\cite{hardexamplesresolution}. This was implemented in a new SDCL solver \sadical.


\prelearn is a preprocessing technique for \pr clauses~\cite{prelearn}. It initially considers many possible clauses and queries \sadical to see which were \pr.

Our work differs as we do not use a \emph{positive reduct} to test if a clause
is \pr. Instead, our clauses \pr by construction as they come from a conditional
autarky. This has the potential downside that our clause may be very large. To
remedy this we apply a shrinking technique to reduce the size of a clause.
Additionally, prior techniques are sensitive to the encoding of the problem.
Minor changes such as literal and clause reordering can tank performance. We
provide two \emph{symmetry hardening} techniques to handle reordering. We
compare our implementation \tool to \sadical and \prelearn in
\autoref{sec:evaluation}.

Kiesl et al.~\cite{conditionalautarkies} first introduced conditional autarkies to identify globally blocked clauses. They aimed to eliminate globally blocked clauses from a formula. This technique allowed them to simulate circuit-simplification techniques. Our work differs from this as we add clauses instead of removing them and we target a different class of benchmarks.

  \section{Methodology}~\label{sec:method}

Our methodology is based on four steps:
\begin{algorithm}
    \caption{Our Methodology}\label{alg:methodology}
    \SetAlgoNoLine
    \SetKwFunction{Method}{Method}
    \SetKwFunction{LCP}{LCP}
    \SetKwFor{For}{for}{:}{}
    \SetKwFor{If}{if}{:}{}
    \SetKwProg{Fn}{Function}{:}{}
    \SetKwBlock{Begin}{}{}

    \Fn{LearnClause($\psi$, $\alpha$)}{
        \For{$i \in vars(\psi)$}{
            \For{$j \in vars(\psi)$}{
                propagate($i$)
                propagate($j$)
                $\alpha_c, \alpha_a := \LCP(\alpha)$
                clause := shrink($\alpha_c, \alpha_a$)
                $\psi := \psi \land clause$
            }
        }
    }
\end{algorithm}

\subsection{Learning Clauses}~\label{subsec:learning}

As described in \autoref{subsec:autarkies}, we can split a partial assignment $\alpha = \alpha_c \sqcup \alpha_a$ and use this to learn a \pr clause. Since all of $\alpha_c$ appears in such a clause, we must try to keep in as small as possible.

We present this algorihtm from Kiesel et al \cite{conditionalautarkies} to find the smallest possible $\alpha_c$:


\begin{algorithm}
    \caption{Minimiazing $\alpha_c$ in $\alpha = \alpha_c \sqcup \alpha_a$}\label{alg:leastcond}
    \SetAlgoNoLine
    \SetKwFunction{LCP}{LeastConditional}
    \SetKwFor{For}{for}{:}{}
    \SetKwFor{If}{if}{:}{}
    \SetKwProg{Fn}{Function}{:}{}
    \SetKwBlock{Begin}{}{}

    \Fn{\LCP{$\psi$, $\alpha$}}{

        $\alpha_c := \emptyset$\;
        \For{$C \in \phi$}{
            \If{$\alpha$ touches $C$ without satisfying $C$}{
                $\alpha_c := \alpha_c \cup (\alpha \cap \overline{C})$\;
            }
        }
        \Return{$\alpha_c$}\;
    }
\end{algorithm}

We can then compute $\alpha_a := \alpha \backslash \alpha_c$. It is pretty easy to see why this is a conditional autarky, since every clause that $\alpha_a$ touches, must be satisfied by some literal in $\alpha$.

We can see that $\alpha_c$ is minimal since for each clause that is touched, we add in those literals from the assignment that touch and do not satisfy the clause. They must be in $\alpha_c$, since otherwise $\alpha_a$ will touch a clause that is not satisfied in $\alpha_a$ and $\alpha_c$, violating the conditional autarky property.

\subsection{Shrinking Clauses}~\label{subsec:shrinking}

While we minimize $\alpha_c$, this still leads to a long clause in many cases. However, we can shrink the clause using some clever techniques. Notice that if $\alpha_c = c_1, ..., c_n$ and $\alpha_a = a_1, ..., a_m$, we could add clauses of the form $\overline{c_1} \lor ... \lor \overline{c_n} \lor a_i$ where $1 \leq i \leq m$.


% Consider $\overline{\alpha_a}$, the negation of the literals in $\alpha_a$. 
We consider the set $C_0 = \{c_j \in C| \exists a_i \in \alpha_a \; \overline{a_i} \impunit^\phi c_j \}$. This is essentially the set of literals in $\alpha_c$ that can be implied by unit propagation by negating some literal in $\alpha_a$.
%  in the code we propagate -alpha_a[i]
%  then if val (neg_alpha_c[j]) < 0, we add neg_alpha_c[j] to propagated and
%  alpha_a[i] to useful

We also define $A_0 \subseteq \alpha_a$ as a set such that for each $c \in C_0$, there is some $a \in A_0$ such that $\overline{a} \impunit^\phi c$. By definition, there must be some $A_0$, but there could be many possible $A_0$. We will discuss how we pick $A_0$ in \autoref{subsec:sym} as it will be very important for having a technique resistant to different encodings. 


% Instead we look at $\overline{I} = \{x_{2, 1}, x_{1, 2}\}$. Then take the set $\overline{I} \vdash_1 C_0 \subseteq C$, i.e., the maximal subset of $C$ that $\overline{I}$ implies via unit propagation.

We want learn the clause $\bigvee_{c \in C \backslash C_0} \overline{c} \lor \bigvee_{a \in A_0} a$


\begin{theorem}~\label{thm:shrunkgbcequisat}
    The formula $\phi$ is satisfiable if and only if $\phi \land (\bigvee_{c \in C \backslash C_0} \overline{c} \lor \bigvee_{a \in A_0} a)$ is satisfiable.
\end{theorem}

\begin{proof}
    \underline{$\Leftarrow$:} This is immediate


    \underline{$\Rightarrow$:}  As a corollary to \autoref{thm:gbcequisat}, $\phi$ is satisfiable if and only if $\phi \land (\bigvee_{c \in C} \overline{c} \lor \bigvee_{a \in A_0} a)$ is satisfiable.

    Thus we can assume $\phi \land (\bigvee_{c \in C} \overline{c} \lor \bigvee_{a \in A_0} a)$ is satisfiable by some satisfying assignment $\beta$.

    We claim $\beta$ is a satisfying assignment for $\phi \land (\bigvee_{c \in C \backslash C_0} \overline{c} \lor \bigvee_{a \in A_0} a)$. This could only not be the case if (1)~for all $a \in A_0$ $\overline{a} \in \beta$ and (2)~there is some $c \in C_0$ such that $\overline{c} \in \beta$. 

    But by definition there is some $a \in A_0$ such that $\overline{a} \impunit^\phi c$. Thus $\phi \land \overline{a} \land \overline{c}$ is unsatisfiable via unit propagation. However, this cannot be the case since $\overline{a}, \overline{c} \in \beta$ and $\beta$ is a satisfying assignment for $\phi$.
\end{proof}

\subsection{Resilience to Encoding}~\label{subsec:sym}

As mentioned in \autoref{subsec:shrinking}, our choice for $A_0$ can matter a lot for which clause we learn. 


\noindent \textbf{Greedy Set Cover:} %~\label{subsubsec:greedysetcover}
In order to minimize the size of the clause, we may want the smallest possible $A_0$ that maximizes $C_0$. Say for each $a \in \alpha_a$, we define $\alpha_a^{SETS}(a) = \{ c \in \alpha_c \; | \; \overline{a} \impunit^\phi c\}$. 

In the case that $C_0 = \alpha_C$, finding the smallest $A_0$ is exactly the set cover problem with $\alpha_a^{SETS}$. This is NP-hard and we do not want to spend time figuring this out. Instead, we can use the greedy algorithm to solve, which is a $1 + \ln |\alpha_c|$ approximation algorithm. Specifically, the greedy algorithm will find the largest set in $\{\alpha_a^{SETS} \cap (\alpha_c \backslash C_0) | a \in \alpha_a \}$ at each step and add it to $C_0$. It stops when each of the $\alpha_a^{SETS} \cap (\alpha_c \backslash C_0)$ are empty.
% either $C_0 = \alpha_C$ or there are no terms 


\begin{algorithm}
    \caption{Algorithm finding $A_0$}\label{alg:finda0}
    \SetAlgoNoLine
    \SetKwFunction{FindA}{FindA0}
    \SetKwFor{For}{for}{:}{}
    \SetKwFor{If}{if}{:}{}
    \SetKwProg{Fn}{Function}{:}{}
    \SetKwBlock{Begin}{}{}

    \Fn{\FindA{$\psi$, $\alpha_a$, $\alpha_c$}}{
        $\alpha_a$ := \texttt{sort($\alpha_a$)}\;
        $\alpha_a^{SETS}$ := init \texttt{Array[len($\alpha_a$)]}\;
        \For{$i \in \texttt{range}(\alpha_a)$}{
            \texttt{propagate($-\alpha_a[i]$)}\;
            \texttt{implied} := $\{\}$\;
            \For{$c \in \alpha_c$}{
                \texttt{propagate(-c)}\;
                \If{\texttt{unsat}}{
                $\texttt{implied} := \texttt{implied} \cup \{c\}$\;
            }
            }
            $\alpha_a^{SETS}[i]$ := \texttt{implied}\;
        }
        \Return{\texttt{greedySetCover}($\alpha_a^{SETS}$)}\;
    }
\end{algorithm}
% todo : this algorithm is not actually what we do, since we don't propagate on c -> see if this is a problem ever

In \autoref{alg:finda0}, we describe our process for calculating $A_0$. We initially pre-sort $alpha_a$ by implication, which we will later discuss in more detail. Then we intialize $\alpha_a^{SETS}$ as an array, we iterate the counter $i$ through $\alpha_a$, populating $\alpha_a^{SETS}[i]$ with the set of literals in $\alpha_c$ that are implied by $-\alpha_a^{SETS}[i]$. Finally, we apply \texttt{greedySetCover}



\noindent \textbf{Sorting $\alpha_a$ by Implication:}%~\label{subsubsec:impordering}
The greedy set cover algorithm described above is almost sufficient for being insensitive to heuristics, but there is one other optimization that we made. \autoref{alg:finda0} describes sorting $\alpha_a$ as an initial step. This is important for cases where there are more than one cover of the same size. The most common occurence of this is for $a_1, a_2 \in \alpha_a$, we may have that $\overline{a_2} \impunit^\phi \overline{a_1} \impunit^\phi c_1, ..., c_n$. 

Thus, greedy set cover could pick $\{a_1\}$ or $\{a_2\}$ as singleton sets. However, since $a_1 \impunit^\phi a_2$, we really want to learn the unit clause $a_1$ since it is much more powerful.

\subsection{Other Heuristics}~\label{subsec:heuristics}

We employ several other heuristics:

\noindent \textbf{Checking if trivial:} Introducing globally blocked clauses can affect how we divide $\alpha = \alpha_c \sqcup \alpha_a$ as done in \autoref{alg:leastcond}. Thus, it is better to avoid introducing globally blocked clauses if they are not useful. We do this by checking if $\phi \impunit C$ for each potential clause $C$. If it is implied by unit propagation, we do not learn it since we consider it easy to learn.


  \input{texs/4.optimizations}
  \section{Motivating Example}~\label{sec:motivatex}


In this section, we will walk through the classic example of the Pigeonhole Principle as a SAT problem~\cite{}, and how our techniques can be used to solve it.

The problem asks whether we can put $m$ pigeons in $n$ holes such that (1)~every pigeon is in a hole, and (2)~no hole contains more than one pigeon. This can be easily encoded as a SAT problem with variable $x_{i, j}$ represents putting the $i$-th pigeon into the $j$-th hole. We can represent constraint (1) as $\overline{x_{i, j}} \lor \overline{x_{k, j}}$ for each $ 1 \leq i \neq k \leq m$ and constraint (2) as $\bigvee_{1 \leq j \leq n} x_{i, j}$ for each $1 \leq i \leq m$

We will only consider the case where $m = n + 1$ and this is always unsatisfiable. We are able to give an $O(n^3)$ size proof for the unsatisfiability of the pigeonhole principle.

The key idea is learning binary clauses such as either pigeon 1 is not in hole 2 or pigeon 2 is not in hole 1. Then we learn the clause that either pigeon 1 is not in hole 3 or pigeon 2 is not in hole 1, and so on. These clauses may seem strange, but we can prove (in the \pr system) that their addition preserves satisfiability. We keep going until we eventually get that if pigeon 2 is in hole 1, then pigeon 1 is not in any hole. This violates constraint (1) and thus, we learn that pigeon 2 is not in hole 1. 


This results in an $O(n^3)$ size pigeon hole proof for the pigeonhole benchmarks. Additionally, since discovering globally blocked clauses is linear in the size of the formula and the formula is $O(n^2)$ for the pigeonhole benchmarks. The total runtime of our algorithm on pigeonhole benchmarks is $O(n^5)$

To better motivate this, we will discuss an example based on the pigeonhole problem with $5$ pigeons and $4$ holes. First, we include a visual and then explain what we have done.
% \begin{figure}[!hbt]
%   \centering
%   \subfloat[Learns the clause $\overline{x_{1, 1}}$]{\includegraphics[width=0.4\textwidth]{pigeonhole/1.png}
%   % \subcaption{}
%   }
%   \qquad
%   \subfloat[Learns the clause $\overline{x_{2, 1}}$]{\includegraphics[width=0.4\textwidth]{pigeonhole/2.png}}\\
%   \subfloat[Learns the clause $\overline{x_{3, 1}}$]{\includegraphics[width=0.4\textwidth]{pigeonhole/3.png}}
%   \qquad
%   \subfloat[Learns the clauses $\overline{x_{3, 2}}$ and $\overline{x_{4, 2}}$]{\includegraphics[width=0.4\textwidth]{pigeonhole/4.png}}\\
%   \subfloat[Learns the clause $\overline{x_{4, 3}}$]{\includegraphics[width=0.4\textwidth]{pigeonhole/5.png}}
%   \qquad
%   \subfloat[Implies the clause $x_{4, 4}$]{\includegraphics[width=0.4\textwidth]{pigeonhole/6.png}}\\
%   \subfloat[Implies $\overline{x_{1, 4}}$, $\overline{x_{2, 4}}$, $\overline{x_{3, 4}}$, $\overline{x_{5, 4}}$]{\includegraphics[width=0.4\textwidth]{pigeonhole/7.png}}
%   \qquad
%   \subfloat[Implies $x_{3, 3}$]{\includegraphics[width=0.4\textwidth]{pigeonhole/8.png}}\\
%   \subfloat[Implies $\overline{x_{1, 3}}$, $\overline{x_{2, 3}}$, $\overline{x_{5, 3}}$]{\includegraphics[width=0.4\textwidth]{pigeonhole/9.png}}
%   \qquad
%   \subfloat[Implies $x_{2, 2}$]{\includegraphics[width=0.4\textwidth]{pigeonhole/10.png}}
%   \\
%   \subfloat[Implies $\overline{x_{2, 1}}$, $\overline{x_{2, 5}}$]{\includegraphics[width=0.4\textwidth]{pigeonhole/11.png}}
%   \qquad
%   \subfloat[Contradiction, this implies both $x_{1, 1}$ and $x_{5, 1}$!]{\includegraphics[width=0.4\textwidth]{pigeonhole/12.png}}

%   \caption{An illustration of how CAutiCal learns clauses for Pigeonhole with $5$ pigeons and $4$ holes. All of these unit clauses are learned after computing an $O(n^2)$ number of globally blocked clauses}
%   \label{fig:pigeonhole}
% \end{figure}

Essentially, we learn the following sets of clauses:

We make a set of alternating decisions $x_{1, 1}, x_{2, j}$ and then $x_{1, j} x_{2, 1}$  for $j = 2, ..., 4$.

The first pair (when $j = 2$ will allow us to lead us to learn the two globally blocked clauses:

\begin{gather*}
    \textcolor{purple}{\overline{x_{3, 1}} \lor \overline{x_{4, 1}} \lor \overline{x_{5, 1}}} \lor \textcolor{purple}{\overline{x_{3, 2}} \lor \overline{x_{4, 2}} \lor \overline{x_{5, 2}}} \lor \textcolor{cyan}{\overline{x_{1, 2}}} \\
    \textcolor{purple}{\overline{x_{3, 1}} \lor \overline{x_{4, 1}} \lor \overline{x_{5, 1}}} \lor \textcolor{purple}{\overline{x_{3, 2}} \lor \overline{x_{4, 2}} \lor \overline{x_{5, 2}}} \lor \textcolor{cyan}{\overline{x_{2, 1}}} 
\end{gather*}

Then when we introduce the decision $x_{1, 2}, x_{2, 1}$, we would reach a conflict by unit propagation, and thus introduce the conflict clause $\overline{x_{1, 2}} \lor \overline{x_{2, 1}}$.

We repeat this to also obtain the conflict clauses $\overline{x_{1, 2}} \lor \overline{x_{2, 1}}$, $\overline{x_{1, 3}} \lor \overline{x_{2, 1}}$, $\overline{x_{1, 4}} \lor \overline{x_{2, 1}}$.

The globally blocked clauses combined with these conflict clauses is enough to learn the clause $\overline{x_{2, 1}}$. We repeat to learn $\overline{x_{3, 1}}, \overline{x_{4, 1}}$ and so on until we reach a contradiction. 


% \begin{example}
%     Consider the following SAT problem:
    
% \end{example}

  \section{Implementation}~\label{sec:implementation}

We implement our technique in a tool \tool (a fork of \cadical
%  commit f13d74439a5b5c963ac5b02d05ce93a8098018b8
) as a preprocessing step. By default, \tool will run the preprocessing step for
30 seconds. After this time, it will exit and commence normal solving,
regardless of whether it has found any \pr clauses. This implementation adds ??
lines of C++ code to \cadical and we believe it is implementable within any CDCL
SAT solver. We experimented with other heuristics, which we briefly describe. We
include an evaluation of the heuristics in \autoref{subsec:heuristics}.

% \subsection{Other Heuristics}~\label{subsec:heuristics}

% todo: maybe move these two paragraphs to implementation section

% \noindent \textbf{Sorting $\alpha_a$ by Implication:}%~\label{subsubsec:impordering}
% The greedy set cover algorithm described above is almost sufficient for being insensitive to heuristics, but there is one other optimization that we made. \autoref{alg:finda0} describes sorting $\alpha_a$ as an initial step. This is important for cases where there are more than one cover of the same size. The most common occurence of this is for $a_1, a_2 \in \alpha_a$, we may have that $\overline{a_2} \impunit^\phi \overline{a_1} \impunit^\phi c_1, ..., c_n$. 

% Thus, greedy set cover could pick $\{a_1\}$ or $\{a_2\}$ as singleton sets. However, since $a_1 \impunit^\phi a_2$, we really want to learn the unit clause $a_1$ since it is much more powerful.

\subsubsection{Sorting $\alpha_a$ by Implication}~\label{subsubsec:impordering}
The greedy set cover algorithm described above is almost sufficient for being
insensitive to heuristics, but there is one other optimization that we made.
\autoref{alg:finda0} describes sorting $\alpha_a$ as an initial step. This is
important for cases where there are more than one cover of the same size. The
most common occurence of this is for $a_1, a_2 \in \alpha_a$, we may have that
$\overline{a_2} \impunit^\phi \overline{a_1} \impunit^\phi c_1, ..., c_n$. 


\subsubsection{Filtering Trivial Clauses}~\label{subsubsec:filteringtriv}
Introducing globally blocked clauses can affect how we divide $\alpha = \alpha_c
\sqcup \alpha_a$ as done in \autoref{alg:leastcond}. Thus, it is better to avoid
introducing globally blocked clauses if they are not useful. We do this by
checking if $\phi \impunit C$ for each potential clause $C$. If it is implied by
unit propagation, we do not learn it since we consider it easy to learn.

\subsubsection{Filtering based on Clause Length}~\label{subsubsec:filtering-length}
Another heuristic for filtering is based on clause size. If a clause is longer
than some size $k$ we may choose not to learn it as shorter clauses are
typically more useful. This can be combined with filtering out trivial clauses
or used separately.

\subsubsection{Ordering Literals}~\label{subsubsec:ordering-literals}
So far, we are considering literals $i$ and $j$ to propagate on randomly.
However, there could be better ways to order literals. One techinque is to order
$i$ based on the number of clauses it occurs in. Another is, once an $i$ is
chosen, we pick $j$ based on if it touches $i$, i.e. if $j$ occurs in the same
clause as $i$ or a variable propagate by $i$. This was originally proposed in
Reeves et al ~\cite{prelearn}.

\subsubsection{Alternative Shrinking techniques}~\label{subsubsec:shrink-techniques}
We discussed in \autoref{subsec:shrink} that we can shrink a checking if $\phi
\impunit a \lor c$, where $a \in \alpha_a$ and $c \in \alpha_c$. While, this is
not the most expensive check, it can be made cheaper. Essentially, we just check
if there is a clause $a \lor c$ in the original formula. While, there may be
pairs $a \lor c$ that were missed, this is fast and good enough for certain
benchmarks such as the pigeonhole principle. Another alternative is to not
shrink the clause at all.

% Here is a complete list of flags discussed in 

  \section{Evaluation}~\label{sec:evaluation}

In this section, we aim to answer the following research questions:


\begin{enumerate}
    \item Does our technique outperform other SAT solvers on certain benchmark families?
    \item Does our technique underperform compared to other SAT solvers on certain benchmark families?
    \item Is our technique less sensitive to encoding choices compared to other \pr learning techniques?
\end{enumerate}

We do so by implementing our technique in a tool \tool (a fork of \cadical commit f13d74439a5b5c963ac5b02d05ce93a8098018b8). In \autoref{subsec:pigeonhole-results}, we compare \tool to standard SAT solvers as well as \pr learning techniques. We evaluate based on time taken, the length of the proof, and the sensitivity to renaming variables and reordering clauses.

Second, in \autoref{subsec:satcomp-results}, we compare \tool to other SAT solvers on the benchmarks from the annual Satisfiability competition from the years 2022, 2023, and 2024. 

Third, we analyze the use of specific heuristic choices in \tool by turning heuristics off one by one and observing the effect on the performance of \tool. This is described in \autoref{subsec:analysis-of-heuristics}.

Finally, we conclude, in \autoref{subsec:discussion}, by discussing the benchmarks families upon which our technique performs well and poorly.


\subsection{Pigeonhole results}~\label{subsec:pigeonhole-results}

\subsection{SATCOMP results}~\label{subsec:satcomp-results}

\subsection{Analysis of heuristics}~\label{subsec:analysis-of-heuristics}

\subsection{Discussion of Benchmark Families}~\label{subsec:discussion}
  \input{texs/8.conclusion}

%   \input{texs/back}
%   % \input{texs/method}
%   \input{texs/context}
%   \input{texs/shake}
%   \input{texs/eval}
%   \input{texs/related}
%   \input{texs/limit}
%   \input{texs/conclude}

  
  \newpage
  
  \clearpage
  
  \bibliographystyle{IEEEtran}
  \bibliography{base}
  
  % \input{texs/appendix}
  
  \end{document}
  