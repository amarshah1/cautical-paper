% \begin{filecontents*}{\jobname.xmpdata}
%     \Title{Context Pruning for More Robust SMT-based Program Verification}
%     \Author{Yi Zhou\sep Jay Bosamiya\sep Jessica Li\sep Marijn \sep Bryan Parno}
%     \Publisher{TU Wien Academic Press}
%   \end{filecontents*}
  
  \PassOptionsToPackage{dvipsnames}{xcolor}
  \documentclass[year=25,pdfa]{fmcad}
  
  \usepackage{cite}
  \usepackage{amsmath,amssymb,amsfonts,amsthm}
  % \usepackage{algorithmic}
  \usepackage{graphicx}
  \usepackage{textcomp}
  \usepackage[dvipsnames]{xcolor}
  \usepackage{bussproofs}  % For natural deduction proofs with \inferrule
  \usepackage{mathpartir}
  \usepackage{orcidlink}
  \usepackage{booktabs}
  \usepackage{siunitx}
  % \usepackage{appendix}

  
  % \usepackage[margin=1in]{geometry}    % Exert more control over the margins
  % \usepackage{mathptm}     % Puts math text into the postscript Times font and symbol font (else uses the metafont-generated characters)
  % \usepackage{amsmath}     % Provides various features to facilitate writing math formulas and to improve the typographical quality of their output
  \usepackage{float}       % Basic package for floating objects like figures and tables
  \usepackage{placeins}    % Provides \FloatBarrier for better float control
  % \usepackage{psfrag}      % Used to insert text into figures
  % \usepackage{floatflt}    % Used to wrap text around figures
  % \usepackage{graphicx}    % Used to display graphic figures and such
  \usepackage{xspace}      % Intelligently adds space after a word via \xspace
  % \usepackage{color}       % Used to highlight comments
  % \usepackage{pifont}      % Provides the ding symbol used for comments
  % \usepackage{subfigure}   % Place two related figures side-by-side
  % \usepackage{url}         % Properly formats URLs
  % \usepackage{array}       % Add more space to the tables
  % \usepackage{algorithm}   % Defines algorithm float environment
  % \usepackage{algpseudocode}
  \usepackage{multirow}    % Use to create table with a column that spans multiple rows
  \usepackage{enumerate}   % Control the numbering for enumerated lists
  \usepackage[square,comma,numbers,sort&compress]{natbib}      % Offers more control over citation appearance and bib spacing
  % \usepackage{enumitem}
  \usepackage{diagbox}
  % \usepackage{makecell}
  \usepackage{booktabs}
  % \usepackage[T1]{fontenc}
  \usepackage[noend]{algpseudocode}
  % \usepackage{algpseudocode}
  % \usepackage{authblk}
  \usepackage{pifont}
  % \usepackage[switch]{lineno} % Line numbering for draft-mode
  % \usepackage{tikz}
  % \usetikzlibrary{shapes.geometric, arrows}
  % \usetikzlibrary{decorations.pathreplacing,calligraphy}
  \usepackage{tikz}
  \usetikzlibrary{matrix}

  % ------------------------------------------------------------------
  % Dot grid styles 
  \newcommand*\dotdiam{15pt}      % <— tweak this for bigger / smaller circles
  \newcommand*\dotgap {4pt}      % <— negative → dots touch; 0pt = tangent

  \tikzset{
    cell/.style   = {draw,circle,fill=gray!60,
                    minimum size=\dotdiam, inner sep=0pt, outer sep=0pt,
                    text height=1.5ex, text depth=0.5ex},
    redcell/.style= {cell,fill=red!60},
    redpluscell/.style= {cell,fill=red!60,execute at begin node={\raisebox{-1pt}{\Large +}}},
    redminuscell/.style= {cell,fill=red!60,execute at begin node={\raisebox{-1pt}{\Large -}}},
    greencell/.style  = {cell,fill=green!60},
    greenpluscell/.style  = {cell,fill=green!60,execute at begin node={\raisebox{-1pt}{\Large +}}},
    greenminuscell/.style  = {cell,fill=green!60,execute at begin node={\raisebox{-1pt}{\Large -}}},
    graycell/.style = {cell,fill=gray!40},
    graypluscell/.style = {cell,fill=gray!40,execute at begin node={\raisebox{-1pt}{\Large +}}},
    grayminuscell/.style = {cell,fill=gray!40,execute at begin node={\raisebox{-1pt}{\Large -}}},
    bluecell/.style = {cell,fill=blue!20},
    bluepluscell/.style = {cell,fill=blue!20,execute at begin node={\raisebox{-1pt}{\Large +}}},
    blueminuscell/.style = {cell,fill=blue!20,execute at begin node={\raisebox{-1pt}{\Large -}}},
    yellowcell/.style = {cell,fill=yellow!60},
    yellowpluscell/.style = {cell,fill=yellow!60,execute at begin node={\raisebox{-1pt}{\Large +}}},
    yellowminuscell/.style = {cell,fill=yellow!60,execute at begin node={\raisebox{-1pt}{\Large -}}},
    purplecell/.style = {cell,fill=purple!60},
    purplepluscell/.style = {cell,fill=purple!60,execute at begin node={\raisebox{-1pt}{\Large +}}},
    purpleminuscell/.style = {cell,fill=purple!60,execute at begin node={\raisebox{-1pt}{\Large -}}},
    orangecell/.style = {cell,fill=orange!60},
    orangepluscell/.style = {cell,fill=orange!60,execute at begin node={\raisebox{-1pt}{\Large +}}},
    orangeminuscell/.style = {cell,fill=orange!60,execute at begin node={\raisebox{-1pt}{\Large -}}},
    gridmatrix/.style = {
        matrix of nodes,
        nodes   = {cell},
        column sep=\dotgap, row sep=\dotgap,
        % ampersand replacement=\&,                      % enable & inside matrix
        nodes in empty cells,                          % draw every entry
    },
  }
  \usepackage{subcaption}
  
% }
  \usepackage{outlines}
  \usepackage{listings}
  % \usepackage[caption=false]{subfig}
  \usepackage{soul}
  \usepackage{microtype}
  
  \def\BibTeX{{\rm B\kern-.05em{\sc i\kern-.025em b}\kern-.08em
      T\kern-.1667em\lower.7ex\hbox{E}\kern-.125emX}}
  
  % Used for displaying a sample figure. If possible, figure files should
  % be included in EPS format.
  %
  % If you use the hyperref package, please uncomment the following line
  % to display URLs in blue roman font according to Springer's eBook style:
  % \renewcommand\UrlFont{\color{blue}\rmfamily}
  
  
  \usepackage[linesnumbered,ruled,vlined]{algorithm2e}
  % \SetAlCapSkip{1em} % or 10pt, 12pt, etc.
  \usepackage{hyperref}
  \hypersetup{
  %\ifpdf
  %        pdftex,
  %\else
  %        dvipdf,
  %\fi
    %pdftex,              % using ps2pdf vs pdftex
    %ps2pdf,              % using ps2pdf vs pdftex
          %hypertex,
    colorlinks=true,     % color the words instead of use a colored box
    urlcolor=blue,       % \href{...}{...} external (URL)
  %  filecolor=blue,      % \href{...} local file
    linkcolor=blue,     % \ref{...} and \pageref{...}
    citecolor=blue,     % \cite{}
  % letterpaper=true,
    plainpages=false,
  %    plainpages          boolean         true
  %    Forces page anchors to be named by the arabic form of the page number,
  %    rather than the formatted form.
    breaklinks=true,
  %    breaklinks          boolean         false
  %    Allows link text to break across lines; since this cannot be accommodated in
  %    PDF, it is only set true by default if the pdftex driver is used. This makes
  %    links on multiple lines into different PDF links to the same target.
  % pagebackref=true,
  %     Adds ?backlink? text to the end of each item in the bibliography, as a list
  %     of section numbers. This can only work properly if there is a blank line
  %     after each \bibitem.
    bookmarksnumbered=true,
  %    bookmarksnumbered   boolean         false
  %    If Acrobat bookmarks are requested, include section numbers.
    bookmarksopen=true,
  %    bookmarksopen       boolean         false
  %    If Acrobat bookmarks are requested, show them with all the subtrees expanded.
    pdftitle={},
    pdfauthor={},
    pdfkeywords={},
  % pdfpagelabels=true,
    pdfpagemode=UseOutlines  % None, UseThumbs, UseOutlines, FullScreen
  }

  \renewcommand{\sectionautorefname}{Section}
  \renewcommand{\subsectionautorefname}{Subsection}
  \renewcommand{\algorithmautorefname}{Algorithm}
  \newcommand*{\definitionautorefname}{Definition}


  
  \usepackage[capitalise]{cleveref}
%   \usepackage{stys/dafny}
%   \usepackage{stys/smtlib}
  % \usepackage{array}
  \usepackage{tabularx}
  \usepackage{enumitem}
  
%   \input{00-common-preamble.tex}
  
  % \algrenewcommand\algorithmicindent{0.65em}%

  \newtheorem{definition}{Definition}
  \newtheorem{theorem}{Theorem}
  \newtheorem{lemma}{Lemma}
  \newtheorem{claim}{Claim}


  
  \newcommand{\unsat}{\texttt{unsat}\xspace}
  \newcommand{\sat}{\texttt{sat}\xspace}
  \newcommand{\tool}{CA\textsc{uti}C\textsc{aL}\xspace}
  \newcommand{\toolminus}{$\text{CA}\textsc{uti}\text{C}\textsc{aL}^{-}$\xspace}
  \newcommand{\pr}{\textsf{PR}\xspace}
  \newcommand{\er}{\texttt{ER}\xspace}
  \newcommand{\impunit}{\vdash_1}
  \newcommand{\impunitclause}[3]{#1 \land \overline{#2} \impunit #3}
    \newcommand{\impunitclauseNoNeg}[3]{#1 \land #2 \impunit #3}
  \newcommand{\lingeling}{\textsc{lingeling}\xspace}
%  \newcommand{\cadical}{{\sf CaDiCaL}\xspace}
  \newcommand{\cadical}{C\textsc{a}D\textsc{i}C\textsc{aL}\xspace}
  \newcommand{\sadical}{S\textsc{a}D\textsc{i}C\textsc{aL}\xspace}
  \newcommand{\glucoser}{G\textsc{lucos}ER\xspace}
  \newcommand{\prelearn}{PR\textsc{e}L\textsc{earn}\xspace}
  \newcommand{\ph}[1]{\ensuremath{\mathrm{PHP}(#1)}}
  
    \newcommand{\kissat}{\textsc{Kissat}\xspace}
    \newcommand{\cryptoMiniSAT}{\textsc{CryptoMiniSAT}\xspace}


  \newcommand{\impunitclause}[3]{$#1 \impunit #2 \lor #3$} 
  
  \begin{document}
  \title{Learning Short Clauses via Conditional Autarkies}
  
  % \author{TODO: add authors and orcid \\
  \author{Amar Shah \orcidlink{0009-0008-8282-2142}, Twain Byrnes \orcidlink{0009-0000-7939-1195}, Joseph Reeves \orcidlink{0000-0002-4585-0565}, Marijn J.\ H.\ Heule \orcidlink{0000-0002-5587-8801} \\
  \emph{Carnegie Mellon University, Pittsburgh, PA, USA} \\
  \texttt{\small \{amarshah,binarynewts,jereeves,marijn\}@cmu.edu}}
  
  \maketitle
  
  \begin{abstract}
    State-of-the-art Boolean satisfiability (SAT) solvers increasingly use
    techniques beyond resolution. One of the strongest such techniques is
    Propagation Redundancy (\pr) learning. Solvers utilizing \pr learning may
    admit short proofs for benchmark families with exponentially large
    resolution proofs, including pigeonhole and mutilated chessboard. However,
    existing \pr clause learning techniques require an NP-hard check; hence, they
    are computationally expensive and difficult to add to existing tools.

    We propose a new technique for learning \pr clauses based on conditional
    autarkies and implement it in the SAT solver \cadical. Our method is
    modular, allowing for cross-solver compatibility, and learns \pr clauses in
    linear time. Additionally, we introduce a number of heuristics, including a
    clause-shrinking technique and filtering to avoid trivial \pr clauses,
    ensuring that our method learns useful clauses. We show that this is
    competitive with state-of-the-art \pr clause learning techniques, and
    improves performance on a portion of SAT competition benchmarks.

  \end{abstract}
    
  % \begin{IEEEkeywords}
  % component, formatting, style, styling, insert
  % \end{IEEEkeywords}
  
  \section{Introduction}~\label{sec:intro}


Boolean satisfiability (SAT) solving is a core tool in computer science with
applications in program
verification~\cite{BillionQueries,sat-hardwareverification,ic3,bmc},
planning~\cite{planning,planningassat}, cryptography~\cite{cryptominisat}, and
mathematics~\cite{chromaticnumber,pythagoreantriples,kellersconjecture,emptyhexagon}.
As its usage expands, so too does the need for more powerful and specialized
solving techniques.



% It is most commonly solved using the conflict-driven clause learning (CDCL)
% framework. This framework explores the state space of possible solutions,
% while learning clauses when it reaches a conflict, to aid future explorations.
One such class of techniques is propagation redundant (\pr) clause
learning~\cite{prclauses}. In this technique, a solver generates and adds
clauses which are satisfiability-preserving; that is, clauses whose conjunction
with the formula is satisfiable if and only if the original formula is
satisfiable. The solver aims to learn clauses that drastically shrink the
potential solution space, such as clauses that break symmetries.

%A classic problem involving symmetries is the pigeonhole problem, which asks if
%it is possible to fit $n+1$ pigeons into $n$ holes with at most one pigeon per
%hole. In this problem, one may learn a clause equivalent to the lemma:
%\emph{pigeon 1 is not in hole 1}. Adding this clause restricts the search space
%for the first pigeon (and thus the problem), but it preserves satisfiability,
%as hole 1 is symmetric with all holes.

% One can ensure that these clauses are satisfiable within the given proof
% system.

% for instance vivification and blocked clause addition. These techinques use
% other facts about the formula to learn clauses which are often useful for
% solving.

The power of this technique is tied to the strength of the underlying proof
system. A more powerful proof system can admit shorter proofs, which can be
found and checked quicker. Most SAT solvers rely on resolution-based proof
systems, which are ineffective for many hard instances. Consider the pigeonhole
principle (PHP), which asks if it is possible to fit $n+1$ pigeons into $n$
holes with at most one pigeon per hole. With a linear number of \pr reasoning
steps, one may learn a \pr clause equivalent to the lemma: \emph{pigeon 1 is not
in hole 1}. Adding this clause restricts the search space for the first pigeon
(and thus the problem), but it preserves satisfiability, as hole 1 is symmetric
with all holes. 

It is difficult to perform similar reasoning compactly with resolution, and in
fact, PHP is known to require exponentially large resolution
proofs~\cite{hakenpigeonhole}. Proof systems
based on \pr clauses, however, have short proofs for such problems, including a
cubic proof (in the number of pigeons) for PHP~\cite{prclauses}. PHP is especially
important as it occurs frequently as a subproblem for many different SAT benchmarks.

The theoretical power of \pr learning comes at a cost, and efficiently learning
useful \pr clauses is a challenge. State-of-the-art techniques such as
Satisfiability Driven Clause Learning (SDCL) rely on calling another SAT solver
to verify that a candidate clause is \pr \cite{sadical}. 
% This is both expensive and difficult to integrate into existing tools. 
Thus, in the worst case, the solver takes exponential time to learn a single \pr
clause. This makes integration into high-performance solvers difficult and
limits their practical impact. To the best of our knowledge, none of the most popular SAT solvers
such as \cadical~\cite{cadical}, \kissat~\cite{kissat},
\cryptoMiniSAT~\cite{cryptominisat}, or \lingeling~\cite{lingeling} support
\pr clause learning in their main branch.
% maybe cite cryptominisat instead of lingeling? todo : check crypto mini sat to
% see if it uses \pr clauses

% todo : should we explain what conditional autarkies are? or go more into depth
% about globally blocked clauses?
In this work, we propose a new, more efficient approach to learn \pr clauses
based on conditional autarkies. Intuitively, a conditional autarky provides a
way to force the value of certain variables given a set of assumptions, e.g., in
PHP if pigeons $3 $ to $ n+1$ are not in hole $1$ or hole $2$ (see \autoref{subfig:pigeonholeclause-c}
for a visual representation), then pigeon $1$
can be placed in hole $1$ and pigeon $2$ can be placed in hole $2$. Kiesl et.
al. proposed an algorithm for finding conditional autarkies in linear
time~\cite{conditionalautarkies}, using them to delete clauses. In our work, we
will use conditional autarkies to construct and add \pr clauses. 
%We make use of an algorithm proposed by Kiesl et. al. for finding a conditional
%autarky given a assignment in linear time. 
%
%Given a assignment to a SAT formula, our method divides the assignment
%into a \emph{conditional} part and an \emph{autarky} part. Using this
%distinction, we can learn a globally blocked clause (a type of \pr clause) in
%linear time.

While conditional autarkies present a means for bridging the theoretical gap in
identifying \pr clauses, the \pr clauses they produce are not immediately useful
in real SAT solving applications. There are two major practical limitations: (1)
larger \pr clauses may not meaningfully constrain the search space, and (2) some
smaller \pr clauses may be trivial and distract the solver. We solve (1) by
introducing a shrinking procedure to extract compact, useful \pr clauses, and
(2) by introducing a number of heuristics for filtering away trivial \pr
clauses. Returning to PHP, instead of the conditional autarky described above
which produces a clause with at least $2n$ literals, it is possible through
shrinking to learn a binary \pr clause either forbidding pigeon $1$ in hole $2$ or
pigeon $2$ in hole $1$.
% todo: would like to say more about what these heuristics are

% Integrating \pr clause learning into a SAT solver is still an active area of
% study

To summarize, we make the following contributions: 

\begin{enumerate} 
    \item We introduce a method for learning \pr clauses in linear time in the
    size of the formula. 
    \item We develop a number of shrinking and filtering heuristics, to learn
    concise and useful \pr clauses.% from conditional autarkies.. 
    \item We implement these techniques in a solver, \tool\footnote{\tool's code
    is available at \url{https://github.com/amarshah10/cautical}.} (a fork of
    the state-of-the-art SAT solver \cadical), and evaluate it on pigeonhole 
    principle benchmarks and a suite of benchmarks from the annual SAT competitions.
\end{enumerate}
  \section{Background}~\label{sec:background}

We begin with some SAT preliminaries. Variables $x_1, x_2, ...$ take values
\emph{true} ($\top$) or \emph{false} ($\bot$). A literal $l$ is a variable
$x$ or its negation $\overline{x}$. A clause $C$ is a disjunction of literals,
e.g., $C = x_1 \lor \overline{x}_4 \lor x_6$. A clause may also be represented by the set of its literals:
the prior clause $C = \{x_1, \overline{x}_4, x_6\}$. This motivates denoting that a literal
$l$ occurs in clause $C$ via $l \in C$. A conjunctive normal form (CNF)
formula $\phi$ is a conjunction of (disjunctive) clauses. In this paper,
all formulas $\phi$ are CNF.

We use $var(\phi)$ to represent the set of variables occurring in formula $\phi$.
A partial assignment $\alpha : V \rightarrow \{\top, \bot\}$ on $\phi$
maps variables $V \subseteq var(\phi)$ to true or false. If $V
= var(\phi)$, then $\alpha$ is a (total) assignment. We abuse notation by
denoting $\alpha(l) = \alpha(x)$ if $l = x$ and $\alpha(l) = \neg \alpha(x)$ if
$l = \overline{x}$.

A formula restricted to a partial assignment $\phi|_\alpha$ is the formula resulting from
mapping variables in the domain of $\alpha$ to their assignments. We can
simplify such a formula by removing \emph{literals} assigned false and
\emph{clauses} including a literal assigned true (and thus satisfied).
For example, with formula $\phi =
(x_1 \lor \overline{x}_4 \lor x_6) \land (x_2 \lor x_3) \land \overline{x}_5$
and assignment $\alpha$ mapping $x_1$ to $\bot$ and $x_2$ to $\top$,
we have $\phi|_\alpha = (\overline{x}_4 \lor x_6) \land \overline{x}_5$.

A clause $C = l_1 \lor \dots \lor l_m$ is \emph{blocked} by an assignment
that maps each $l_i \in C$ to false ($\bot$). An assignment $\alpha$
\emph{touches} a clause $C$ if there is a variable $x$ in the domain of $\alpha$
such that $x \in C$ or $\overline{x} \in C$. We say $\alpha$ \emph{satisfies}
$C$ if $\alpha(l) = \top$ and $l \in C$.

Assignment $\alpha$ \emph{satisfies} formula $\phi$ if it satisfies every
clause $C \in \phi$. If there exists such a satisfying assignment, then $\phi$ is
\emph{satisfiable}. The boolean satisfiability problem asks whether a given
formula $\phi$ is satisfiable.

\subsection{Redundant Clauses}~\label{subsec:redundant}

A clause $C$ is \emph{redundant} (or \emph{satisfiability-preserving}) with
respect to a formula $\phi$ if the formulas $\phi$ and $\phi \land C$ are
\emph{equisatisfiable}, i.e. $\phi$ is satisfiable if and only if $\phi \land C$
is satisfiable.

In a \emph{clausal proof system}, each step will add or remove a redundant
clause $C$. The step may contain extra information, such as a boolean witness,
justifying why $C$ is redundant. A list of redundant clauses ending with the
empty clause $\bot$ is a proof of unsatisfiability for formula $\phi$.

% If $\phi$ is unsatisfiable, then any clause $C$ will be redundant. However, when solving, we do not know whether $\phi$ is satisfiable a priori. Clausal proof systems can justify that a clause is redundant.

Adding redundant clauses may be helpful in certain cases, since adding a clause
will constrain the set of possible solutions. However, it could be harmful as
new clauses may negatively interact with solver heuristics.

% \subsection{Reasoning Steps}~\label{subsec:reasoning-steps}

\emph{Resolution} is the main proof step in a clausal proof system. It states that two clauses $C \lor x$ and $\overline{x} \lor D$ and returns a new clause $C \lor D$.

\begin{equation*}
    \inferrule*[Right=Resolution]{
        C \lor x \\ \overline{x} \lor D
    }{
        C \lor D
    }    
\end{equation*}

\emph{Unit propagation} is a core reasoning techniques in a SAT solver. Starting with empty partial assignment $\alpha$, if a formula $\phi$, contains a \emph{unit clause} $l$, i.e. a clause with only a single literal, then we set $\alpha(l) = \top$. Then, this unit is propagated, i.e. we consider the formula $\phi|_\alpha$ and continue this process until there are no unit clauses remaining. When unit propagation terminates with $\phi|_\alpha = \bot$, we say that it derives a conflict.
% todo: notation issue: I never defined \bot as the empty clause and I also am using \bot both as false and the empty clause.

Unit propagation can be thought of as a proof step. Indeed, it can be thought of as a mechanized application of resolution.

\begin{equation*}
    % \inferrule*[Right=Unit-Pos]{
    %     C \lor l \\ l
    % }{
    %     \bot
    % }
    % \qquad \qquad \qquad
    \inferrule*[Right=Unit-Prop]{
        C \lor l \\ \overline{l}
    }{
        C
    }
\end{equation*}

Given a formula $\phi$ and clause $C = l_1 \lor ... \lor l_k$, we can say $\phi \impunit C$, i.e. ``$\phi$ implies $C$ via unit propagation,'' if $\phi \land \overline{l}_1 \land \dots \land \overline{l}_k$ derives a conflict. Here, $C$ is a simple example of a redundant clause w.r.t. $\phi$. 

% This can be justified in almost any proof system for propositional logic. However, this is not very useful as it rules out







For two formulas, we say $\phi \impunit \psi$ if for every clause $C \in \psi$, $\phi \impunit C$. 
% For literals $l_1$ and $l_2$, we sometimes notate $l_1 \impunit^\phi l_2$ to mean $\phi \vdash_1 \overline{l_1} \land l_2$. Informally, we can think of this as saying ``$l_1$ implies $l_2$ via unit propagation on $\phi$,'' which means that .
% see definition here: https://www.cs.cmu.edu/~mheule/publications/prencode.pdf
% However, this is not very useful as it is quiet easy for the solver to figure this out. Instead, we must appeal to a more complicated notion of redundancy.

% The propagation redundant (\pr)

\subsection{Propagation Redundant}~\label{subsec:pr}

While resolution is complete for propositional logic, more powerful proof steps can yield shorter proofs and faster runtimes. One such step is based on \pr clauses.

\begin{definition}[Propagation Redundant (\pr) clauses]
    For formula $\phi$, we say that clause $C$ (blocked by $\beta$) is propagation redundant (\pr) w.r.t. $\phi$ if there exists an assignment $\omega$ known as the witness such that $F|_\beta \impunit F|_\omega$ and $\omega$ satisfies $C$
\end{definition}

Such a clause $C$ must be redundant. Say there is a satisfying assignment $\alpha$ for $\phi$. Then if $\alpha$ is not a satisfying assignment for $\phi \land C$, then it must be that $\beta \subseteq \alpha$, i.e. $\alpha$ extends $\beta$. However, since $F|_\beta \impunit F|_\omega$, it must be that any assignment that satisfies $F|_\beta$, will satisfy $F|_\omega$. 

Then we can define $\alpha' (x) = \omega(x)$ if $x \in \omega$ and $\alpha'(x) = \alpha(x)$ otherwise. Thus, $\alpha'$ satisfies $\phi$ and $\alpha'$ satisfies $C$.

% Intuitively, we can think of adding $C$ as the constraint that prunes all assignments that extend $\beta$. Since $F|_\beta \impunit F|_\omega$, it must be that any assignment that satisfies $F_\beta$, will satisfy $F_\omega$.

% Additionally, since $\omega$ satisfies $C$, removing the assignments that extend $\beta$, will not affect satisfiability.
% % todo : this needs ot be explained better.

% If $C$ is a \pr clause w.r.t $\phi$, then $\phi$ and $\phi \land C$ are equisatisfiable. Indeed, the clause defined in \autoref{thm:gbcequisat} is \pr with witness $\alpha_a$. 

However, checking if a clause is \pr is NP-complete~\cite{prclause}, so witnesses must be provided for proof checking. \pr clauses subsume many classes of redundant clauses, including resolution asymmetric tautologies (RATs)~\cite{rat}, blocked clauses~\cite{blockedclause}, set-blocked clauses~\cite{setblocked}, and globally-blocked clauses~\cite{conditionalautarkies}.

\subsection{Autarkies and Globally Blocked Clauses}~\label{subsec:autarkies}

% A key realization in prior work~\cite{}

\begin{definition}[Autarky]
    A nonempty assignment $\alpha$ is an autarky for a formula $\phi$ if every clause $C \in \phi$ touched by $\alpha$ is satisfied.
\end{definition}

% An autarky $\alpha_a = a_1, ..., a_m$ can be very useful as $a_1 \land ... \land a_m$ is an equisatisfiable clause. However, these can be difficult to find.

In plain words, an autarky is an assignment that satisfies every clause it touches. For example, if $\alpha$ was a satisfying assignment, it would be an autarky since it satisfies every clause.

\begin{definition}[Conditional Autarky]
    A nonempty assignment $\alpha = \alpha_c \sqcup \alpha_a$ (disjoint union) is a conditional autarky for a formula $\phi$ if $\alpha_a$ is an autarky for $\phi|_{\alpha_c}$.
\end{definition}

Specifically, we can think about searching for conditional autarkies by first looking for a partial assignment $\alpha_c$, and then finding an autarky $\alpha_a$ on the reduced formula $\phi|_{\alpha_c}$.

Conditional autarkies can be very useful for learning equisatisfiable clauses, for instance, if $\alpha_c = c_1, ..., c_n$ and $\alpha_a = a_1, ..., a_m$, we add clauses of the form:

\begin{equation*}
    [c_1 \land ... \land c_n] \rightarrow [a_1 \land ... \land a_m]
\end{equation*}

This results in $m$ different clauses as in the following theorem from Kiesl et al.~\cite{conditionalautarkies}:

\begin{theorem}~\label{thm:gbcequisat}
    Formula $\phi$ and $\phi \land \bigwedge_{1 \leq i \leq m} (\overline{c_1} \lor ... \overline{c_n} \lor a_i)$ are equisatisfiable.
\end{theorem}

This means that $\phi$ is satisfiable if and only if $\phi \land \bigwedge_{1 \leq i \leq m} (\overline{c_1} \lor ... \overline{c_n} \lor a_i)$ is satisfiable. Thus, the solver can add any of the $m$ clauses $\overline{c_1} \lor ... \overline{c_n} \lor a_i$ and preserve satisfiability. Each of these clauses is a \pr clause with witness $\alpha_a$. Each of these clauses is a \emph{globally blocked clause}, however, this is not too important for our purposes, so we will not discuss it further.
% Additionally, we can see that this is a \pr clause:

% todo: this theorem works for us because of the algorithm we use to prove this, but it is not true in the general case.

% \begin{theorem}~\label{thm:totalassignmnet}
%     For $1 \leq i \leq m$, the clause $C = \overline{c_1} \lor ... \overline{c_n} \lor a_i$ is a \pr clause w.r.t $\phi$ with witness $\alpha_a$
% \end{theorem}

% \begin{proof}
%    Since $a_i \in \alpha_a$, $\alpha_a$ satisfies $C$.
   
%    Define the assignment that blocks $C$ as $\beta = c_1, ..., c_n, \overline{a_i}$. Now we want to show that: $F|_\beta \impunit F|_{\alpha_a}$. Take a clause $C \in F|_{\alpha_a}$. We know that
% \end{proof}




\subsection{Related Work}~\label{subsec:relatedwork}

The pigeonhole problem asks if $n+1$ pigeons can fit into $n$ holes. The
mutilated chessboard problem asks if a $2n \times 2n$ board with two diagonally
opposite corners removed can be tiled with $2 \times 1$ rectangular dominoes.
Both are unsatisfiable and it is easy for a human to see why. However, it has
been shown there are no polynomial-sized resolution proofs for either
problem~\cite{hakenpigeonhole,mutilatedchessboard-exponential}.
% todo: need to clarify if I should say that there are not polynomial-sized resolution proofs for these problems or if there are no sub-exponential-sized resolution proofs for these problems.

% todo: need to be consistent about whether I am using terminology "pigeonhole principle" or "pigeonhole problem"

% todo: I need to cite some paper that introduces pigeonhole and mutilated chessboard


 While extended resolution (\er) provided $O(n^4)$ for the pigeonhole problems, the proof system involved introducing new variables~\cite{er}. In general, the search space for new variables is infinite and thus tools like \glucoser based on \er did not scale well~\cite{glucoser}.

The \pr proof system remedies this by producing $O(n^3)$ proofs for the pigeonhole formula~\cite{prclauses} without learning any new variables. Later, this was shown to produce short proofs for mutilated chessboard problems~\cite{mutilatedchessboard-pr}. 
% todo we use O(n^_) where n is formula, but also where n is the number of variables.

Solvers that implement the \pr proof system typically use the satisfaction-driven clause learning (SDCL) framework~\cite{sdcl}, which extends conflict-driven clause learning (CDCL)~\cite{cdcl}. 
% Here, if unit propagation does not lead to a conflict, then they attempt to learn a \pr clause instead.
% \pr clause learning was first introduced in an extension of the solver \lingeling~\cite{prclause}.
After propagating an assignment, they would check if the clause $C$ that was blocked by this assignment was \pr. This was done by creating a new SAT formula called the \emph{positive reduct}. If the positive reduct was satisfiable, then $C$ is a \pr clause and it was added. This was implemented in an extension of the solver \lingeling and was shown to scale well on pigeonhole benchmarks.

Later, two new variants of the positive reduct were proposed for more aggressive pruning of the search space \cite{sadical}. This allowed SDCL to solve other difficult problems such as Tseitin formulas~\cite{hardexamplesresolution}. This was implemented in a new SDCL solver \sadical.


\prelearn is a preprocessing technique for \pr clauses~\cite{prelearn}. It initially considers many possible clauses and queries \sadical to see which were \pr.

Our work differs as we do not use a \emph{positive reduct} to test if a clause
is \pr. Instead, our clauses \pr by construction as they come from a conditional
autarky. This has the potential downside that our clause may be very large. To
remedy this we apply a shrinking technique to reduce the size of a clause.
Additionally, prior techniques are sensitive to the encoding of the problem.
Minor changes such as literal and clause reordering can tank performance. We
provide two \emph{symmetry hardening} techniques to handle reordering. We
compare our implementation \tool to \sadical and \prelearn in
\autoref{sec:evaluation}.

Kiesl et al.~\cite{conditionalautarkies} first introduced conditional autarkies to identify globally blocked clauses. They aimed to eliminate globally blocked clauses from a formula. This technique allowed them to simulate circuit-simplification techniques. Our work differs from this as we add clauses instead of removing them and we target a different class of benchmarks.

  \section{Methodology}~\label{sec:method}

\subsection{Motivating Example}~\label{sec:motivatex}

\begin{figure*}[!t]
    \centering
    \begin{subfigure}[t]{0.23\textwidth}
    \centering
    \begin{tikzpicture}
      % Draw the 4×5 grid and name it "M"
      \matrix (M) [gridmatrix] {
        % row 1
        |[bluepluscell]| & |[graycell]| & |[graycell]| & |[graycell]| \\
        % row 2
        |[graycell]|  & |[bluepluscell]| & |[graycell]| & |[graycell]| \\
        % row 3
        |[graycell]| & |[graycell]| & |[graycell]| & |[graycell]| \\
        % row 4
        |[graycell]| & |[graycell]| & |[graycell]| & |[graycell]| \\
        % row 5
        |[graycell]| & |[graycell]| & |[graycell]| & |[graycell]| \\
      };

      % Column labels: attach ABOVE the north anchor of row 1 cells
      \foreach \c [count=\i from 1] in {$h_1$,$h_2$,$h_3$,$h_4$} {
        \node[anchor=south] at (M-1-\i.north) {\c};
      }

      % Row labels: attach LEFT of the west anchor of column 1 cells
      \foreach \r [count=\j from 1] in {$p_1$,$p_2$,$p_3$,$p_4$,$p_5$} {
        \node[anchor=east] at (M-\j-1.west) {\r};
      }
    \end{tikzpicture}
    \captionsetup{width=0.8\textwidth}  % override for this one only
    \caption{We initially decide on $x_{1, 1}$ and $x_{2, 2}$.}~\label{subfig:pigeonholeclause-a}  
\end{subfigure}
\begin{subfigure}[t]{0.23\textwidth}
    \centering
    \begin{tikzpicture}
        % Draw the 4×5 grid and name it "M"
        \matrix (M) [gridmatrix] {
        % row 1
        |[bluepluscell]| & |[yellowminuscell]| & |[graycell]| & |[graycell]| \\
        % row 2
        |[yellowminuscell]|  & |[bluepluscell]| & |[graycell]| & |[graycell]| \\
        % row 3
        |[yellowminuscell]| & |[yellowminuscell]| & |[graycell]| & |[graycell]| \\
        % row 4
        |[yellowminuscell]| & |[yellowminuscell]| & |[graycell]| & |[graycell]| \\
        % row 5
        |[yellowminuscell]| & |[yellowminuscell]| & |[graycell]| & |[graycell]| \\
        };

        % Column labels: attach ABOVE the north anchor of row 1 cells
        \foreach \c [count=\i from 1] in {$h_1$,$h_2$,$h_3$,$h_4$} {
            \node[anchor=south] at (M-1-\i.north) {\c};
        }

        % Row labels: attach LEFT of the west anchor of column 1 cells
        \foreach \r [count=\j from 1] in {$p_1$,$p_2$,$p_3$,$p_4$,$p_5$} {
            \node[anchor=east] at (M-\j-1.west) {\r};
        }
    \end{tikzpicture}
    \captionsetup{width=0.8\textwidth}  % override for this one only
    \caption{Then unit propagation gives us $\overline{x}_{2,1}, \ldots, \overline{x}_{5,1},$ $\overline{x}_{1, 2}, \overline{x}_{3, 2}, \ldots, \overline{x}_{5, 2}$.}~\label{subfig:pigeonholeclause-b} 
\end{subfigure}
\begin{subfigure}[t]{0.23\textwidth}
    \centering
    \begin{tikzpicture}
    % Draw the 4×5 grid and name it "M"
    \matrix (M) [gridmatrix] {
        % row 1
        |[orangepluscell]| & |[orangeminuscell]| & |[graycell]| & |[graycell]| \\
        % row 2
        |[orangeminuscell]|  & |[orangepluscell]| & |[graycell]| & |[graycell]| \\
        % row 3
        |[purpleminuscell]| & |[purpleminuscell]| & |[graycell]| & |[graycell]| \\
        % row 4
        |[purpleminuscell]| & |[purpleminuscell]| & |[graycell]| & |[graycell]| \\
        % row 5
        |[purpleminuscell]| & |[purpleminuscell]| & |[graycell]| & |[graycell]| \\
    };

    % Column labels: attach ABOVE the north anchor of row 1 cells
    \foreach \c [count=\i from 1] in {$h_1$,$h_2$,$h_3$,$h_4$} {
        \node[anchor=south] at (M-1-\i.north) {\c};
    }

    % Row labels: attach LEFT of the west anchor of column 1 cells
    \foreach \r [count=\j from 1] in {$p_1$,$p_2$,$p_3$,$p_4$,$p_5$} {
        \node[anchor=east] at (M-\j-1.west) {\r};
    }
    \end{tikzpicture}
    \captionsetup{width=0.8\textwidth}  % override for this one only
    \caption{Literals $\overline{x}_{3, 1}, \ldots, \overline{x}_{5, 1},$
    $\overline{x}_{3, 2}, \ldots, \overline{x}_{5, 2}$ are in $\alpha_c$ by
    their respective constraint (1). The rest are in
    $\alpha_a$.}~\label{subfig:pigeonholeclause-c}
\end{subfigure}
\begin{subfigure}[t]{0.23\textwidth}
    \centering
    \begin{tikzpicture}
    % Draw the 4×5 grid and name it "M"
    \matrix (M) [gridmatrix] {
        % row 1
        |[graycell]| & |[orangeminuscell]| & |[graycell]| & |[graycell]| \\
        % row 2
        |[orangeminuscell]|  & |[graycell]| & |[graycell]| & |[graycell]| \\
        % row 3
        |[graycell]| & |[graycell]| & |[graycell]| & |[graycell]| \\
        % row 4
        |[graycell]| & |[graycell]| & |[graycell]| & |[graycell]| \\
        % row 5
        |[graycell]| & |[graycell]| & |[graycell]| & |[graycell]| \\
    };

    % Column labels: attach ABOVE the north anchor of row 1 cells
    \foreach \c [count=\i from 1] in {$h_1$,$h_2$,$h_3$,$h_4$} {
        \node[anchor=south] at (M-1-\i.north) {\c};
    }

    % Row labels: attach LEFT of the west anchor of column 1 cells
    \foreach \r [count=\j from 1] in {$p_1$,$p_2$,$p_3$,$p_4$,$p_5$} {
        \node[anchor=east] at (M-\j-1.west) {\r};
    }
    \end{tikzpicture}
    \captionsetup{width=0.8\textwidth}  % override for this one only
    \caption{Finally, we shrink and learn the clause $\overline{x}_{1, 2} \lor \overline{x}_{2, 1}$.
    E.g., we can omit $\overline{x}_{3, 2}$ from the condition because of clause $\overline{x}_{1, 2} \lor \overline{x}_{3, 2}$.}~\label{subfig:pigeonholeclause-d}
\end{subfigure}


    \caption{Learning the clause $\overline{x}_{1, 2} \lor \overline{x}_{2, 1}$
    for \ph{4}}~\label{fig:pigeonholeclauses}
  \end{figure*}

We use the pigeonhole principle as a motivating example. The problem $\ph{n}$
asks whether we can put $n+1$ pigeons in $n$ holes such that (1)~every pigeon is
in a hole, and (2)~no hole contains more than one pigeon. This can be encoded as
a SAT problem where variable $x_{i, j}$ represents putting the $i$-th pigeon
into the $j$-th hole. Constraint (1) is encoded as $\bigvee_{1 \leq j \leq n}
x_{i, j}$ for each $1 \leq i \leq n+1$ and (2) as $\overline{x}_{i, j} \lor
\overline{x}_{k, j}$ for each $ 1 \leq i < k \leq n+1$ and $1 \leq j \leq n$.

\autoref{fig:pigeonholeclauses} provides a visualization where the rows
represent the pigeons and the columns represent the holes. The cell in row $i$
and column $j$ represents the literal $x_{i, j}$.
A $+$ in the $(i, j)$-th cell indicates that $x_{i, j}$ is set to $\top$, and a
$-$ symbol indicates $\bot$. Thus, constraint (1) asks that each row has at
least one $+$ and constraint (2) asks that each column has at most one $+$.

We achieve an $O(n^3)$ \pr proof for \ph{n}, matching the best known
result~\cite{prclauses}. In \autoref{fig:pigeonholeclauses} we learn the clause
$\overline{x}_{1, 2} \lor \overline{x}_{2, 1}$. After learning $n$ such clauses,
we learn the clause $\overline{x}_{1, 2}$, i.e. “pigeon 1 is not in hole 2.”
Since there are n+1 pigeons and n holes, we must rule out $O(n^2)$ pigeon-hole
pairs, so the proof has size $O(n^3)$. We learn these proofs with a very low
constant factor and robustness against encoding perturbations (see
\autoref{subsec:eval-pigeonhole}).

\subsection{PR Clause Learning Framework}~\label{subsec:methodology}

\begin{algorithm}\caption{Learning PR clauses}\label{alg:methodology}
    \SetAlgoNoLine \SetKwFunction{Learn}{LearnClause}
    \SetKwFunction{LCP}{LeastConditional} \SetKwFunction{Propagate}{Propagate}
    \SetKwFunction{Shrink}{Shrink} \SetKwFunction{Backtrack}{Backtrack}
    \SetKwFunction{Filter}{Filter} \SetKwFor{For}{for}{:}{}
    \SetKwFor{If}{if}{:}{} \SetKwProg{Fn}{Function}{:}{} \SetKwBlock{Begin}{}{}
    \Fn{\Learn{$\formula$, $\alpha$}}{ \For{$i \in vars(\formula)$}{
    \label{line:choose-i} \For{$j \in vars(\formula)$}{ \label{line:choose-j}
    \Propagate($i$); \\
                \Propagate($j$); \\
                $\alpha_c, \alpha_a := \LCP(\formula, \alpha)$; \\
                $C := \Shrink(\formula,\alpha_c, \alpha_a)$; \\
                \If{not  \Filter{C, $\formula$}}{ $\formula := \formula \land C$;}
                \Backtrack{}; } } }
\end{algorithm}

We present the general framework for using conditional autarkies to learn PR
clauses in \autoref{alg:methodology}. A conditional autarky is computed based on
a assignment, so the algorithm first selects a set of literals to propagate
(lines $4,5$), creating the assignment $\alpha$. We propagate two literals at a
time using nested \textbf{for} loops, and undo the propagations after each
iteration (backtracking in line $10$). In line $6$ \autoref{alg:leastcond}
makes a single pass over the formula to generate a conditional autarky, with
conditional part $\alpha_c$ and autarky part $\alpha_a$. The PR clause $C$
is generated from the conditional autarky and shrunk in line $7$, and if $C$
does not pass the usefulness heuristics it is filtered away in line $8$. Details
regarding design choices and heuristics are found in
\autoref{sec:implementation}, including a discussion of \texttt{Filter}.

In the following sections, we will describe PR clause learning and shrinking in
the context of our running example  \ph{4}. We will consider the decisions $i =
x_{1, 1}$ and $j = x_{2, 2}$ shown in \autoref{subfig:pigeonholeclause-a}, with
unit propagation shown in \autoref{subfig:pigeonholeclause-b}.

\subsection{Learning PR Clauses}~\label{subsec:learning}


As described in \autoref{subsec:autarkies}, any assignment can be split into a
conditional and a (potentially empty) autarky part, $\alpha = \alpha_c \sqcup \alpha_a$, and the
following clauses are PR: $\bigvee_{c \in \alpha_c} \overline{c} \lor a$ for
any $a \in \alpha_a$. In order to produce smaller PR clauses, we use
\autoref{alg:leastcond} from Kiesl et al. \cite{conditionalautarkies} to find
the unique smallest possible $\alpha_c$.


\begin{algorithm}
    \caption{Unique minimal $\alpha_c$ in $\alpha = \alpha_c \sqcup
    \alpha_a$}\label{alg:leastcond} \SetAlgoNoLine
    \SetKwFunction{LCP}{LeastConditional} \SetKwFor{For}{for}{:}{}
    \SetKwFor{If}{if}{:}{} \SetKwProg{Fn}{Function}{:}{} \SetKwBlock{Begin}{}{}

    \Fn{\LCP{$\formula$, $\alpha$}}{

        $\alpha_c := \emptyset$\; \For{$C \in \formula$}{ \If{$\alpha$ touches
        $C$ without satisfying $C$}{ $\alpha_c := \alpha_c \cup (\alpha \cap
        \overline{C})$\; } } \Return{$\alpha_c$, $\alpha \backslash \alpha_c$}\;
        }
\end{algorithm}

\autoref{alg:leastcond} returns $\alpha_c \sqcup \alpha_a$, which is a
conditional autarky, as every clause that $\alpha_a$ touches is satisfied by a
literal in $\alpha$.

Additionally, $\alpha_c$ is minimal: for any other conditional autarky $\alpha =
\alpha_c' \sqcup \alpha_a'$, it must be the case that $\alpha_c \subseteq
\alpha_c'$ since for each clause that is touched but not satisfied, we add to
$\alpha_c$ all literals from the assignment that touch and do not satisfy this
clause. These literals must be in $\alpha_c'$, since otherwise $\alpha_a'$ will touch a
clause that is not satisfied by $\alpha$, violating the conditional autarky
property.

Running \autoref{alg:leastcond} on the assignment from
\autoref{subfig:pigeonholeclause-b}  
gives the conditional part (red) and autarky part (orange) in
\autoref{subfig:pigeonholeclause-c}. The assigned literals from pigeons $3,4,$
and $5$ appear in $\alpha_c$ because they touch but do not satisfy the
constraint ($1$) clauses for the respective pigeons stating that every pigeon is
in a hole. However, the constraint ($1$) clauses are satisfied for pigeons $1$
and $2$, along with the touched constraint ($2$) clauses, placing the pigeons
$1$ and $2$ literals in $\alpha_a$. Intuitively, if pigeons $3,4,$ and $5$ are
not in holes $1$ or $2$, then pigeon $1$ can be placed in hole $1$ : $x_{3,1}
\lor x_{3,2} \lor x_{4,1} \lor x_{4,2} \lor x_{5,1} \lor x_{5,2} \lor x_{1,1} $
and pigeon $1$ can be kept out of hole $2$ : $x_{3,1} \lor x_{3,2} \lor x_{4,1}
\lor x_{4,2} \lor x_{5,1} \lor x_{5,2} \lor \overline{x}_{1,2} $ (likewise for
pigeon $2$ but with the holes swapped). 
The shrinking technique in the following section will help reduce the size of
these large PR clauses. 



\subsection{Shrinking PR Clauses}~\label{subsec:shrinking}

The PR clause derived from a conditional autarky can be too weak to effectively
reduce the search space.
Shrinking the PR clause, i.e., removing literals from the clause via
resolution, will strengthen its pruning power. 

At a high-level, we will generate two sets: $C_0 \subseteq \alpha_c$ and $A_0
\subseteq \alpha_a$, and show that we can learn the PR clause $\bigvee_{c \in C
\backslash C_0} \overline{c} \lor \bigvee_{a \in A_0} a$. 

First we start with $C_0$, the set of literals in the conditional part that are
inconsistent with any literal in the autarky part. Two literals $l_i$ and $l_j$
are inconsistent in a formula $\formula$ if
$\impunitclauseNoNeg{\formula}{l_i}{\overline{l}_j}$, meaning the clause
$\overline{l}_i \lor \overline{l}_j$ is RUP. Formally, $C_0 = \{c_j \in \alpha_c
\mid \exists a_i \in \alpha_a \text{ s.t. }
\impunitclause{\formula}{a_i}{c_j} \}$, with $a_i$ appearing negated
in the inconsistency check so that we can perform a resolution between the
derived binary clause  $a_i \lor c_j$ and the original PR clause $C$.

Next we generate $A_0$ which is a subset of the autarky literals that are
together inconsistent with the literals in $C_0$. So, for each $c \in C_0$ there
exists an $a \in A_0$ such that $\impunitclause{\formula}{a}{c}$. There always
exists at least one $A_0$, namely $\alpha_a$ which is used to define $C_0$, but
if a smaller $A_0$ exists it can be used to shrink the size of the PR clause. 


\begin{theorem}~\label{thm:shrunkgbcequisat} Let $\Gamma$ be a formula and
    $\alpha = \alpha_c \sqcup \alpha_a$ be a conditional autarky on $\Gamma$,
    with $\alpha_c = c_1, \dots, c_n$ and $\alpha_a = a_1, \dots, a_m$.
    Let $A_0 \subseteq \alpha_a$ be non-empty.
    Let $C_0 = \{c_j \in \alpha_c
\mid \exists a_i \in A_0 \text{ s.t. }
\impunitclause{\formula}{a_i}{c_j} \}$.
    Then formula $\formula$ is satisfiable if and only if
    $\formula \land (\bigvee_{c \in C \backslash C_0} \overline{c} \lor
    \bigvee_{a \in A_0} a)$ is satisfiable.
\end{theorem}

\begin{proof}
    \underline{$\Leftarrow$:} This is immediate


    \underline{$\Rightarrow$:} From \autoref{thm:gbcequisat}, $\formula$ is
    satisfiable implies $\formula \land (\bigvee_{c \in C} \overline{c} \lor a)$
    is satisfiable for any $a \in \alpha_a$. We will pick an $a \in A_0$. Hence,
    $\formula \land (\bigvee_{c \in C} \overline{c} \lor \bigvee_{a \in A_0} a)$
    is satisfiable since we are weakening a clause.

    By the definition of inconsistency, for each $c \in C_0$ there exists an $a
    \in A_0$ such that $\impunitclause{\formula}{a}{c}$, allowing us to derive
    the RUP clause  $a \lor c$. Each binary clause can be resolved with $
    (\bigvee_{c \in C} \overline{c} \lor \bigvee_{a \in A_0} a)$, producing
    $(\bigvee_{c \in C \backslash C_0} \overline{c} \lor \bigvee_{a \in A_0}
    a)$. 
    
\end{proof}

In fact, $\bigvee_{c \in C \backslash C_0} \overline{c} \lor \bigvee_{a \in A_0}
a$ is a PR clause with witness $\alpha_a$. 
% todo: might need to explain this more 


\noindent \textbf{Greedy Set Cover.}~\label{subsec:sym}
We can interpret the problem of finding the smallest possible $A_0$ as a set
cover problem where each literal $a \in \alpha_a$ defines the set  ${\it
SETS}(a) = \{ c \in \alpha_c \; | \; \impunitclause{\formula}{a}{c}\}$. Finding
the minimum set cover is NP-hard, so we instead approximate using a greedy
algorithm, returning a set cover at most roughly $(\ln |\alpha_c| + 1)\times$
the size of the smallest set cover~\cite{greedysetcover}. 


The greedy algorithm initializes $C_0$ as empty, and in each iteration finds the
$a \in \alpha_a$ that generates the largest set ${\it SETS}(a) \cap (\alpha_c
\backslash C_0)$, adding $a$ to $A_0$ and adding ${\it SETS}(a)$ to $C_0$. The
algorithm terminates once all sets in $\{{\it SETS}(a) \cap (\alpha_c \backslash
C_0) | a \in \alpha_a \}$ are empty. Notice that oftentimes, $C_0 \subsetneq \alpha_c$,
i.e. we do not achieve a complete cover of $\alpha_c$.


\begin{algorithm}
    \caption{Algorithm finding $A_0$}\label{alg:finda0} \SetAlgoNoLine
    \SetKwFunction{Shrink}{Shrink} \SetKwFor{For}{for}{:}{}
    \SetKwFor{If}{if}{:}{} \SetKwProg{Fn}{Function}{:}{} \Fn{\Shrink{$\formula$,
    $\alpha_a$, $\alpha_c$}}{
        ${\it SETS}$ := init \texttt{Array[len($\alpha_a$)]}\label{line:setsinits}\; \For{$i \in
        \texttt{range}(\alpha_a)$}{\label{line:fori}
        \texttt{Propagate($\overline{\alpha_a[i]}$)}\label{line:propagate}\; \texttt{implied} :=
        $\{\}$\; \For{$c \in \alpha_c$}{ \texttt{Propagate($\overline{c}$)}\;
        \If{\texttt{unsat}}{ $\texttt{implied} := \texttt{implied} \cup \{c\}$\;
        } \texttt{Backtrack(1)}\; } ${\it SETS}[i]$ := \texttt{implied}\label{line:setsi}\; }
        \Return{\texttt{GreedySetCover}(${\it SETS}$)}\; }
\end{algorithm}


In \autoref{alg:finda0}, we describe our process for calculating $A_0$. 
We initialize ${\it SETS}$ as an array (line~\ref{line:setsinits}), iterate the counter $i$ through
$\alpha_a$ (line~\ref{line:fori}), populating ${\it SETS}[i]$ with the set of literals $c \in \alpha_c$
such that $\impunitclause{\formula}{a_i}{c}$ (lines~\ref{line:propagate}-\ref{line:setsi}). Finally, we apply
\texttt{GreedySetCover} to get a small set $A_0$ that covers as much of $C$ as
possible.

Returning to the pigeonhole example, $C_0$ contains all of the literals in
$\alpha_c$ because of the binary clauses in constraint (2). The smallest
possible $A_0$ includes $\overline{x}_{2, 1}$ and $\overline{x}_{1, 2}$, whose
sets cover $\overline{x}_{3, 1}, \overline{x}_{4, 1},  \ldots, \overline{x}_{5,
1}$and  $\overline{x}_{3, 2}, \overline{x}_{4, 2},  \ldots, \overline{x}_{5, 2}$
respectively. Whereas, the sets from $x_{1, 1}$ and $x_{2, 2}$ are empty (they
are not inconsistent with the other literals). Therefore, we can learn the
clause $\overline{x}_{2, 1} \lor \overline{x}_{1, 2}$, shown in
\autoref{subfig:pigeonholeclause-d}.

  \section{Implementation}~\label{sec:implementation}

We implement our technique in \tool (a fork of \cadical
%  commit f13d74439a5b5c963ac5b02d05ce93a8098018b8
). By default, \tool searches for \pr clauses as a preprocessing step for 30
seconds. After this time limit, the solver exits and commences normal solving,
regardless of whether or not it has found any \pr clauses. This adds $\sim\!800$ lines
of C++ code to \cadical and is implementable within any CDCL SAT solver.
 

We experimented with other heuristics, which we briefly describe.
We include an evaluation of the heuristics in \autoref{subsec:heuristics}.

% \subsection{Other Heuristics}~\label{subsec:heuristics}

% todo: maybe move these two paragraphs to implementation section

% \noindent \textbf{Sorting $\alpha_a$ by Implication:}%~\label{subsubsec:impordering}
% The greedy set cover algorithm described above is almost sufficient for being insensitive to heuristics, but there is one other optimization that we made. \autoref{alg:finda0} describes sorting $\alpha_a$ as an initial step. This is important for cases where there are more than one cover of the same size. The most common occurence of this is for $a_1, a_2 \in \alpha_a$, we may have that $\overline{a_2} \impunit^\formula \overline{a_1} \impunit^\formula c_1, \dots, c_n$. 

% Thus, greedy set cover could pick $\{a_1\}$ or $\{a_2\}$ as singleton sets. However, since $a_1 \impunit^\formula a_2$, we really want to learn the unit clause $a_1$ since it is much more powerful.

\subsubsection{Sorting $\alpha_a$ by Implication}~\label{subsubsec:impordering}
% The greedy set cover algorithm described above is almost sufficient for being
% insensitive to heuristics, but there is one other optimization that we made.
We could sort $\alpha_a$ by implication before \autoref{alg:finda0}
\autoref{line:sort-alphaa}. This is important when there is more than one set
cover of the same size. For instance, if  $a_1, a_2 \in \alpha_a$ and
% todo: fix overline over a_1
$\impunitclause{\formula}{a_1}{a_2}$. Then, we would prefer to learn a clause with 
$a_1$ since it implies $a_2$. We select for this by first sorting $\alpha_a$ by
implication, i.e., we would move $a_1$ before $a_2$ in $\alpha_a$ if it were not
already there.
% todo: maybe delete this, if we do not evaluate on it

% and $\alpha_c = c$, we may have
% $\impunitclause{\psi}{a_1}{a_2}$ and $\impunitclause{\psi}{a_2}{c}$. Hence, we
% would also have $\impunitclause{\psi}{a_1}{c}$. Thus, we could learn the unit
% clause $a_1$ or the unit clause $a_2$.

% Here $a_1 \lor c$ is the more
% ``powerful'' clause, so it is better to learn. 

% todo: talk about the multiple propagation heuristic used for mutilated chessboard

\subsubsection{Filtering Trivial Clauses}~\label{subsubsec:filteringtriv}
Introducing globally blocked clauses can affect how we divide $\alpha = \alpha_c
\sqcup \alpha_a$ as in \autoref{alg:leastcond}. Thus, it is better to avoid
introducing globally blocked clauses if they are not useful. We do this by
checking if $\formula \impunit C$ for each potential clause $C$. If it is implied by
unit propagation, we consider it easy to learn and do not add it.

\subsubsection{Filtering Long Clauses}~\label{subsubsec:filtering-length}
Another heuristic for filtering is based on clause size. If a clause is longer
than some set length, we may choose not to learn it, as shorter clauses are
typically more useful. This can be combined with filtering out trivial clauses
or used separately.

\subsubsection{Ordering Literals}~\label{subsubsec:ordering-literals}
So far, we propagate on literals $i$ and $j$ selected randomly.
However, there could be better ways to order literals. One technique is to order
$i$ based on the number of clauses it occurs in. Another is to pick $j$ from the neighbors of $i$. This was originally proposed in
Reeves et al.~\cite{prelearn}.

\subsubsection{Alternative Shrinking
techniques}~\label{subsubsec:shrink-techniques} As discussed in
\autoref{subsec:shrinking}, we may shrink a clause if
$\impunitclause{\formula}{a}{c}$, where $a \in \alpha_a$ and $c \in \alpha_c$.
Checking if $\impunitclause{\formula}{a}{c}$ can be expensive. A cheaper option is
to see if there is a clause $a \lor c$ in the original formula. While this may
miss some pairs, it is sufficient in certain cases, such as for the
pigeonhole benchmarks. Another alternative is to not shrink the clause
at all.

% Here is a complete list of flags discussed in 

  \section{Evaluation}~\label{sec:evaluation}

In this section, we empirically evaluate our techniques against other \pr clause
learning techniques. In doing so, we aim to answer the following research questions:


\begin{enumerate}
    \item Can our approach provide short \pr proofs for certain benchmark families?
    \item Is our approach less sensitive to encoding choices compared to other
    \pr learning techniques?
    % \item Can these techniques underperform or outperform SAT solvers on other certain benchmark
    % families?
\end{enumerate}


We compare the two main tools learning \pr: clauses \sadical (based on SDCL)
and \prelearn (a preprocessing technique that calls \sadical). To be consistent with our approach,
we run \prelearn with 50 iterations and for 30 seconds. We also compare to \cadical as a baseline with no \pr clause learning.

In \autoref{subsec:eval-pigeonhole}, we compare all approaches
on the pigeonhole principle. We evaluate based on time taken, proof length, and sensitivity to the encoding of the formula. 

In \autoref{subsec:eval-satcomp}, we evaluate the solvers on benchmarks from the
annual '22, '23, and '24 SAT competition's main
tracks~\cite{satcomp2022,satcomp2023,satcomp2024}.

In \autoref{subsec:eval-discussion}, we highlight certain benchmark families
that benefit from \pr clause learning. We evaluate the different approaches on these families.


Finally, we analyze the use of specific heuristic choices in \tool
by turning heuristics off individually and observing their effect (\autoref{subsec:eval-heuristics}). 

All experiments were performed in the Anvil Supercomputing Center on nodes with
128 cores and 2 GB RAM per core~\cite{anvil}. We ran 64 experiments in parallel per node
with a 5,000 second timeout, the default timeout for the SAT competition.

\subsection{Pigeonhole results}~\label{subsec:eval-pigeonhole}

\begin{figure*}[!t]
    \centering
    \begin{subfigure}[t]{0.4\textwidth}
        \centering
        \includegraphics[width=\textwidth]{figs/pigeonhole_runtime_comparison.jpg}
        % \caption{Runtime on Pigeonhole Principle formulae}
        \label{fig:pigeonhole-runtime-comparison}
    \end{subfigure}
    \hspace{0.06\textwidth}
    \begin{subfigure}[t]{0.4\textwidth}
        \centering
        \includegraphics[width=\textwidth]{figs/pigeonhole_proof_size_comparison.jpg}
        % \caption{Proof Size for Pigeonhole Principle formulae}
        \label{fig:pigeonhole-proof-size-comparison}
    \end{subfigure}
    \caption{Comparison of \tool, \cadical, \sadical, and \prelearn on pigeonhole principle benchmarks up to size $40$. The y-axis is on a cube root scale. The performance of a solver on the original benchmark is shown with a solid line. The median of 5 scranfilized queries is shown with a dashed line. If a solver times out on a query in 5000s, it is not shown.}
    \label{fig:pigeonhole-results}
\end{figure*}


Approaches based on SDCL, such as \sadical, are successful for learning $O(n^3)$ proofs for the pigeonhole
principle, but are very sensitive to the encoding of the formula. We compare the solvers on
pigeonhole principle from \ph{2} to \ph{40} and plot these results in \autoref{fig:pigeonhole-results}. As the expected best-behavior is cubic, we use a cube root scale for the y-axis.

As expected, \cadical grows exponentially, while \sadical and \tool scale cubicly in both runtime and proof size. Significantly, \tool is able to learn $3.59$-$3.64$x shorter proofs compared to \sadical. 
% on formulae larger than \ph{10}
This is because \sadical deletes clauses more frequently, as to not exclude other clauses from being learned. Frequent deletion is not necessary in \tool because of its clause shrinking technique.
\prelearn scales cubicly on small formulae, but for \ph{22} and larger, will not learn enough useful \pr clauses in the preprocessing step and times out.

% As we discuss in
% \autoref{app:pigeonhole}, \tool learns proofs of size $\approx \frac13 n^3$,
% which is expected to be shorter than known \pr proofs for the pigeonhole principle.

Additionally, we evaluate all solvers on scranfilized variations of the
pigeonhole principle. Scranfilization is a technique for generating an
equivalently satisfiable formula~\cite{scranfilize}. We use the tool
\texttt{scranfilize}~\cite{scranfilize} with the options permuting variables, permuting clauses, and flipping literals (with probability $0.5$) all turned on. We run each solver on 5 scranfilized variations for
each benchmark and take the median runtime and proof size. This is shown in \autoref{fig:pigeonhole-results} with dashed lines.

\sadical and \prelearn exhibit an exponential trend for runtime and proof size on the scranfilized benchmarks. \sadical will spend all its time in the main SDCL loop not learning enough useful clauses. \prelearn will learn some useful \pr clauses in preprocessing, but not enough to sufficiently shrink the search space for formulae larger than \ph{16}.

On the other hand, \tool almost matches its non-scranfilized performance, demonstrating that it learns useful \pr clauses regardless of the encoding.

% In conclusion, \tool is able to match \sadical's runtime for the pigeonhole principle while shorter proofs by a constant factor. Additionally, \tool is insensitive to permuting variables, permuting clauses, and flipping literals as it relies on conditional autarkies, a global property of the formula.



\subsection{SAT competition results}~\label{subsec:eval-satcomp}

\begin{figure*}[!t]
    \centering
    \begin{subfigure}[t]{0.45\textwidth}
        \centering
        \includegraphics[width=\textwidth]{figs/cadical_vs_cautical_nontrivial.jpg}
        \caption{Comparing \tool to \cadical. The color indicates the number of \pr clauses learnt by \tool}
        \label{fig:cautical-vs-cadical}
    \end{subfigure}
    % \hspace{0.06\textwidth}
    \begin{subfigure}[t]{0.45\textwidth}
        \centering
        \includegraphics[width=\textwidth]{figs/cadical_vs_prelearn_nontrivial.jpg}
        \caption{Comparing \prelearn to \cadical. The color indicates the number of \pr clauses learnt by \prelearn}
        \label{fig:cautical-vs-prelearn}
    \end{subfigure}
    \caption{Performance comparison of \tool with and \prelearn with \cadical on SAT competition benchmarks. We filter out all benchmarks where the solvers do not learn any \pr clauses. The color indicates the number of \pr clauses learnt by the solver.}
    \label{fig:solver-comparison}
\end{figure*}

We compare the performance of \tool with \cadical and \prelearn on the benchmarks from the '22,'23, and '24 SAT competition's main tracks~\cite{satcomp2022,satcomp2023,satcomp2024}. We remove duplicates and  exclude all benchmarks with more than twenty million clauses. This gives us a total of 1089 benchmarks. We excluded \sadical as it can only solve 22 of these benchmarks.

\autoref{tab:solver-stats} shows the number of instances solved by each solver and the number of queries for which \prelearn and \cadical learn additional \pr clauses, how many improve upon \cadical by at least 5\%, and how many solve a formula that \cadical does not solve. We divide the benchmarks based on number of clauses (0-10k or 10k-20M) and status (SAT or UNSAT).

PAR-2 score is a standard metric used to evaluate the performance of solvers.
It is evaluated as the sum of the runtimes of solved instances and twice the timeout of unsolved instances. On this dataset, \cadical has a PAR-score of 3521.79 seconds, \prelearn has a PAR-score of 3331.41 seconds, and \tool has a PAR-score of 3442.14 seconds. This can be explained by \prelearn solving the most formulae at 771, followed by \tool at 754, and \cadical at 749.

%  I don't know if this actually makes sense
Notably, \tool solves 36 formulae that \cadical does not solve, while \prelearn solves 30 that \cadical does not solve. \tool improves on 219 formulae from \cadical, while \prelearn improves on 146 formulae. We attribute this to more selective \pr clause learning techniques in \tool. Learning unhelpful \pr clauses can have a detrimental effect on performance. \tool only learns 50 \pr clauses or more for 57 formulae, while \prelearn learns 50 \pr clauses or more for 256 formulae. On those clauses, \tool improves the \cadical's runtime by ??\%, while \prelearn improves it by ??\%.

\autoref{fig:solver-comparison} shows \tool and \cadical's performance relative to \cadical. We only include points where \tool (for \autoref{subfig:cautical-vs-cadical}) and \prelearn (for \autoref{subfig:cautical-vs-prelearn}) learn at least one \pr clause. \autoref{subfig:cautical-vs-cadical} shows that \tool learns a large number of \pr clauses for formulae that it can solve quickly and \cadical times out on. \autoref{subfig:cautical-vs-prelearn} shows that \prelearn also solve a number of formulae that \tool cannot, but it learns a large number of \pr clauses for other formulae.







% prelearn PAR score:  3331.406813590451
% cautical PAR score:  3442.137998163452
% cadical PAR score:  3521.7908356290172

\begin{table}[ht]
    \centering
    \sisetup{table-format=3}        % remove if you are not using siunitx
    \begin{tabular}{lrrrrr}
      \toprule
      & \multicolumn{2}{c}{0--10k} & \multicolumn{2}{c}{10k--20M} \\
      \cmidrule(lr){2-3} \cmidrule(lr){4-5}
      & SAT & UNSAT & SAT & UNSAT & Total \\
      \midrule
    %   Total Formulas & 70 & 132 & 395 & 416 & 1099 \\
      \cadical Solved  &  54 &  73 & 319 & 303 & 749 \\
      \midrule
      \prelearn \\
      \; Total &  52 &  90 & 322 & 307 & 771 \\
      \; Learnt clauses   &  40 &  73 & 179 & 145 & 431\\
      \; Learnt $>50$ clauses   &  22 &  51 & 104 &  79 & 256\\
      \; Improved &  11 &  42 &  57 &  36 & 146\\
      \; Exclusively &   1 &  17 &   6 &   6 & 30 \\
      \midrule
      \tool \\
      \; Total &  52 &  87 & 317 & 298 & 754 \\
      \; Learnt clauses     &   16 &  58 &  30 &  35 & 139 \\
      \; Learnt $>50$ clauses  &   1  &  39 &  6 &  11 & 57 \\
      \; Improved &  23  &  48 &  89 &  59 & 219 \\
      \; Exclusively &   0 &  18 &   9 &   9 & 36 \\
      \bottomrule
    \end{tabular}
    \caption{Number of solved instances.}
    \label{tab:solver-stats}
  \end{table}

  % Category                  0-10k SAT  0-10k UNSAT  10k-20M SAT  10k-20M UNSAT  20M+ SAT  20M+ UNSAT  Total
% Total Formulas                   70          132          395            416        40          46   1099 -> issue with this total is that certain formula statuses are unknown, so not counted here
% Cadical Solved                   54           73          319            303        35          46    830
% Has PR Clauses                   40           73          179            145         0           3    440
% Has Many PR Clauses              22           51          104             79         0           0    256
% Prelearn Solved                  52           90          322            307        35          46    852
% Prelearn Only                     1           17            6              6         0           0     30
% Prelearn Improved                11           42           57             36         1           1    148
% Cautical Solved                  52           87          317            298        32          29    815
% Cautical Only                     0           18            9              9         0           0     36
% Cautical Improved                23           48           89             59         0           8    227
% Blocked Clauses                  16           58           30             35         0           0    139
% Has Many Blocked Clauses          1           39            6             11         0           0     57



  \begin{figure*}[!t]
    \centering
    \begin{subfigure}[t]{0.45\textwidth}
        \centering
        \includegraphics[width=\textwidth]{figs/prelearn_vs_cautical.jpg}
        \subcaption{Performance comparison of \tool to \prelearn. The color indicates the number of \pr clauses learnt by \tool. We filter out all formula where neither \tool nor \prelearn learnt any \pr clauses.}
        \label{fig:cautical-vs-cadical}
    \end{subfigure}
    % \hspace{0.06\textwidth}
    \begin{subfigure}[t]{0.45\textwidth}
        \centering
        \includegraphics[width=\textwidth]{figs/clauses_histogram.jpg}
        \subcaption{Histogram showing the number of \pr clauses learnt by \tool and \prelearn}
        \label{fig:cautical-vs-prelearn}
    \end{subfigure}
    \caption{Comparison of \tool and \prelearn on SAT competition benchmarks.}
    % \label{fig:solver-comparison}
\end{figure*}





% \subsection{\toolminus}

% Motivated by the fact that \tool improves mostly on formulae where it learns many clauses, we define a new variant \toolminus. After running the 30 second preprocessing step, if \toolminus does not learn at least $50$ \pr clauses, it will completely reset the context and run cadical. In \autoref{fig:toolminus-comparison}, we compare \tool to \prelearn and \toolminus to \prelearn on the SAT competition benchmarks.


% \begin{figure*}[!t]
%     \centering
%     \begin{subfigure}[t]{0.45\textwidth}
%         \centering
%         \includegraphics[width=\textwidth]{figs/prelearn_vs_cautical.jpg}
%         \caption{Comparing \tool to \prelearn. The color indicates the number of \pr clauses learnt by \tool}
%         \label{fig:cautical-vs-cadical}
%     \end{subfigure}
%     % \hspace{0.06\textwidth}
%     \begin{subfigure}[t]{0.45\textwidth}
%         \centering
%         \includegraphics[width=\textwidth]{figs/prelearn_vs_cautical_minus_time.jpg}
%         \caption{Comparing \toolminus to \prelearn. The color indicates the number of \pr clauses learnt by \toolminus}
%         \label{fig:cautical-vs-prelearn}
%     \end{subfigure}
%     \caption{Performance comparison of \tool with and \toolminus with \prelearn on SAT competition benchmarks. We filter out all benchmarks where both \tool and \prelearn do not learn any \pr clauses. The color indicates the number of \pr clauses learnt by the \tool or \toolminus.}
%     % \label{fig:solver-comparison}
% \end{figure*}

% version of the graphs without filtering out clauses:
% prelearn_vs_cautical.jpg


\subsection{Discussion of Benchmark Families}~\label{subsec:eval-discussion}

We identify six benchmark families for which \pr clauses perform well. We choose them  based on prior work ~\cite{prelearn} and our experiments on SAT competition benchmarks:

\begin{enumerate}
    \item \texttt{mutilated-chessboard}: A famous problem asking if one can use $2$-by-$1$ tiles to cover an $n$-by-$n$ chessboard with opposite corners removed. This is known to be difficult for resolution~\cite{chessboard-resolution}, but have $O(n^3)$ \pr proofs~\cite{mutilatedchessboard-pr}.
    \item \texttt{perfect-matching}: A generalization of the pigeonhole principle and mutilated chessboard problems with various at-most-one constraints~\cite{bipartgen}
    \item \texttt{pcmax-scheduling}: A problem encoding the scheduling problem mapping $n$ tasks with known execution times to $m$ identical processors~\cite{pcmax}.
    % \item https://helda.helsinki.fi/server/api/core/bitstreams/3f1f286b-3def-49e9-98ba-f887b1bc250e/content -> page 31
    \item \texttt{register-allocation}: A problem representing the graph coloring problem generated by simulating register allocation on individual Python functions~\cite{register-allocation}.
    \item \texttt{relativized\_pigeonhole}: A generalization of the pigeonhole principle where we place $n+1$ pigeons in $n$ holes with $k$ nesting places~\cite{relativized-pigeonhole}.
    % todo: ask joseph about citation
    \item \texttt{satcoin}: An unsatisfiable variant of a bitcoin mining problem~\cite{satcoin}.
    \item \texttt{test\_configuration}: A problem asking if there is a list of configurations of size $k$ that covers every pairwise combination of configurations of a SAT solver~\cite{test-configuration}.
\end{enumerate}

\begin{figure*}[!t]
    \centering
    \begin{subfigure}[t]{0.3\textwidth}
            \centering
            \includegraphics[width=\textwidth]{figs/cadical_vs_cautical_interesting.jpg}
            \caption{Comparison with \cadical}
            \label{fig:cautical-vs-cadical}
    \end{subfigure}
        % \hspace{0.06\textwidth}
    % \hspace{0.06\textwidth}
    \begin{subfigure}[t]{0.3\textwidth}
        \centering
        \includegraphics[width=\textwidth]{figs/prelearn_vs_cadical_interesting.jpg}
        \caption{Comparison with \prelearn}
        \label{fig:cautical-vs-prelearn}
    \end{subfigure}
    \begin{subfigure}[t]{0.3\textwidth}
        \centering
        \includegraphics[width=\textwidth]{figs/prelearn_vs_cautical_interesting_legend.jpg}
        \caption{Comparison with \prelearn}
        \label{fig:cautical-vs-prelearn}
    \end{subfigure}

    \caption{Performance comparison of \tool with other solvers}
    % \label{fig:solver-comparison}
\end{figure*}


\subsection{Analysis of heuristics}~\label{subsec:eval-heuristics}

We provide a brief analysis of the heuristics used in \tool.


\begin{figure*}[!t]
    \centering
    \begin{subfigure}[t]{0.3\textwidth}
            \centering
            \includegraphics[width=\textwidth]{figs/global_time_lim_heuristic_comparison.jpg}
            \caption{Time limit}
            \label{fig:global-time-comparison}
    \end{subfigure}
        % \hspace{0.06\textwidth}
    % \hspace{0.06\textwidth}
    \begin{subfigure}[t]{0.3\textwidth}
        \centering
        \includegraphics[width=\textwidth]{figs/globalbcp_heuristic_comparison.jpg}
        \caption{BCP}
        \label{fig:globalbcp}
    \end{subfigure}
    \begin{subfigure}[t]{0.3\textwidth}
        \centering
        \includegraphics[width=\textwidth]{figs/globaldontfilter_heuristic_comparison.jpg}
        \caption{No filter}
        \label{fig:globaldontfilter}
    \end{subfigure}

    \caption{Performance comparison of \tool with other solvers}
    % \label{fig:solver-comparison}
\end{figure*}

\begin{figure*}[!t]
    \centering
    \begin{subfigure}[t]{0.3\textwidth}
            \centering
            \includegraphics[width=\textwidth]{figs/globaldouble_heuristic_comparison.jpg}
            \caption{Double}
            \label{fig:cautical-vs-cadical}
    \end{subfigure}
        % \hspace{0.06\textwidth}
    % \hspace{0.06\textwidth}
    \begin{subfigure}[t]{0.3\textwidth}
        \centering
        \includegraphics[width=\textwidth]{figs/globalisort_heuristic_comparison.jpg}
        \caption{Sort $i$ beforehand by frequency used}
        \label{fig:cautical-vs-prelearn}
    \end{subfigure}
    \begin{subfigure}[t]{0.3\textwidth}
        \centering
        \includegraphics[width=\textwidth]{figs/globalmaxlen_heuristic_comparison.jpg}
        \caption{Max length}
        \label{fig:cautical-vs-prelearn}
    \end{subfigure}

    \caption{Performance comparison of \tool with other solvers}
    % \label{fig:solver-comparison}
\end{figure*}

\begin{figure*}[!t]
    \centering
    \begin{subfigure}[t]{0.3\textwidth}
            \centering
            \includegraphics[width=\textwidth]{figs/globalnoshrink_heuristic_comparison.jpg}
            \caption{No shrink}
            \label{fig:cautical-vs-cadical}
    \end{subfigure}
        % \hspace{0.06\textwidth}
    % \hspace{0.06\textwidth}
    \begin{subfigure}[t]{0.3\textwidth}
        \centering
        \includegraphics[width=\textwidth]{figs/globalorderi_heuristic_comparison.jpg}
        \caption{Order $i$ beforehand in increasing order}
        \label{fig:cautical-vs-prelearn}
    \end{subfigure}
    \begin{subfigure}[t]{0.3\textwidth}
        \centering
        \includegraphics[width=\textwidth]{figs/globaltouch_heuristic_comparison.jpg}
        \caption{Touched heuristic}
        \label{fig:cautical-vs-prelearn}
    \end{subfigure}

    \caption{Performance comparison of \tool with other solvers}
    % \label{fig:solver-comparison}
\end{figure*}


\subsection{Research Questions}~\label{subsec:eval-research-questions}

We answer the research questions posed at the start of this section. For
question 1, we conclude positively that \tool is able to provide short \pr
proofs for certain benchmark families. We match the best known result for The

However, it is currently unknown whether conditional autarkies can be used to
learn the shortest known \pr proofs for the mutilated
chessboard~\cite{mutilatedchessboard-pr} or Tseitin graph
formulae~\cite{sadical}. Additionally, there are benchmarks in the ??, ??, and
?? families that \prelearn solves and \tool does not.

For question 2, we can conclude positively that \tool is less sensitive to
encoding choices compared to other \pr learning techniques. We show that \tool
has $O(n^3)$ proofs for the pigeonhole principle even when the variables are
permuted, clauses are permuted, and literals are flipped. Additionally, \tool
does not consider any information about the encoding when propagating literals
(it does so randomly). In fact, we show that heuristics used to rank order
literals have an adverse effect on performance.
  \section{Future Work}~\label{sec:futurework}

% The goal of this paper was to introduce the theory and provide a set of
% heuristics that work. \tool is designed so that trying different heuristics
% is easy. There are a number of interesting future directions to explore:

We believe conditional autarkies provide a route to incorporate \pr learning
as a regular part of state-of-the-art SAT solvers such as \cadical. However,
there are a few important limitations that need to be addressed first.

\begin{enumerate}
    \item \textbf{Improving the algorithms}: The heuristics in \tool are
    designed to be modular and easy to modify. However, they can
    be directly incorporated into the main algorithm. For instance, we could embed the
    binary clause filter by modifying the greedy set cover to discard clauses
    with more than 2 literals.
    \item \textbf{Shortest proofs for specific families:} Prior work has shown
    $O(n^3)$ \pr proofs of mutilated chessboard~\cite{mutilatedchessboard-pr}.
    While \tool learns useful clauses to speed up solving for
    mutilated chessboard, it is unknown whether conditional autarkies can match
    the $O(n^3)$ complexity. Similar questions exist for other well-known
    families such as Tseitin formulas over expander
    graphs~\cite{er,hardexamplesresolution}.
    \item \textbf{Learning PR clauses with an inprocessing step:} We
    experimented with learning PR clauses via inprocessing (instead of
    preprocessing). However, the current \pr proof system defines a PR clause
    as in \autoref{def:pr} where $\Gamma$ is the set of all clauses in the
    formula, including the learnt, redundant clauses. However, not all redundant clauses 
    must be considered. This could produce shorter clauses via conditional autarky.
    However, to realize this, we must modify the \pr proof checker.

    % todo: what is the definition of redundant here
    
    % is \pr relative to the
    % set of all clauses learnt so far. However, clauses that are \pr relative to
    % the original set of clauses are still satisfiability preserving and our
    % technique can learn easier clauses considering this notion of \pr. Resolving
    % this is much more complicated and requires updating the \pr proof system and
    % checker.
\end{enumerate}
  \section{Conclusion}~\label{sec:conclusion}


\pr clause learning is effective for learning short proofs for 
difficult problems. However, current \pr clause learning techniques require
an NP-hard check. We solve this by providing a technique that learns \pr clauses
in a linear time preprocessing step. We provide clause shrinking and filtering 
techniques to ensure that we learn useful \pr clauses.

Our implementation, \tool, provides short \pr proofs matching the best possible results for 
the pigeonhole principle. While prior work is only effective on specific pigeonhole
encodings, \tool finds these proofs even when the formula is scranfilized.

Additionally, \tool is effective on a number of benchmarks from the SAT competition,
including most of the families identified in \autoref{subsec:eval-discussion}.

In the future, we hope that \pr clause learning can find its way into the main branch
of a popular SAT solver. We believe this work is an effective step in this direction.


  \section{Acknowledgements}

We would like to thank the anonymous reviewers for their helpful comments and
suggestions. This work was supported Supported in part by a fellowship award
under contract FA9550-21-F-0003 through the National Defense Science and
Engineering Graduate (NDSEG) Fellowship Program, sponsored by the Air Force
Research Laboratory (AFRL), the Office of Naval Research (ONR) and the Army
Research Office (ARO).
  
  \bibliographystyle{IEEEtran}
  \bibliography{base}
  
  \end{document}
  