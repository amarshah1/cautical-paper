\section{Future Work}~\ref{sec:futurework}

The goal of this paper was to introduce the theory and provide a set of
heuristics that work. \tool is designed so that trying different heuristics
is easy. There are a number of interesting future directions to explore:

\begin{enumerate}
    \item \textbf{Improving the heuristics}: For example, we could embed the
    binary clause filter by exiting the greedy set cover when more than 2
    literals are used.
    \item \textbf{Shortest proofs for specific families:} Prior work has shown
    $O(n^3)$ \pr proofs of mutilated chessboard TODO: cite.  While we show that
    we can learn useful clauses that speed up solving for mutilated chessboard
    using our technique, it is still an open question whether this can match the
    $O(n^3)$ complexity. Similar questions exist for other well-known families
    such as Tseitin formulas TODO: cite and Urumquat TODO: cite.
    \item \textbf{Learning \pr clauses with an inprocessing step:} We
    experimented with learning \pr clauses via inprocessing (instead of
    preprocessing). However, the current \pr proof system is \pr relative to the
    set of all clauses learnt so far. However, clauses that are \pr relative to
    the original set of clauses are still satisfiability preserving and our
    technique can learn easier clauses considering this notion of \pr. Resolving
    this is much more complicated and requires updating the \pr proof system and
    checker.
\end{enumerate}