\section{Future Work}~\label{sec:futurework}

We believe conditional autarkies provide a route to incorporate \pr learning
as a regular part of state-of-the-art SAT solvers such as \cadical. However,
there are a few important limitations that need to be addressed first.

\begin{enumerate}
    \item \textbf{Improving the algorithms}: The heuristics in \tool are
    designed to be modular and easy to modify. However, they can
    be directly incorporated into the main algorithm. For instance, we could embed the
    binary clause filter by modifying the greedy set cover to discard clauses
    with more than 2 literals.
    \item \textbf{Shortest proofs for specific families:} Prior work has shown
    $O(n^3)$ \pr proofs of mutilated chessboard~\cite{mutilatedchessboard-pr}.
    While \tool learns useful clauses to speed up solving for
    mutilated chessboard, it is unknown whether conditional autarkies can match
    the $O(n^3)$ complexity. Similar questions exist for other well-known
    families such as Tseitin formulas over expander
    graphs~\cite{er,hardexamplesresolution}.
    \item \textbf{Learning PR clauses with an inprocessing step:} We
    experimented with learning PR clauses via inprocessing (instead of
    preprocessing). However, the current \pr proof system defines a PR clause
    as in \autoref{def:pr} where $\Gamma$ is the set of all clauses in the
    formula, including the learnt, redundant clauses. However, not all redundant clauses 
    must be considered. This could produce shorter clauses via conditional autarky.
    However, to realize this, we must modify the \pr proof checker.
\end{enumerate}