\section{Future Work}~\label{sec:futurework}

% The goal of this paper was to introduce the theory and provide a set of
% heuristics that work. \tool is designed so that trying different heuristics
% is easy. There are a number of interesting future directions to explore:

We believe conditional autarkies provide a route to incorporate \pr learning
as a regular part of state-of-the-art SAT solvers such as \cadical. However,
there are a few important limitations that need to be addressed first.

\begin{enumerate}
    \item \textbf{Improving the algorithms}: The heuristics in \tool are
    designed to be modular and easy to modify. However, they can
    be directly incorporated into the main algorithm. For instance, we could embed the
    binary clause filter by modifying the greedy set cover to discard clauses
    with more than 2 literals.
    \item \textbf{Shortest proofs for specific families:} Prior work has shown
    $O(n^3)$ \pr proofs of mutilated chessboard~\cite{mutilatedchessboard-pr}.
    While \tool learns useful clauses to speed up solving for
    mutilated chessboard, it is unknown whether conditional autarkies can match
    the $O(n^3)$ complexity. Similar questions exist for other well-known
    families such as Tseitin formulas over expander
    graphs~\cite{er,hardexamplesresolution}.
    \item \textbf{Learning \pr clauses with an inprocessing step:} We
    experimented with learning \pr clauses via inprocessing (instead of
    preprocessing). However, the current \pr proof system defines a \pr clause
    as in \autoref{def:pr} where $\Gamma$ is the set of all clauses in the
    formula, including the learnt clauses. However, if a learnt clause comes
    from the weaker proof system, it does not need to be in $\Gamma$. This could
    in theory allow us to learn a shorter clause, via conditional autarky.
    However, to realize this, we need to change the definition of \pr and the
    implementation of the \pr proof checker.

    % todo: what is the definition of redundant here
    
    % is \pr relative to the
    % set of all clauses learnt so far. However, clauses that are \pr relative to
    % the original set of clauses are still satisfiability preserving and our
    % technique can learn easier clauses considering this notion of \pr. Resolving
    % this is much more complicated and requires updating the \pr proof system and
    % checker.
\end{enumerate}