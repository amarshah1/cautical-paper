\section{Methodology}~\label{sec:method}

\subsection{Motivating Example}~\label{sec:motivatex}

\begin{figure*}[!t]
    \centering
    \begin{subfigure}[t]{0.23\textwidth}
    \centering
    \begin{tikzpicture}
      % Draw the 4×5 grid and name it "M"
      \matrix (M) [gridmatrix] {
        % row 1
        |[bluepluscell]| & |[graycell]| & |[graycell]| & |[graycell]| \\
        % row 2
        |[graycell]|  & |[bluepluscell]| & |[graycell]| & |[graycell]| \\
        % row 3
        |[graycell]| & |[graycell]| & |[graycell]| & |[graycell]| \\
        % row 4
        |[graycell]| & |[graycell]| & |[graycell]| & |[graycell]| \\
        % row 5
        |[graycell]| & |[graycell]| & |[graycell]| & |[graycell]| \\
      };

      % Column labels: attach ABOVE the north anchor of row 1 cells
      \foreach \c [count=\i from 1] in {$h_1$,$h_2$,$h_3$,$h_4$} {
        \node[anchor=south] at (M-1-\i.north) {\c};
      }

      % Row labels: attach LEFT of the west anchor of column 1 cells
      \foreach \r [count=\j from 1] in {$p_1$,$p_2$,$p_3$,$p_4$,$p_5$} {
        \node[anchor=east] at (M-\j-1.west) {\r};
      }
    \end{tikzpicture}
    \captionsetup{width=0.8\textwidth}  % override for this one only
    \caption{We initially propagate on $x_{1, 1}$ and $x_{2, 2}$.}~\label{subfig:pigeonholeclause-a}  
\end{subfigure}
\begin{subfigure}[t]{0.23\textwidth}
    \centering
    \begin{tikzpicture}
        % Draw the 4×5 grid and name it "M"
        \matrix (M) [gridmatrix] {
        % row 1
        |[bluepluscell]| & |[yellowminuscell]| & |[graycell]| & |[graycell]| \\
        % row 2
        |[yellowminuscell]|  & |[bluepluscell]| & |[graycell]| & |[graycell]| \\
        % row 3
        |[yellowminuscell]| & |[yellowminuscell]| & |[graycell]| & |[graycell]| \\
        % row 4
        |[yellowminuscell]| & |[yellowminuscell]| & |[graycell]| & |[graycell]| \\
        % row 5
        |[yellowminuscell]| & |[yellowminuscell]| & |[graycell]| & |[graycell]| \\
        };

        % Column labels: attach ABOVE the north anchor of row 1 cells
        \foreach \c [count=\i from 1] in {$h_1$,$h_2$,$h_3$,$h_4$} {
            \node[anchor=south] at (M-1-\i.north) {\c};
        }

        % Row labels: attach LEFT of the west anchor of column 1 cells
        \foreach \r [count=\j from 1] in {$p_1$,$p_2$,$p_3$,$p_4$,$p_5$} {
            \node[anchor=east] at (M-\j-1.west) {\r};
        }
    \end{tikzpicture}
    \captionsetup{width=0.8\textwidth}  % override for this one only
    \caption{Then unit propagation gives us $\overline{x}_{2,1}, \ldots, \overline{x}_{5,1},$ $\overline{x}_{1, 2}, \overline{x}_{3, 2}, \ldots, \overline{x}_{5, 2}$.}~\label{subfig:pigeonholeclause-b} 
\end{subfigure}
\begin{subfigure}[t]{0.23\textwidth}
    \centering
    \begin{tikzpicture}
    % Draw the 4×5 grid and name it "M"
    \matrix (M) [gridmatrix] {
        % row 1
        |[orangepluscell]| & |[orangeminuscell]| & |[graycell]| & |[graycell]| \\
        % row 2
        |[orangeminuscell]|  & |[orangepluscell]| & |[graycell]| & |[graycell]| \\
        % row 3
        |[purpleminuscell]| & |[purpleminuscell]| & |[graycell]| & |[graycell]| \\
        % row 4
        |[purpleminuscell]| & |[purpleminuscell]| & |[graycell]| & |[graycell]| \\
        % row 5
        |[purpleminuscell]| & |[purpleminuscell]| & |[graycell]| & |[graycell]| \\
    };

    % Column labels: attach ABOVE the north anchor of row 1 cells
    \foreach \c [count=\i from 1] in {$h_1$,$h_2$,$h_3$,$h_4$} {
        \node[anchor=south] at (M-1-\i.north) {\c};
    }

    % Row labels: attach LEFT of the west anchor of column 1 cells
    \foreach \r [count=\j from 1] in {$p_1$,$p_2$,$p_3$,$p_4$,$p_5$} {
        \node[anchor=east] at (M-\j-1.west) {\r};
    }
    \end{tikzpicture}
    \captionsetup{width=0.8\textwidth}  % override for this one only
    \caption{Literals $\overline{x}_{3, 1}, \ldots, \overline{x}_{5, 1},$ $\overline{x}_{3, 2}, \ldots, \overline{x}_{5, 2}$ are in $\alpha_c$ by their respective constraint (1). The rest are in $\alpha_a$.}~\label{subfig:pigeonholeclause-c}
\end{subfigure}
\begin{subfigure}[t]{0.23\textwidth}
    \centering
    \begin{tikzpicture}
    % Draw the 4×5 grid and name it "M"
    \matrix (M) [gridmatrix] {
        % row 1
        |[graycell]| & |[orangeminuscell]| & |[graycell]| & |[graycell]| \\
        % row 2
        |[orangeminuscell]|  & |[graycell]| & |[graycell]| & |[graycell]| \\
        % row 3
        |[graycell]| & |[graycell]| & |[graycell]| & |[graycell]| \\
        % row 4
        |[graycell]| & |[graycell]| & |[graycell]| & |[graycell]| \\
        % row 5
        |[graycell]| & |[graycell]| & |[graycell]| & |[graycell]| \\
    };

    % Column labels: attach ABOVE the north anchor of row 1 cells
    \foreach \c [count=\i from 1] in {$h_1$,$h_2$,$h_3$,$h_4$} {
        \node[anchor=south] at (M-1-\i.north) {\c};
    }

    % Row labels: attach LEFT of the west anchor of column 1 cells
    \foreach \r [count=\j from 1] in {$p_1$,$p_2$,$p_3$,$p_4$,$p_5$} {
        \node[anchor=east] at (M-\j-1.west) {\r};
    }
    \end{tikzpicture}
    \captionsetup{width=0.8\textwidth}  % override for this one only
    \caption{Finally, we shrink and learn the clause $\overline{x}_{1, 2} \lor \overline{x}_{2, 1}$.
    E.g., we can omit $\overline{x}_{3, 2}$ from the condition because of clause $\overline{x}_{1, 2} \lor \overline{x}_{3, 2}$.}~\label{subfig:pigeonholeclause-d}
\end{subfigure}


    \caption{Learning the clause $\overline{x}_{1, 2} \lor \overline{x}_{2, 1}$ for \ph{4}}~\label{fig:pigeonholeclauses}
  \end{figure*}

We use the pigeonhole principle as a motivating example.
The problem $\ph{n}$ asks whether we can put $n+1$ pigeons in $n$ holes such
that (1)~every pigeon is in a hole, and (2)~no hole contains more than one
pigeon. This can be encoded as a SAT problem where variable $x_{i, j}$
represents putting the $i$-th pigeon into the $j$-th hole. Constraint (1)
is encoded as $\bigvee_{1 \leq j \leq n} x_{i, j}$ for each $1 \leq i \leq n+1$ and
(2) as $\overline{x}_{i, j} \lor \overline{x}_{k, j}$ for each $ 1
\leq i < k \leq n+1$ and $1 \leq j \leq n$.

\autoref{fig:pigeonholeclauses} provides a visualization where the
rows represent the pigeons and the columns represent the holes. The cell in row
$i$ and column $j$ represents the literal $x_{i, j}$.
% i.e. whether the $i$-th pigeon is in the $j$-th hole.
A $+$ in the $(i, j)$-th cell indicates that $x_{i, j}$ is set to
true, and a $-$ symbol indicates falsehood.
Thus, constraint (1) asks that each row has at least one $+$ cell and
constraint (2) asks that at most one $+$ cell share a column.

% We only consider the case where $m = n + 1$ which we denote \ph{n}. 
We achieve an $O(n^3)$ \pr proof, matching the best known
result~\cite{prclauses}. We learn these proofs with a very low constant
factor and robustness against encoding perturbations (see \autoref{subsec:eval-pigeonhole}).
%  todo: add a forward reference to where we evaluate this
%  todo: is this the best possible result theoretically?

% There are $O(n^3)$ \pr proofs of the pigeonhole principle, but this is difficult to achieve in practice and requires specific techniques~\cite{prclauses}. We can learn these proofs, with a very low constant factor and little sensitivity to the encoding. 

\subsection{PR Clause Learning Framework}~\label{subsec:methodology}

\begin{algorithm}\caption{Learning \pr clauses}\label{alg:methodology}
    \SetAlgoNoLine
    \SetKwFunction{Learn}{LearnClause}
    \SetKwFunction{LCP}{LeastConditional}
    \SetKwFunction{Propagate}{Propagate}
    \SetKwFunction{Shrink}{Shrink}
    \SetKwFunction{Filter}{Filter}
    \SetKwFor{For}{for}{:}{}
    \SetKwFor{If}{if}{:}{}
    \SetKwProg{Fn}{Function}{:}{}
    \SetKwBlock{Begin}{}{}
    \Fn{\Learn{$\formula$, $\alpha$}}{
        \For{$i \in vars(\formula)$}{ \label{line:choose-i}
            \For{$j \in vars(\formula)$}{ \label{line:choose-j}
                \Propagate($i$); \\
                \Propagate($j$); \\
                $\alpha_c, \alpha_a := \LCP(\formula, \alpha)$; \\
                $C := \Shrink(\formula,\alpha_c, \alpha_a)$; \\
                \If{not \Filter{C,$\formula$}}{
                $\formula := \formula \land C$;}
            }
        }
    }
\end{algorithm}

We present the general framework for using conditional autarkies to learn PR clauses in \autoref{alg:methodology}. 
A conditional autarky is computed based on a partial assignment, 
so the algorithm first selects a set of literals to propagate (lines $4,5$), creating the partial assignment $\alpha$.
We propagate two literals at a time using nested for loops. 
Then, in line $6$ \autoref{alg:leastcond} makes a single pass over the formula to generate a conditional autarky, with conditional part ($\alpha_c$) and autarky part ($\alpha_a$).
The PR clause $C$ is generated from the conditional autarky and shrunk in line $7$, and if $C$ does not pass the usefulness heuristics it is filtered away in line $8$. 
Details regarding design choices and heuristics are found in \autoref{sec:implementation}.

In the following sections, we will describe PR clause learning and shrinking in the context of our running example  \ph{4}.
We will consider the decisions $i = x_{1, 1}$ and $j = x_{2, 2}$ shown in \autoref{subfig:pigeonholeclause-a}, with unit propagation shown in \autoref{subfig:pigeonholeclause-b}.

%. This lets us learn the clause
%$\overline{x}_{1, 2} \lor \overline{x}_{2, 1}$ (see
%\autoref{fig:pigeonholeclauses}), which is necessary for the \pr proof of
%pigeonhole \cite{prclauses}. 



%For motivation, we will discuss the pigeonhole example \ph{4}, where we make
%decisions $i = x_{1, 1}$ and $j = x_{2, 2}$. This lets us learn the clause
%$\overline{x}_{1, 2} \lor \overline{x}_{2, 1}$ (see
%\autoref{fig:pigeonholeclauses}), which is necessary for the \pr proof of
%pigeonhole \cite{prclauses}. 
% We explain how this clause is used in the proof in
% Appendix~\ref{app:pigeonhole}.


%We start by propagating on $x_{1, 1}$ and $x_{2, 2}$, which assigns
%$\overline{x}_{2, 1}, \ldots, \overline{x}_{5, 1}, \overline{x}_{1, 2},
%\overline{x}_{3, 2} \ldots, \overline{x}_{5, 2}$ by constraint (2) (see
%\autoref{subfig:pigeonholeclause-b}).

\subsection{Learning PR Clauses}~\label{subsec:learning}


As described in \autoref{subsec:autarkies}, any partial assignment can be split 
into a conditional and autarky part, 
$\alpha = \alpha_c \sqcup \alpha_a$, 
and the following clauses are PR: $\bigvee_{c \in \alpha_c} \overline{c} \lor a$ for
any $a \in \alpha_a$. 
In order to produce smaller PR clauses, we use the technique 
\autoref{alg:leastcond} from Kiesl et al.
\cite{conditionalautarkies} to find the unique smallest possible $\alpha_c$:

%Since all of $\alpha_c$ appears in such a clause, we want
%minimize it using \autoref{alg:leastcond} from Kiesl et al.
%\cite{conditionalautarkies} to find the unique smallest possible $\alpha_c$:


\begin{algorithm}
    \caption{Unique minimal $\alpha_c$ in $\alpha = \alpha_c \sqcup \alpha_a$}\label{alg:leastcond}
    \SetAlgoNoLine
    \SetKwFunction{LCP}{LeastConditional}
    \SetKwFor{For}{for}{:}{}
    \SetKwFor{If}{if}{:}{}
    \SetKwProg{Fn}{Function}{:}{}
    \SetKwBlock{Begin}{}{}

    \Fn{\LCP{$\formula$, $\alpha$}}{

        $\alpha_c := \emptyset$\;
        \For{$C \in \formula$}{
            \If{$\alpha$ touches $C$ without satisfying $C$}{
                $\alpha_c := \alpha_c \cup (\alpha \cap \lnot{C})$\;
            }
        }
        \Return{$\alpha_c$, $\alpha \backslash \alpha_c$}\;
    }
\end{algorithm}

This is a conditional autarky, since every clause that $\alpha_a$ touches is
satisfied by some literal in $\alpha$.

Additionally, $\alpha_c$ is minimal: for each clause that is touched but not
satisfied, we add to $\alpha_c$ all literals from the assignment that touch and do
not satisfy this clause. They must be in $\alpha_c$, since otherwise $\alpha_a$
will touch a clause that is not satisfied by $\alpha$, violating the conditional
autarky property.

Running \autoref{alg:leastcond} on the partial assignment from \autoref{subfig:pigeonholeclause-b}  
gives the conditional part (red) and autarky part (orange) in \autoref{subfig:pigeonholeclause-c}. 
The assigned literals from pigeons $3,4,$ and $5$ appear in $\alpha_c$ because they touch 
but do not satisfy the constraint ($1$) clauses for the respective pigeons stating that every pigeon is in a hole. 
However, the constraint ($1$) clauses are satisfied for pigeons $1$ and $2$, 
along with the touched constraint $2$ clauses, placing the pigeons $1$ and $2$ literals in $\alpha_a$. 
Intuitively, if pigeons $3,4,$ and $5$ are not in holes $1$ or $2$, then piegon $1$ can be placed in hole $1$ : 
$x_{3,1} \lor x_{3,2} \lor x_{4,1} \lor x_{4,2} \lor x_{5,1} \lor x_{5,2} \lor x_{1,1} $ and pigeon $1$ can be kept out of hole $2$ :
$x_{3,1} \lor x_{3,2} \lor x_{4,1} \lor x_{4,2} \lor x_{5,1} \lor x_{5,2} \lor \overline{x}_{1,2} $ (likewise for pigeon $2$). 
%and pigeon $2$ can be placed in hole $2$ : $x_{3,1} \lor x_{3,2} \lor x_{4,1} \lor x_{4,2} \lor x_{5,1} \lor x_{5,2} \lor x_{2,2} $. 
The shrinking technique in the following section will help reduce the size of these large PR clauses. 


%From our pigeonhole example, we can see the rows in
%\autoref{subfig:pigeonholeclause-c} with no $+$ will have their constraint (1)
%touched but not satisfied, and thus are in $\alpha_c$. However, the rows with a
%$+$ will have their constraint (1) satisfied and thus are in $\alpha_a$. Thus,
%$\alpha_a = \{x_{1, 1}, \overline{x}_{1, 2}, \overline{x}_{2, 1}, x_{2, 2}\}$
%and $\alpha_c = \{\overline{x}_{3, 1}, \ldots, \overline{x}_{5, 1}, \overline{x}_{3,
%2} \ldots, \overline{x}_{5, 2}\}$.

\subsection{Shrinking PR Clauses}~\label{subsec:shrinking}

The PR clause derived from a conditional autarky may be too weak to effectively reduce the search space, 
%In a formula $\formula$ with conditional autarky $\alpha = \alpha_c \sqcup \alpha_a$ where $\alpha_c = c_1, \dots, c_n$ and $\alpha_a =
%a_1, \dots, a_m$, the addition of clause $C = \alpha_c \lor a_i$ may be too weak to effectively reduce the search space.
but shrinking the PR clause, i.e., removing literals from the clause via resolution, will strengthen its pruning power. 
%Thus, our goal is to find resolution candidates containing literals from the autarky part. 
%We thus aim to see which literals from the autarky part we may use to resolve with values from $\alpha_c$ (and thus $C$).

At a high-level, we will generate two sets: $C_0 \subset \alpha_c$ and $A_0 \subset \alpha_a$, 
and show that we can learn the PR clause $\bigvee_{c \in C \backslash C_0} \overline{c} \lor
\bigvee_{a \in A_0} a$. 

First we start with $C_0$, the set of literals in the conditional part that are inconsistent with any literal in the autarky part.
Two literals $l_i$ and $l_j$ are inconsistent in a formula $\formula$ if  $\impunitclauseNoNeg{\formula}{l_i}{\overline{l}_j}$, meaning the clause $\overline{l}_i \lor \overline{l}_j$ is RUP. 
Formally, $C_0 = \{c_j \in \alpha_c \mid \exists a_i \in
\alpha_a \text{ s.t. } \impunitclause{\formula}{a_i}{\overline{c}_j} \}$, with $a_i$ appearing negated in the inconsistency check so that we can perform a resolution between the derived binary clause  $a_i \lor c_j$ and the original PR clause $C$.

Next we generate $A_0$ which is a subset of the autarky literals that are together inconsistent with the literals in $C_0$. 
So, for each $c \in C_0$ there exists an $a \in A_0$ such that $\impunitclause{\formula}{a}{c}$. 
There always exists at least one $A_0$, namely $\alpha_a$ which is used to define $C_0$, 
but if a smaller $A_0$ exists it can be used to shrink the size of the PR clause. 


%
%Consider the set $C_0$ of literals in the conditional part which
%are inconsistent with any literal in the autarky part. 
%Formally, let $C_0 = \{c_j \in C \mid \exists a_i \in
%\alpha_a \text{ s.t. } \impunitclause{\formula}{a_i}{c_j} \}$.
%  todo : provide some intutition for how we pick this set
%
% This is essentially the set of literals in $\alpha_c$ that can be implied by unit propagation by negating some literal in $\alpha_a$.
%
% we could add clauses of the form $\overline{c_1} \lor \dots \lor \overline{c_n} \lor a_i$ where $1 \leq i \leq m$.
%
% , this may still lead to a long clause. We aim to shrink this clause. Notice that if $\alpha_c = c_1, \dots, c_n$ and $\alpha_a = a_1, \dots, a_m$, we could add clauses of the form $\overline{c_1} \lor \dots \lor \overline{c_n} \lor a_i$ where $1 \leq i \leq m$.
%
%
% Consider $\overline{\alpha_a}$, the negation of the literals in $\alpha_a$. 
%
%  in the code we propagate -alpha_a[i]
%  then if val (neg_alpha_c[j]) < 0, we add neg_alpha_c[j] to propagated and
%  alpha_a[i] to useful
%Consider additionally $A_0 \subseteq \alpha_a$ such that for each $c \in C_0$,
%there is some $a \in A_0$ such that $\impunitclause{\formula}{a}{c}$.
%There exists at least one such $A_0$ ($\alpha_a$ is a trivial example); however, a smaller set is typically more helpful for creating useful clauses. 
%We discuss our method to efficiently choose $A_0$ in \autoref{subsec:sym}.
%  as it will be important for having a technique resistant to different encodings. 


% Instead we look at $\overline{I} = \{x_{2, 1}, x_{1, 2}\}$. Then take the set $\overline{I} \vdash_1 C_0 \subseteq C$, i.e., the maximal subset of $C$ that $\overline{I}$ implies via unit propagation.

%We want learn the clause $\bigvee_{c \in C \backslash C_0} \overline{c} \lor
%\bigvee_{a \in A_0} a$


\begin{theorem}~\label{thm:shrunkgbcequisat}
    Let $\Gamma$ be a formula and $\alpha = \alpha_c \sqcup \alpha_a$ be a conditional autarky on $\Gamma$, with $\alpha_c = c_1, \dots, c_n$ and $\alpha_a = a_1, \dots, a_m$. Formula $\formula$ is satisfiable if and only if $\formula \land (\bigvee_{c \in C \backslash C_0} \overline{c} \lor \bigvee_{a \in A_0} a)$ is satisfiable.
\end{theorem}

\begin{proof}
    \underline{$\Leftarrow$:} This is immediate


    \underline{$\Rightarrow$:}  As a corollary to \autoref{thm:gbcequisat},
    $\formula$ is satisfiable if and only if $\formula \land (\bigvee_{c \in C}
    \overline{c} \lor \bigvee_{a \in A_0} a)$ is satisfiable.
    
    By the definition of inconsistency, for each $c \in C_0$ there exists an $a \in A_0$ such that $\impunitclause{\formula}{a}{c}$, 
    allowing us to derive the RUP clause  $a \lor c$. 
    Each binary clause can be resolved with $ (\bigvee_{c \in C}
    \overline{c} \lor \bigvee_{a \in A_0} a)$, producing  $(\bigvee_{c \in C \backslash C_0} \overline{c} \lor \bigvee_{a \in A_0} a)$. 
    

%    Thus we can assume $\formula \land (\bigvee_{c \in C} \overline{c} \lor
%    \bigvee_{a \in A_0} a)$ is satisfiable by some assignment
%    $\beta$.
%
%    We claim $\beta$ is a satisfying assignment for $\formula \land (\bigvee_{c \in
%    C \backslash C_0} \overline{c} \lor \bigvee_{a \in A_0} a)$. If this was not
%    the case then: (1)~for all $a \in A_0$ $\overline{a} \in \beta$ and (2)~there
%    is some $c \in C_0$ such that $\overline{c} \in \beta$. 
%
%    But by definition, there is some $a \in A_0$ such that
%    $\impunitclause{\formula}{a}{c}$. Thus $\formula \land \overline{a} \land
%    \overline{c}$ is unsatisfiable via unit propagation. However, this cannot be
%    the case since $\beta$ is a satisfying assignment for $\formula$ and
%    $\overline{a}, \overline{c} \in \beta$.
\end{proof}

%Another way to see this is: starting with $\bigvee_{c \in C} \overline{c} \lor
%\bigvee_{a \in A_0} a$, we notice that if some clause $c_i \lor a_i$ is implied
%via unit propagation, then we can remove $c_i$ from $C$ by applying resolution
%on the two clauses.
% twain : i tried to explain this at the start of the section; hopefully effectively

%\subsection{Shrinking with Greedy Set Cover}~\label{subsec:sym}
%
%In this section we present a heuristic for minimizing the size of $A_0$ without drastically increasing complexity.


\noindent \textbf{Greedy Set Cover.}~\label{subsec:sym} %~\label{subsubsec:greedysetcover}
We can interpret the problem of finding the smallest possible $A_0$ as a set cover problem 
where each literal $a \in \alpha_a$ defines the set  $SETS(a) = \{ c \in \alpha_c \; | \; \impunitclause{\formula}{a}{c}\}$.
Finding the minimum set cover is NP-hard, so we instead approximate using the greedy algorithm, returning a set cover at most roughly $(\ln |\alpha_c| + 1)\times$ the size of the smallest set cover~\cite{greedysetcover}. 


%Say for each $a \in \alpha_a$, we define $\alpha_a^\mathrm{SETS}(a) = \{ c \in \alpha_c \; | \; \impunitclause{\formula}{a}{c}\}$. 
%When $C_0 = \alpha_c$, finding the smallest $A_0$ is exactly the set cover problem with $\alpha_a^\mathrm{SETS}$. This is NP-hard, so we instead approximate using the greedy algorithm, returning a set cover at most roughly $(\ln |\alpha_c| + 1)\times$ the size of the smallest set cover~\cite{greedysetcover}. 
%  in actuality it is $\ln |\alpha_c| - \ln \ln |\alpha_c| + O(1)$


The greedy algorithm initializes $C_0$ as empty, and in each iteration finds the $a \in \alpha_a$ that generates the largest set $SETS(a) \cap (\alpha_c \backslash C_0)$, adding $a$ to $A_0$ and adding $SETS(a)$ to $C_0$.
The algorithm terminates once all sets in $\{SETS(a) \cap (\alpha_c \backslash C_0) | a \in \alpha_a \}$ are empty.
% either $C_0 = \alpha_C$ or there are no terms 


\begin{algorithm}
    \caption{Algorithm finding $A_0$}\label{alg:finda0}
    \SetAlgoNoLine
    \SetKwFunction{Shrink}{Shrink}
    \SetKwFor{For}{for}{:}{}
    \SetKwFor{If}{if}{:}{}
    \SetKwProg{Fn}{Function}{:}{}
    \Fn{\Shrink{$\formula$, $\alpha_a$, $\alpha_c$}}{
        % $\alpha_a$ := \texttt{sort($\alpha_a$)}\;
        $SETS$ := init \texttt{Array[len($\alpha_a$)]}\;
        \For{$i \in \texttt{range}(\alpha_a)$}{
            \texttt{Propagate($-\alpha_a[i]$)}\;
            \texttt{implied} := $\{\}$\;
            \For{$c \in \alpha_c$}{
                \texttt{propagate(-c)}\;
                \If{\texttt{unsat}}{
                $\texttt{implied} := \texttt{implied} \cup \{c\}$\;
            }
            }
            $SETS[i]$ := \texttt{implied}\;
        }
        \Return{\texttt{GreedySetCover}($SETS$)}\;
    }
\end{algorithm}
% todo : this algorithm is not actually what we do, since we don't propagate on c -> see if this is a problem ever

In \autoref{alg:finda0}, we describe our process for calculating $A_0$. 
% We initially pre-sort $\alpha_a$ by implication, which we will discuss in more
% detail. 
We initialize $SETS$ as an array, iterate the counter
$i$ through $\alpha_a$, populating $SETS[i]$ with the set of literals
$c \in \alpha_c$ such that $\impunitclause{\formula}{a_i}{c}$. Finally, we apply
\texttt{GreedySetCover} to get a small set $A_0$ that covers as much of $C$ as possible.

Returning to the pigeonhole example, 
$C_0$ contains all of the literals in $\alpha_c$ because of the binary clauses in constraint (2). 
The smallest possible $A_0$ includes $\overline{x}_{2, 1}$ and $\overline{x}_{1, 2}$, whose 
sets cover \overline{x}_{3, 1}, \overline{x}_{4, 1},  \ldots, \overline{x}_{5,
1}$and  $\overline{x}_{3, 2}, \overline{x}_{4, 2},  \ldots, \overline{x}_{5,
2}$ respectively. 
Whereas, the sets from $x_{1, 1}$ and $x_{2, 2}$ are empty (they are not inconsistent with the other literals).
Therefore, we can learn the clause $\overline{x}_{2, 1} \lor \overline{x}_{1, 2}$, shown in 
\autoref{subfig:pigeonholeclause-d}.

%In the context of the pigeonhole example, out of the $\alpha_a$,
%$\overline{x}_{2, 1}$ can remove $\overline{x}_{3, 1}, \ldots, \overline{x}_{5,
%1}$ from $\alpha_c$. Additionally, $\overline{x}_{1, 2}$ can be used to remove
%$\overline{x}_{3, 2}, \ldots, \overline{x}_{5, 2}$ from $\alpha_c$. However,
%$x_{1, 1}$ and $x_{2, 2}$ do not remove anything. Thus, the greedy set cover
%algorithm will give us $A_0 = \{\overline{x}_{2, 1}, \overline{x}_{1, 2}\}$ and
%$C_0 = C$.

% \subsection{Algorithm on Pigeonhole Principle}~\label{subsec:pigeonhole}



