\section{Methodology}~\label{sec:method}

\subsection{Motivating Example}~\label{sec:motivatex}

\begin{figure*}[!t]
    \centering
    \begin{subfigure}[t]{0.23\textwidth}
    \centering
    \begin{tikzpicture}
      % Draw the 4×5 grid and name it "M"
      \matrix (M) [gridmatrix] {
        % row 1
        |[bluepluscell]| & |[graycell]| & |[graycell]| & |[graycell]| \\
        % row 2
        |[graycell]|  & |[bluepluscell]| & |[graycell]| & |[graycell]| \\
        % row 3
        |[graycell]| & |[graycell]| & |[graycell]| & |[graycell]| \\
        % row 4
        |[graycell]| & |[graycell]| & |[graycell]| & |[graycell]| \\
        % row 5
        |[graycell]| & |[graycell]| & |[graycell]| & |[graycell]| \\
      };

      % Column labels: attach ABOVE the north anchor of row 1 cells
      \foreach \c [count=\i from 1] in {$h_1$,$h_2$,$h_3$,$h_4$} {
        \node[anchor=south] at (M-1-\i.north) {\c};
      }

      % Row labels: attach LEFT of the west anchor of column 1 cells
      \foreach \r [count=\j from 1] in {$p_1$,$p_2$,$p_3$,$p_4$,$p_5$} {
        \node[anchor=east] at (M-\j-1.west) {\r};
      }
    \end{tikzpicture}
    \captionsetup{width=0.8\textwidth}  % override for this one only
    \caption{We initially propagate on $x_{1, 1}$ and $x_{2, 2}$.}~\label{subfig:pigeonholeclause-a}  
\end{subfigure}
\begin{subfigure}[t]{0.23\textwidth}
    \centering
    \begin{tikzpicture}
        % Draw the 4×5 grid and name it "M"
        \matrix (M) [gridmatrix] {
        % row 1
        |[bluepluscell]| & |[yellowminuscell]| & |[graycell]| & |[graycell]| \\
        % row 2
        |[yellowminuscell]|  & |[bluepluscell]| & |[graycell]| & |[graycell]| \\
        % row 3
        |[yellowminuscell]| & |[yellowminuscell]| & |[graycell]| & |[graycell]| \\
        % row 4
        |[yellowminuscell]| & |[yellowminuscell]| & |[graycell]| & |[graycell]| \\
        % row 5
        |[yellowminuscell]| & |[yellowminuscell]| & |[graycell]| & |[graycell]| \\
        };

        % Column labels: attach ABOVE the north anchor of row 1 cells
        \foreach \c [count=\i from 1] in {$h_1$,$h_2$,$h_3$,$h_4$} {
            \node[anchor=south] at (M-1-\i.north) {\c};
        }

        % Row labels: attach LEFT of the west anchor of column 1 cells
        \foreach \r [count=\j from 1] in {$p_1$,$p_2$,$p_3$,$p_4$,$p_5$} {
            \node[anchor=east] at (M-\j-1.west) {\r};
        }
    \end{tikzpicture}
    \captionsetup{width=0.8\textwidth}  % override for this one only
    \caption{Then unit propagation gives us $\overline{x}_{2,1}, \ldots, \overline{x}_{5,1},$ $\overline{x}_{1, 2}, \overline{x}_{3, 2}, \ldots, \overline{x}_{5, 2}$.}~\label{subfig:pigeonholeclause-b} 
\end{subfigure}
\begin{subfigure}[t]{0.23\textwidth}
    \centering
    \begin{tikzpicture}
    % Draw the 4×5 grid and name it "M"
    \matrix (M) [gridmatrix] {
        % row 1
        |[orangepluscell]| & |[orangeminuscell]| & |[graycell]| & |[graycell]| \\
        % row 2
        |[orangeminuscell]|  & |[orangepluscell]| & |[graycell]| & |[graycell]| \\
        % row 3
        |[purpleminuscell]| & |[purpleminuscell]| & |[graycell]| & |[graycell]| \\
        % row 4
        |[purpleminuscell]| & |[purpleminuscell]| & |[graycell]| & |[graycell]| \\
        % row 5
        |[purpleminuscell]| & |[purpleminuscell]| & |[graycell]| & |[graycell]| \\
    };

    % Column labels: attach ABOVE the north anchor of row 1 cells
    \foreach \c [count=\i from 1] in {$h_1$,$h_2$,$h_3$,$h_4$} {
        \node[anchor=south] at (M-1-\i.north) {\c};
    }

    % Row labels: attach LEFT of the west anchor of column 1 cells
    \foreach \r [count=\j from 1] in {$p_1$,$p_2$,$p_3$,$p_4$,$p_5$} {
        \node[anchor=east] at (M-\j-1.west) {\r};
    }
    \end{tikzpicture}
    \captionsetup{width=0.8\textwidth}  % override for this one only
    \caption{Literals $\overline{x}_{3, 1}, \ldots, \overline{x}_{5, 1},$ $\overline{x}_{3, 2}, \ldots, \overline{x}_{5, 2}$ are in $\alpha_c$ by their respective constraint (1). The rest are in $\alpha_a$.}~\label{subfig:pigeonholeclause-c}
\end{subfigure}
\begin{subfigure}[t]{0.23\textwidth}
    \centering
    \begin{tikzpicture}
    % Draw the 4×5 grid and name it "M"
    \matrix (M) [gridmatrix] {
        % row 1
        |[graycell]| & |[orangeminuscell]| & |[graycell]| & |[graycell]| \\
        % row 2
        |[orangeminuscell]|  & |[graycell]| & |[graycell]| & |[graycell]| \\
        % row 3
        |[graycell]| & |[graycell]| & |[graycell]| & |[graycell]| \\
        % row 4
        |[graycell]| & |[graycell]| & |[graycell]| & |[graycell]| \\
        % row 5
        |[graycell]| & |[graycell]| & |[graycell]| & |[graycell]| \\
    };

    % Column labels: attach ABOVE the north anchor of row 1 cells
    \foreach \c [count=\i from 1] in {$h_1$,$h_2$,$h_3$,$h_4$} {
        \node[anchor=south] at (M-1-\i.north) {\c};
    }

    % Row labels: attach LEFT of the west anchor of column 1 cells
    \foreach \r [count=\j from 1] in {$p_1$,$p_2$,$p_3$,$p_4$,$p_5$} {
        \node[anchor=east] at (M-\j-1.west) {\r};
    }
    \end{tikzpicture}
    \captionsetup{width=0.8\textwidth}  % override for this one only
    \caption{Finally, we shrink and learn the clause $\overline{x}_{1, 2} \lor \overline{x}_{2, 1}$.
    E.g., we can omit $\overline{x}_{3, 2}$ from the condition because of clause $\overline{x}_{1, 2} \lor \overline{x}_{3, 2}$.}~\label{subfig:pigeonholeclause-d}
\end{subfigure}


    \caption{Learning the clause $\overline{x}_{1, 2} \lor \overline{x}_{2, 1}$ for \ph{4}}~\label{fig:pigeonholeclauses}
  \end{figure*}

We use the pigeonhole principle as a motivating example.
The problem $\ph{n}$ asks whether we can put $n+1$ pigeons in $n$ holes such
that (1)~every pigeon is in a hole, and (2)~no hole contains more than one
pigeon. This can be encoded as a SAT problem where variable $x_{i, j}$
represents putting the $i$-th pigeon into the $j$-th hole. Thus, constraint (1)
is $\bigvee_{1 \leq j \leq n} x_{i, j}$ for each $1 \leq i \leq n+1$ and
constraint (2) is $\overline{x}_{i, j} \lor \overline{x}_{k, j}$ for each $ 1
\leq i < k \leq n+1$ and $1 \leq j \leq n$.

\autoref{fig:pigeonholeclauses} provides a visualization where the
rows represent the pigeons and the columns represent the holes. The cell in row
$i$ and column $j$ represents the literal $x_{i, j}$
% i.e. whether the $i$-th pigeon is in the $j$-th hole.
If the $(i, j)$-th cell has a $+$ symbol, this represents $x_{i, j}$ is set to
true and if it has a $-$ symbol this represents $x_{i, j}$ is set to false.
Thus, constraint (1) asks that each row has at least one $+$ cell and
constraint (2) asks that no two $+$ cells share a column.

% We only consider the case where $m = n + 1$ which we denote \ph{n}. 
We achieve an $O(n^3)$ \pr proof, matching the best known
result~\cite{prclauses}. We learn these proofs with a very low constant
factor and little sensitivity to the encoding.
%  todo: add a forward reference to where we evaluate this
%  todo: is this the best possible result theoretically?

% There are $O(n^3)$ \pr proofs of the pigeonhole principle, but this is difficult to achieve in practice and requires specific techniques~\cite{prclauses}. We can learn these proofs, with a very low constant factor and little sensitivity to the encoding. 

\subsection{Algorithm}~\label{subsec:methodology}


At a high level, \autoref{alg:methodology} learns a \pr clause by: (I)
propagating on two variables $i$ and $j$, (II) learning the least conditional
autarky of the partial assignment, (III) shrinking the clause, and (IV) adding
the clause to the formula. 

The \texttt{Propagate} function represents unit propagation.
\texttt{LeastConditional} function achieves II, following prior work
(\autoref{subsec:learning}). The \texttt{Shrink} function takes care of III
(\autoref{subsec:shrinking}). All four stages require several heuristic choices (\autoref{sec:implementation}).

\begin{algorithm}~\label{alg:methodology}
    \caption{Learning \pr clauses}\label{alg:methodology}
    \SetAlgoNoLine
    \SetKwFunction{Learn}{LearnClause}
    \SetKwFunction{LCP}{LeastConditional}
    \SetKwFunction{Propagate}{Propagate}
    \SetKwFunction{Shrink}{Shrink}
    \SetKwFor{For}{for}{:}{}
    \SetKwFor{If}{if}{:}{}
    \SetKwProg{Fn}{Function}{:}{}
    \SetKwBlock{Begin}{}{}
    \Fn{\Learn{$\psi$, $\alpha$}}{
        \For{$i \in vars(\psi)$}{ \label{line:sort-alphaa}
            \For{$j \in vars(\psi)$}{
                \Propagate($i$); \\
                \Propagate($j$); \\
                $\alpha_c, \alpha_a := \LCP(\psi, \alpha)$; \\
                $C := \Shrink(\psi,\alpha_c, \alpha_a)$; \\
                $\psi := \psi \land C$;
            }
        }
    }
\end{algorithm}

For motivation, we will discuss the pigeonhole example \ph{4}, where we make
decisions $i = x_{1, 1}$ and $j = x_{2, 2}$. This lets us learn the clause
$\overline{x}_{1, 2} \lor \overline{x}_{2, 1}$ (see
\autoref{fig:pigeonholeclauses}), which is necessary for the \pr proof of
pigeonhole \cite{prclauses}. 
% We explain how this clause is used in the proof in
% Appendix~\ref{app:pigeonhole}.


We start by propagating on $x_{1, 1}$ and $x_{2, 2}$, which assigns
$\overline{x}_{2, 1}, \ldots, \overline{x}_{5, 1}, \overline{x}_{1, 2},
\overline{x}_{3, 2} \ldots, \overline{x}_{5, 2}$ by constraint (2) (see
\autoref{subfig:pigeonholeclause-b}).

\subsection{Learning Clauses}~\label{subsec:learning}


As described in \autoref{subsec:autarkies}, we can split any partial assignment
$\alpha = \alpha_c \sqcup \alpha_a$ to learn a \pr clause $\alpha_c \lor a$ for
any $a \in \alpha_a$. Since all of $\alpha_c$ appears in such a clause, we want
minimize it using \autoref{alg:leastcond} from Kiesl et al.
\cite{conditionalautarkies} to find the unique smallest possible $\alpha_c$:


\begin{algorithm}
    \caption{Unique minimal $\alpha_c$ in $\alpha = \alpha_c \sqcup \alpha_a$}\label{alg:leastcond}
    \SetAlgoNoLine
    \SetKwFunction{LCP}{LeastConditional}
    \SetKwFor{For}{for}{:}{}
    \SetKwFor{If}{if}{:}{}
    \SetKwProg{Fn}{Function}{:}{}
    \SetKwBlock{Begin}{}{}

    \Fn{\LCP{$\psi$, $\alpha$}}{

        $\alpha_c := \emptyset$\;
        \For{$C \in \psi$}{
            \If{$\alpha$ touches $C$ without satisfying $C$}{
                $\alpha_c := \alpha_c \cup (\alpha \cap \overline{C})$\;
            }
        }
        \Return{$\alpha_c$, $\alpha \backslash \alpha_c$}\;
    }
\end{algorithm}

This is a conditional autarky, since every clause that $\alpha_a$ touches, is
satisfied by some literal in $\alpha$.

Additionally, $\alpha_c$ is minimal: for each clause that is touched but not
satisfied, we add to $\alpha_c$ all literals from the assignment that touch and do
not satisfy this clause. They must be in $\alpha_c$, since otherwise $\alpha_a$
will touch a clause that is not satisfied by $\alpha$, violating the conditional
autarky property.

From our pigeonhole example, we can see the rows in
\autoref{subfig:pigeonholeclause-c} with no $+$, will have their constraint (1)
touched but not satisfied and thus are in $\alpha_c$. However, the rows with a
$+$ will have their constraint (1) satisfied and thus are in $\alpha_a$. Thus,
$\alpha_a = \{x_{1, 1}, \overline{x}_{1, 2}, \overline{x}_{2, 1}, x_{2, 2}\}$
and $\alpha_c = \{\overline{x}_{3, 1}, \ldots, \overline{x}_{5, 1}, \overline{x}_{3,
2} \ldots, \overline{x}_{5, 2}\}$.

\subsection{Shrinking Clauses}~\label{subsec:shrinking}

Unfortunately, a clause $\alpha_c \lor a$ may still be too long. We aim to
shrink this clause. Notice that if $\alpha_c = c_1, \dots, c_n$ and $\alpha_a =
a_1, \dots, a_m$, we can consider the set $C_0 = \{c_j \in C \mid \exists a_i \in
\alpha_a \text{ s.t. } \impunitclause{\phi}{a_i}{c_j} \}$.
%  todo : provide some intutition for how we pick this set

% This is essentially the set of literals in $\alpha_c$ that can be implied by unit propagation by negating some literal in $\alpha_a$.

% we could add clauses of the form $\overline{c_1} \lor ... \lor \overline{c_n} \lor a_i$ where $1 \leq i \leq m$.

% , this may still lead to a long clause. We aim to shrink this clause. Notice that if $\alpha_c = c_1, ..., c_n$ and $\alpha_a = a_1, ..., a_m$, we could add clauses of the form $\overline{c_1} \lor ... \lor \overline{c_n} \lor a_i$ where $1 \leq i \leq m$.


% Consider $\overline{\alpha_a}$, the negation of the literals in $\alpha_a$. 

%  in the code we propagate -alpha_a[i]
%  then if val (neg_alpha_c[j]) < 0, we add neg_alpha_c[j] to propagated and
%  alpha_a[i] to useful

We also define $A_0 \subseteq \alpha_a$ as a set such that for each $c \in C_0$,
there is some $a \in A_0$ such that $\impunitclause{\phi}{a}{c}$. By definition,
there exists such an $A_0$, but there are many possible options. We discuss how
we choose $A_0$ in \autoref{subsec:sym}.
%  as it will be important for having a technique resistant to different encodings. 


% Instead we look at $\overline{I} = \{x_{2, 1}, x_{1, 2}\}$. Then take the set $\overline{I} \vdash_1 C_0 \subseteq C$, i.e., the maximal subset of $C$ that $\overline{I}$ implies via unit propagation.

We want learn the clause $\bigvee_{c \in C \backslash C_0} \overline{c} \lor
\bigvee_{a \in A_0} a$


\begin{theorem}~\label{thm:shrunkgbcequisat}
    The formula $\phi$ is satisfiable if and only if $\phi \land (\bigvee_{c \in C \backslash C_0} \overline{c} \lor \bigvee_{a \in A_0} a)$ is satisfiable.
\end{theorem}

\begin{proof}
    \underline{$\Leftarrow$:} This is immediate


    \underline{$\Rightarrow$:}  As a corollary to \autoref{thm:gbcequisat},
    $\phi$ is satisfiable if and only if $\phi \land (\bigvee_{c \in C}
    \overline{c} \lor \bigvee_{a \in A_0} a)$ is satisfiable.

    Thus we can assume $\phi \land (\bigvee_{c \in C} \overline{c} \lor
    \bigvee_{a \in A_0} a)$ is satisfiable by some assignment
    $\beta$.

    We claim $\beta$ is a satisfying assignment for $\phi \land (\bigvee_{c \in
    C \backslash C_0} \overline{c} \lor \bigvee_{a \in A_0} a)$. If this was not
    the case then: (1)~for all $a \in A_0$ $\overline{a} \in \beta$ and (2)~there
    is some $c \in C_0$ such that $\overline{c} \in \beta$. 

    But by definition, there is some $a \in A_0$ such that
    $\impunitclause{\phi}{a}{c}$. Thus $\phi \land \overline{a} \land
    \overline{c}$ is unsatisfiable via unit propagation. However, this cannot be
    the case since $\beta$ is a satisfying assignment for $\phi$ and
    $\overline{a}, \overline{c} \in \beta$.
\end{proof}

Another way to see this is: starting with $\bigvee_{c \in C} \overline{c} \lor
\bigvee_{a \in A_0} a$, we notice that if some clause $c_i \lor a_i$ is implied
via unit propagation, then we can remove $c_i$ from $C$ by applying resolution
on the two clauses.

\subsection{Clause Shrinking Algorithm}~\label{subsec:sym}

As mentioned in \autoref{subsec:shrinking}, our choice for $A_0$ is very important.. 


\noindent \textbf{Greedy Set Cover:} %~\label{subsubsec:greedysetcover}
In order to minimize the size of the clause, we may want the smallest possible $A_0$ that maximizes $C_0$. Say for each $a \in \alpha_a$, we define $\alpha_a^\mathrm{SETS}(a) = \{ c \in \alpha_c \; | \; \impunitclause{\phi}{a}{c}\}$. 

When $C_0 = \alpha_C$, finding the smallest $A_0$ is exactly the set cover problem with $\alpha_a^\mathrm{SETS}$. This is NP-hard. We can approximate using the greedy algorithm, which returns a set cover at most roughly $(\ln |\alpha_c| + 1)$-times the size of the smallest set cover~\cite{greedysetcover}. 
%  in actuality it is $\ln |\alpha_c| - \ln \ln |\alpha_c| + O(1)$


Specifically, the greedy algorithm will find the largest set in $\{\alpha_a^\mathrm{SETS} \cap (\alpha_c \backslash C_0) | a \in \alpha_a \}$ at each step and add it to $C_0$. It stops when each of the $\alpha_a^\mathrm{SETS} \cap (\alpha_c \backslash C_0)$ are empty.
% either $C_0 = \alpha_C$ or there are no terms 


\begin{algorithm}
    \caption{Algorithm finding $A_0$}\label{alg:finda0}
    \SetAlgoNoLine
    \SetKwFunction{Shrink}{Shrink}
    \SetKwFor{For}{for}{:}{}
    \SetKwFor{If}{if}{:}{}
    \SetKwProg{Fn}{Function}{:}{}
    \Fn{\Shrink{$\psi$, $\alpha_a$, $\alpha_c$}}{
        % $\alpha_a$ := \texttt{sort($\alpha_a$)}\;
        $\alpha_a^\mathrm{SETS}$ := init \texttt{Array[len($\alpha_a$)]}\;
        \For{$i \in \texttt{range}(\alpha_a)$}{
            \texttt{Propagate($-\alpha_a[i]$)}\;
            \texttt{implied} := $\{\}$\;
            \For{$c \in \alpha_c$}{
                \texttt{propagate(-c)}\;
                \If{\texttt{unsat}}{
                $\texttt{implied} := \texttt{implied} \cup \{c\}$\;
            }
            }
            $\alpha_a^\mathrm{SETS}[i]$ := \texttt{implied}\;
        }
        \Return{\texttt{GreedySetCover}($\alpha_a^\mathrm{SETS}$)}\;
    }
\end{algorithm}
% todo : this algorithm is not actually what we do, since we don't propagate on c -> see if this is a problem ever

In \autoref{alg:finda0}, we describe our process for calculating $A_0$. 
% We initially pre-sort $\alpha_a$ by implication, which we will discuss in more
% detail. 
We initialize $\alpha_a^\mathrm{SETS}$ as an array, iterate the counter
$i$ through $\alpha_a$, populating $\alpha_a^\mathrm{SETS}[i]$ with the set of literals
$c \in \alpha_c$ such that $\impunitclause{\psi}{\alpha_a^\mathrm{SETS}[i]}{c}$. Finally, we apply
\texttt{GreedySetCover} to get a small set $A_0$ that covers as much of $C$ as we can.

In the context of the pigeonhole example, out of the $\alpha_a$,
$\overline{x}_{2, 1}$ can remove $\overline{x}_{3, 1}, \ldots, \overline{x}_{5,
1}$ from $\alpha_c$. Additionally, $\overline{x}_{1, 2}$ can be used to remove
$\overline{x}_{3, 2}, \ldots, \overline{x}_{5, 2}$ from $\alpha_c$. However,
$x_{1, 1}$ and $x_{2, 2}$ do not remove anything. Thus, the greedy set cover
algorithm will give us $A_0 = \{\overline{x}_{2, 1}, \overline{x}_{1, 2}\}$ and
$C_0 = C$.

% \subsection{Algorithm on Pigeonhole Principle}~\label{subsec:pigeonhole}



