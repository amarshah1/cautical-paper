\section{Introduction}~\label{sec:intro}


Boolean satisfiability (SAT) solving is a core tool in computer science with
applications in program
verification~\cite{BillionQueries,sat-hardwareverification,ic3,bmc},
planning~\cite{planning,planningassat}, cryptography~\cite{cryptominisat}, and
mathematics~\cite{chromaticnumber,pythagoreantriples,kellersconjecture,emptyhexagon}.
As its usage expands, so too does the need for more powerful and specialized
solving techniques.


One such class of techniques is propagation redundant (PR) clause
learning~\cite{prclauses}. In this technique, a solver generates and adds
clauses which are satisfiability-preserving; that is, clauses whose conjunction
with the formula is satisfiable if and only if the original formula is
satisfiable. The solver aims to learn clauses that drastically shrink the
potential solution space, such as clauses that break symmetries.


The power of this technique is tied to the strength of the underlying proof
system: propagation redundancy (\pr). A more powerful proof system can admit
shorter proofs, which can be found and checked faster. Most SAT solvers rely on
resolution-based proof systems, which are ineffective for many hard instances.
Consider the pigeonhole principle (PHP), which asks if it is possible to fit
$n+1$ pigeons into $n$ holes with at most one pigeon per hole. Using \pr
reasoning, one may learn a PR clause equivalent to the lemma: \emph{pigeon 1 is
not in hole 1}. Adding this clause restricts the search space for the first
pigeon (and thus the problem), but it preserves satisfiability, as hole 1 is
symmetric with all holes. 

It is difficult to perform similar reasoning compactly with resolution, and in
fact, PHP requires exponentially large resolution proofs~\cite{hakenpigeonhole}.
The \pr proof system, however, has short proofs for such problems, including a
cubic proof (in the number of pigeons) for PHP~\cite{prclauses}. PHP frequently
occurs as a subproblem for SAT benchmarks, so it is important to be able to
solve it efficiently.

The theoretical power of PR learning comes at a cost: efficiently learning
useful PR clauses is a challenge. State-of-the-art techniques such as
Satisfiability Driven Clause Learning (SDCL) rely on calling another SAT solver
to verify that a candidate clause is PR \cite{sadical}. 
Thus, in the worst case, the solver takes exponential time to learn a single PR
clause. This makes integration into high-performance solvers difficult and
limits their practical impact. To the best of our knowledge, none of the most popular SAT solvers
such as \cadical~\cite{cadical}, \kissat~\cite{kissat},
\cryptoMiniSAT~\cite{cryptominisat}, or \lingeling~\cite{lingeling} support
PR clause learning in their main branch.

In this work, we propose a new, efficient approach to learn PR clauses
based on conditional autarkies. Intuitively, a conditional autarky provides a
way to force the value of certain variables given a set of assumptions, e.g., in
PHP if pigeons $3 $ to $ n+1$ are not in hole $1$ or hole $2$ (see \autoref{subfig:pigeonholeclause-c}
for a visual representation), then pigeon $1$
can be placed in hole $1$ and pigeon $2$ can be placed in hole $2$. Kiesl et.
al. proposed an algorithm for finding conditional autarkies in linear
time~\cite{conditionalautarkies}, using them to delete clauses. In our work, we
will use conditional autarkies to construct and add PR clauses. 


While conditional autarkies present a means for bridging the theoretical gap in
identifying PR clauses, the PR clauses they produce are often not immediately useful
in real SAT solving applications. There are two major practical limitations: (1)
larger PR clauses may not meaningfully constrain the search space, and (2) some
smaller PR clauses may be trivial and distract the solver. We solve (1) by
introducing a shrinking procedure to extract compact, useful PR clauses, and
(2) by introducing a number of heuristics for filtering away trivial PR
clauses. Returning to PHP, instead of the conditional autarky described above
which produces a clause with at least $2n$ literals, it is possible through
shrinking to learn a binary PR clause either forbidding pigeon $1$ in hole $2$ or
pigeon $2$ in hole $1$.


We make the following contributions: 

\begin{enumerate} 
    \item We introduce a method for learning PR clauses in linear time relative to the
    size of the formula. 
    \item We develop a number of shrinking and filtering heuristics to learn
    concise and useful PR clauses. 
    \item We implement these techniques in a solver, \tool\footnote{\tool's code
    is available at \url{https://github.com/amarshah10/cautical}.} (a fork of
    the state-of-the-art SAT solver \cadical), and evaluate it on both PHP benchmarks 
    and a suite of benchmarks from the annual SAT competitions.
\end{enumerate}