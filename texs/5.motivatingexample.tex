\section{Motivating Example}~\label{sec:motivatex}


In this section, we will walk through the classic example of the Pigeonhole Principle as a SAT problem~\cite{}, and how our techniques can be used to solve it.

The problem asks whether we can put $m$ pigeons in $n$ holes such that (1)~every pigeon is in a hole, and (2)~no hole contains more than one pigeon. This can be easily encoded as a SAT problem with variable $x_{i, j}$ represents putting the $i$-th pigeon into the $j$-th hole. We can represent constraint (1) as $\overline{x_{i, j}} \lor \overline{x_{k, j}}$ for each $ 1 \leq i \neq k \leq m$ and constraint (2) as $\bigvee_{1 \leq j \leq n} x_{i, j}$ for each $1 \leq i \leq m$

We will only consider the case where $m = n + 1$ and this is always unsatisfiable. We are able to give an $O(n^3)$ size proof for the unsatisfiability of the pigeonhole principle.

The key idea is learning binary clauses such as either pigeon 1 is not in hole 2 or pigeon 2 is not in hole 1. Then we learn the clause that either pigeon 1 is not in hole 3 or pigeon 2 is not in hole 1, and so on. These clauses may seem strange, but we can prove (in the \pr system) that their addition preserves satisfiability. We keep going until we eventually get that if pigeon 2 is in hole 1, then pigeon 1 is not in any hole. This violates constraint (1) and thus, we learn that pigeon 2 is not in hole 1. 


This results in an $O(n^3)$ size pigeon hole proof for the pigeonhole benchmarks. Additionally, since discovering globally blocked clauses is linear in the size of the formula and the formula is $O(n^2)$ for the pigeonhole benchmarks. The total runtime of our algorithm on pigeonhole benchmarks is $O(n^5)$

To better motivate this, we will discuss an example based on the pigeonhole problem with $5$ pigeons and $4$ holes. First, we include a visual and then explain what we have done.
% \begin{figure}[!hbt]
%   \centering
%   \subfloat[Learns the clause $\overline{x_{1, 1}}$]{\includegraphics[width=0.4\textwidth]{pigeonhole/1.png}
%   % \subcaption{}
%   }
%   \qquad
%   \subfloat[Learns the clause $\overline{x_{2, 1}}$]{\includegraphics[width=0.4\textwidth]{pigeonhole/2.png}}\\
%   \subfloat[Learns the clause $\overline{x_{3, 1}}$]{\includegraphics[width=0.4\textwidth]{pigeonhole/3.png}}
%   \qquad
%   \subfloat[Learns the clauses $\overline{x_{3, 2}}$ and $\overline{x_{4, 2}}$]{\includegraphics[width=0.4\textwidth]{pigeonhole/4.png}}\\
%   \subfloat[Learns the clause $\overline{x_{4, 3}}$]{\includegraphics[width=0.4\textwidth]{pigeonhole/5.png}}
%   \qquad
%   \subfloat[Implies the clause $x_{4, 4}$]{\includegraphics[width=0.4\textwidth]{pigeonhole/6.png}}\\
%   \subfloat[Implies $\overline{x_{1, 4}}$, $\overline{x_{2, 4}}$, $\overline{x_{3, 4}}$, $\overline{x_{5, 4}}$]{\includegraphics[width=0.4\textwidth]{pigeonhole/7.png}}
%   \qquad
%   \subfloat[Implies $x_{3, 3}$]{\includegraphics[width=0.4\textwidth]{pigeonhole/8.png}}\\
%   \subfloat[Implies $\overline{x_{1, 3}}$, $\overline{x_{2, 3}}$, $\overline{x_{5, 3}}$]{\includegraphics[width=0.4\textwidth]{pigeonhole/9.png}}
%   \qquad
%   \subfloat[Implies $x_{2, 2}$]{\includegraphics[width=0.4\textwidth]{pigeonhole/10.png}}
%   \\
%   \subfloat[Implies $\overline{x_{2, 1}}$, $\overline{x_{2, 5}}$]{\includegraphics[width=0.4\textwidth]{pigeonhole/11.png}}
%   \qquad
%   \subfloat[Contradiction, this implies both $x_{1, 1}$ and $x_{5, 1}$!]{\includegraphics[width=0.4\textwidth]{pigeonhole/12.png}}

%   \caption{An illustration of how CAutiCal learns clauses for Pigeonhole with $5$ pigeons and $4$ holes. All of these unit clauses are learned after computing an $O(n^2)$ number of globally blocked clauses}
%   \label{fig:pigeonhole}
% \end{figure}

Essentially, we learn the following sets of clauses:

We make a set of alternating decisions $x_{1, 1}, x_{2, j}$ and then $x_{1, j} x_{2, 1}$  for $j = 2, ..., 4$.

The first pair (when $j = 2$ will allow us to lead us to learn the two globally blocked clauses:

\begin{gather*}
    \textcolor{purple}{\overline{x_{3, 1}} \lor \overline{x_{4, 1}} \lor \overline{x_{5, 1}}} \lor \textcolor{purple}{\overline{x_{3, 2}} \lor \overline{x_{4, 2}} \lor \overline{x_{5, 2}}} \lor \textcolor{cyan}{\overline{x_{1, 2}}} \\
    \textcolor{purple}{\overline{x_{3, 1}} \lor \overline{x_{4, 1}} \lor \overline{x_{5, 1}}} \lor \textcolor{purple}{\overline{x_{3, 2}} \lor \overline{x_{4, 2}} \lor \overline{x_{5, 2}}} \lor \textcolor{cyan}{\overline{x_{2, 1}}} 
\end{gather*}

Then when we introduce the decision $x_{1, 2}, x_{2, 1}$, we would reach a conflict by unit propagation, and thus introduce the conflict clause $\overline{x_{1, 2}} \lor \overline{x_{2, 1}}$.

We repeat this to also obtain the conflict clauses $\overline{x_{1, 2}} \lor \overline{x_{2, 1}}$, $\overline{x_{1, 3}} \lor \overline{x_{2, 1}}$, $\overline{x_{1, 4}} \lor \overline{x_{2, 1}}$.

The globally blocked clauses combined with these conflict clauses is enough to learn the clause $\overline{x_{2, 1}}$. We repeat to learn $\overline{x_{3, 1}}, \overline{x_{4, 1}}$ and so on until we reach a contradiction. 


% \begin{example}
%     Consider the following SAT problem:
    
% \end{example}
