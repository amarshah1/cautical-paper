\section{Motivating Example}~\label{sec:motivatex}

\begin{figure}[!t]
    \centering
    \begin{subfigure}[t]{0.23\textwidth}
    \centering
    \begin{tikzpicture}
      % Draw the 4×5 grid and name it "M"
      \matrix (M) [gridmatrix] {
        % row 1
        |[graycell]| & |[graycell]| & |[graycell]| & |[graycell]| \\
        % row 2
        |[redcell]|  & |[graycell]| & |[graycell]| & |[graycell]| \\
        % row 3
        |[graycell]| & |[graycell]| & |[graycell]| & |[graycell]| \\
        % row 4
        |[graycell]| & |[graycell]| & |[graycell]| & |[graycell]| \\
        % row 5
        |[graycell]| & |[graycell]| & |[graycell]| & |[graycell]| \\
      };

      % Column labels: attach ABOVE the north anchor of row 1 cells
      \foreach \c [count=\i from 1] in {$h_1$,$h_2$,$h_3$,$h_4$} {
        \node[anchor=south] at (M-1-\i.north) {\c};
      }

      % Row labels: attach LEFT of the west anchor of column 1 cells
      \foreach \r [count=\j from 1] in {$p_1$,$p_2$,$p_3$,$p_4$,$p_5$} {
        \node[anchor=east] at (M-\j-1.west) {\r};
      }
    \end{tikzpicture}
    \captionsetup{width=0.8\textwidth}  % override for this one only
    \caption{The first unit learned is of the form $x_{2, 1}$}~\label{subfig:pigeonhole-a}  
\end{subfigure}
\begin{subfigure}[t]{0.23\textwidth}
    \centering
    \begin{tikzpicture}
        % Draw the 4×5 grid and name it "M"
        \matrix (M) [gridmatrix] {
            % row 1
            |[graycell]| & |[graycell]| & |[graycell]| & |[graycell]| \\
            % row 2
            |[redcell]|  & |[graycell]| & |[graycell]| & |[graycell]| \\
            % row 3
            |[redcell]| & |[graycell]| & |[graycell]| & |[graycell]| \\
            % row 4
            |[redcell]| & |[graycell]| & |[graycell]| & |[graycell]| \\
            % row 5
            |[graycell]| & |[graycell]| & |[graycell]| & |[graycell]| \\
        };

        % Column labels: attach ABOVE the north anchor of row 1 cells
        \foreach \c [count=\i from 1] in {$h_1$,$h_2$,$h_3$,$h_4$} {
            \node[anchor=south] at (M-1-\i.north) {\c};
        }

        % Row labels: attach LEFT of the west anchor of column 1 cells
        \foreach \r [count=\j from 1] in {$p_1$,$p_2$,$p_3$,$p_4$,$p_5$} {
            \node[anchor=east] at (M-\j-1.west) {\r};
        }
    \end{tikzpicture}
    \captionsetup{width=0.8\textwidth}  % override for this one only
    \caption{In a similar fashion we then learn the units $\overline{x}_{3, 1}$ and $\overline{x}_{4, 1}$}~\label{subfig:pigeonhole-b} 
\end{subfigure}
\begin{subfigure}[t]{0.23\textwidth}
    \centering
    \begin{tikzpicture}
    % Draw the 4×5 grid and name it "M"
    \matrix (M) [gridmatrix] {
        % row 1
        |[graycell]| & |[redcell]| & |[redcell]| & |[redcell]| \\
        % row 2
        |[redcell]|  & |[graycell]| & |[graycell]| & |[graycell]| \\
        % row 3
        |[redcell]| & |[graycell]| & |[graycell]| & |[graycell]| \\
        % row 4
        |[redcell]| & |[graycell]| & |[graycell]| & |[graycell]| \\
        % row 5
        |[graycell]| & |[graycell]| & |[graycell]| & |[graycell]| \\
    };

    % Column labels: attach ABOVE the north anchor of row 1 cells
    \foreach \c [count=\i from 1] in {$h_1$,$h_2$,$h_3$,$h_4$} {
        \node[anchor=south] at (M-1-\i.north) {\c};
    }

    % Row labels: attach LEFT of the west anchor of column 1 cells
    \foreach \r [count=\j from 1] in {$p_1$,$p_2$,$p_3$,$p_4$,$p_5$} {
        \node[anchor=east] at (M-\j-1.west) {\r};
    }
    \end{tikzpicture}
    \captionsetup{width=0.8\textwidth}  % override for this one only
    \caption{By the symmetry argument, we learn units $\overline{x}_{1, 2}$, $\overline{x}_{1, 3}$, and $\overline{x}_{1, 4}$}~\label{subfig:pigeonhole-c}
\end{subfigure}
\begin{subfigure}[t]{0.23\textwidth}
    \centering
    \begin{tikzpicture}
    % Draw the 4×5 grid and name it "M"
    \matrix (M) [gridmatrix] {
        % row 1
        |[greencell]| & |[redcell]| & |[redcell]| & |[redcell]| \\
        % row 2
        |[redcell]|  & |[graycell]| & |[graycell]| & |[graycell]| \\
        % row 3
        |[redcell]| & |[graycell]| & |[graycell]| & |[graycell]| \\
        % row 4
        |[redcell]| & |[graycell]| & |[graycell]| & |[graycell]| \\
        % row 5
        |[redcell]| & |[graycell]| & |[graycell]| & |[graycell]| \\
    };

    % Column labels: attach ABOVE the north anchor of row 1 cells
    \foreach \c [count=\i from 1] in {$h_1$,$h_2$,$h_3$,$h_4$} {
        \node[anchor=south] at (M-1-\i.north) {\c};
    }

    % Row labels: attach LEFT of the west anchor of column 1 cells
    \foreach \r [count=\j from 1] in {$p_1$,$p_2$,$p_3$,$p_4$,$p_5$} {
        \node[anchor=east] at (M-\j-1.west) {\r};
    }
    \end{tikzpicture}
    \captionsetup{width=0.8\textwidth}  % override for this one only
    \caption{Constraint (1) gives $x_{1, 1}$, then constraint (2) gives $\overline{x}_{5, 1}$, completing the reduction to \ph{3}}~\label{subfig:pigeonhole-d}
\end{subfigure}


    \caption{Process for reducing \ph{4} to \ph{3}}~\label{fig:pigeonhole}
  \end{figure}

We will interleave our methodology with the pigeonhole principle as a motivating
example. The problem \ph{n} asks whether we can put $n+1$ pigeons in $n$ holes
such that (1)~every pigeon is in a hole, and (2)~no hole contains more than one
pigeon. This can be easily encoded as a SAT problem where variable $x_{i, j}$
represents putting the $i$-th pigeon into the $j$-th hole. We can translate
constraint (1) as $\bigvee_{1 \leq j \leq n} x_{i, j}$ for each $1 \leq i \leq
m$. We can also translate constraint (2) as $\overline{x_{i, j}} \lor
\overline{x_{k, j}}$ for each $ 1 \leq i \neq k \leq m$ and $1 \leq j \leq n$.

We represent this diagrammatically as in \autoref{fig:pigeonhole}, where the
rows are the pigeons and the columns are the holes. The cell in row $i$ and
column $j$ represents the literal $x_{i, j}$, i.e. whether the $i$-th pigeon is
in the $j$-th hole. If the $(i, j)$-th cell is green this represents $x_{i, j}$
is set to true and if it is red this represents $x_{i, j}$ is set to false.
Thus, constraint (1) asks that each row has at least one green cell and
constraint (2) asks that no two green cells share a column.

We will only consider the case where $m = n + 1$ which we denote \ph{n}. This is
clearly unsatisfiable, however there are no polynomial-sized resolution proofs
of \ph{n}~\cite{hakenpigeonhole}. There are $O(n^3)$ \pr proofs of the
pigeonhole principle, but this is difficult to achieve in practice and requires
specific techniques~\cite{prclauses}. We can learn these proofs, with a very low
constant factor and little sensitivity to the encoding.


\subsection{Pigeonhole Setup}~\label{subsec:pigeonhole-setup}

Our technique makes a pair of decisions, propagates and then divides this assignment into a conditional and an autarky part, and finally shrinks the clause to derive a useful \pr clause.

While our technique does allow for decisions on negative literals, this is not useful for the Pigeonhole Principle. We first decide on some literal $i$. For ease, we assume that $i = x_{1, 1}$. This does not matter because of symmetry.

Notice that the set of literals we consider for $j$ are the \emph{touched literals}, i.e. those that are in a clause with $x_{1, 1}$ or any of the variables it propagates. By constraint (2), $x_{1, 1}$ propagates $\overline{x_{k, 1}}$ for $2 \leq k \leq n + 1$. But these literals will touch every literal in the formula via the clauses from constraint (1). 


\subsection{Solving the Pigeonhole Principle}~\label{subsec:solvingpigeonhole}

Below we describe three stages at which we learn useful clauses:


\subsubsection{Useful Binary Clauses}~\label{subec:usefulbinaryclauses} When learning a second variable $j= x_{2, 2}$, propagation gives us the literals $\overline{x_{2, 1}}, ... \overline{x_{n+1, 1}}$ and $\overline{x_{1, 2}}, \overline{x_{3, 2}}, ... \overline{x_{n+1, 2}}$. This can be divided into a autarky part $\alpha_a = x_{1, 1}, x_{2, 2}, \overline{x_{1, 2}}, \overline{x_{2, 1}}$ and a conditional part $\alpha_c = \overline{x_{3, 1}}, \ldots, \overline{x_{n+1, 1}}, \overline{x_{3, 2}}, \ldots, \overline{x_{n+1, 2}}$.

This can be shrunk to the clause $\overline{x_{1, 2}} \lor \overline{x_{2, 1}}$.
% todo : maybe say more about this shrinking

Then when considering the next assumption $j = x_{2, 3}$, we end up learning the clause $\overline{x_{1, 3}} \lor \overline{x_{2, 1}}$. This continues until we learn $\overline{x_{1, n}} \lor \overline{x_{2, 1}}$. Notice that have the clause $\overline{x_{1, n}} \lor \overline{x_{2, 1}}$ from constraint (2) and the clause $\bigwedge_{1 \leq j \leq n} x_{1, j}$ from constraint (1). These combined will yield the unit clause $\overline{x_{2, 1}}$ as shown in \autoref{subfig:pigeonhole-a}.

Similary, we can learn the unit clauses $\overline{x_{3, 1}}, \overline{x_{4, 1}}, \ldots, \overline{x_{n, 1}}$ as shown in \autoref{subfig:pigeonhole-b}.


\subsubsection{Useful Units} In the last row of learning, we get something even more useful. Starting with the assumption $j = x_{n+1, 2}$, we propagate the literals $\overline{x_{n + 1, 1}}$ and $\overline{x_{1, 2}}, \overline{x_{2, 2}}, \ldots, \overline{x_{n, 2}}$.  This can be divided into a autarky part $\alpha_a = x_{1, 1}, x_{n+1, 2}, \overline{x_{1, 2}}, \overline{x_{n+1, 1}}$ and a conditional part $\alpha_c = \overline{x_{1, 2}}, \ldots, \overline{x_{n, 2}}$.

Shrinking gives us just the unit clause $\overline{x_{1, 2}}$. Notice that if we had does this at the beginning, we would have also had $\overline{x_{2, 1}}, \ldots, \overline{x_{n, 1}}$ and thus shrinking would have given a binary clause.

In a similar manner, we also learn the units $\overline{x_{2, 1}}, \ldots, \overline{x_{n, 1}}$ as shown in \autoref{subfig:pigeonhole-c}.

\subsubsection{Final Simplifications} Finally, constraint (1) will allow us to learn the unit $x_{1, 1}$ and constraint (2) will imply the unit $x_{n+1, 1}$ as shown in \autoref{subfig:pigeonhole-d}. Thus, we have successfully reduced the \ph{n} to \ph{n-1}. Then we pick some other variable for $i$ (for instance $i = x_{2, 2}$) and repeat the process.

\subsection{Complexity Analysis}~\label{subsec:complexity analysis}

The complexity of this technique is $O(n^3)$ for the proof size. The time it takes to execute a proof step is linear in the size of a formula. Since the formula for pigeonhole has size $O(n^2)$, the total runtime is $O(n^5)$.

% \subsubsection{Proof Size}~\label{subsubsec:proof-size} 
We make a brief remark on the constant factors in the proof size as they compare favorably to similar techniques. First, in the useful binary clause phase, for each unit we learn, we must learn $n-1$ binary clauses. We end learning $n-1$ units, for a total of $(n-1)^2$ clauses (note the final binary clause and the unit clause can be learned together in a single step). Second, in the useful unit phase, we learn $n-1$ units, for a total of $n-1$ clauses. Third, in the final simplification phase, we learn 3 clauses. This gives a total of $n(n-1) + 3$ clauses to reduce \ph{n} to \ph{n-1}.

Thus, there needs to be 