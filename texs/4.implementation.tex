\section{Implementation}~\label{sec:implementation}

We implement our technique in \tool (a fork of \cadical
%  commit f13d74439a5b5c963ac5b02d05ce93a8098018b8
). By default, \tool searches for \pr clauses as a preprocessing step for 30
seconds. After this time limit, the solver exits and commences normal solving,
regardless of whether or not it has found any \pr clauses. This adds $\sim\!800$ lines
of C++ code to \cadical and is implementable within any CDCL SAT solver.
 

We experimented with other heuristics, which we briefly describe.
We include an evaluation of the heuristics in \autoref{subsec:heuristics}.

% \subsection{Other Heuristics}~\label{subsec:heuristics}

% todo: maybe move these two paragraphs to implementation section

% \noindent \textbf{Sorting $\alpha_a$ by Implication:}%~\label{subsubsec:impordering}
% The greedy set cover algorithm described above is almost sufficient for being insensitive to heuristics, but there is one other optimization that we made. \autoref{alg:finda0} describes sorting $\alpha_a$ as an initial step. This is important for cases where there are more than one cover of the same size. The most common occurence of this is for $a_1, a_2 \in \alpha_a$, we may have that $\overline{a_2} \impunit^\formula \overline{a_1} \impunit^\formula c_1, \dots, c_n$. 

% Thus, greedy set cover could pick $\{a_1\}$ or $\{a_2\}$ as singleton sets. However, since $a_1 \impunit^\formula a_2$, we really want to learn the unit clause $a_1$ since it is much more powerful.

\subsubsection{Sorting $\alpha_a$ by Implication}~\label{subsubsec:impordering}
% The greedy set cover algorithm described above is almost sufficient for being
% insensitive to heuristics, but there is one other optimization that we made.
We could sort $\alpha_a$ by implication before \autoref{alg:finda0}
\autoref{line:sort-alphaa}. This is important when there is more than one set
cover of the same size. For instance, if  $a_1, a_2 \in \alpha_a$ and
% todo: fix overline over a_1
$\impunitclause{\formula}{a_1}{a_2}$. Then, we would prefer to learn a clause with 
$a_1$ since it implies $a_2$. We select for this by first sorting $\alpha_a$ by
implication, i.e., we would move $a_1$ before $a_2$ in $\alpha_a$ if it were not
already there.
% todo: maybe delete this, if we do not evaluate on it

% and $\alpha_c = c$, we may have
% $\impunitclause{\psi}{a_1}{a_2}$ and $\impunitclause{\psi}{a_2}{c}$. Hence, we
% would also have $\impunitclause{\psi}{a_1}{c}$. Thus, we could learn the unit
% clause $a_1$ or the unit clause $a_2$.

% Here $a_1 \lor c$ is the more
% ``powerful'' clause, so it is better to learn. 

% todo: talk about the multiple propagation heuristic used for mutilated chessboard

\subsubsection{Filtering Trivial Clauses}~\label{subsubsec:filteringtriv}
Introducing globally blocked clauses can affect how we divide $\alpha = \alpha_c
\sqcup \alpha_a$ as in \autoref{alg:leastcond}. Thus, it is better to avoid
introducing globally blocked clauses if they are not useful. We do this by
checking if $\formula \impunit C$ for each potential clause $C$. If it is implied by
unit propagation, we consider it easy to learn and do not add it.

\subsubsection{Filtering Long Clauses}~\label{subsubsec:filtering-length}
Another heuristic for filtering is based on clause size. If a clause is longer
than some set length, we may choose not to learn it, as shorter clauses are
typically more useful. This can be combined with filtering out trivial clauses
or used separately.

\subsubsection{Ordering Literals}~\label{subsubsec:ordering-literals}
So far, we propagate on literals $i$ and $j$ selected randomly.
However, there could be better ways to order literals. One technique is to order
$i$ based on the number of clauses it occurs in. Another is to pick $j$ from the neighbors of $i$. This was originally proposed in
Reeves et al.~\cite{prelearn}.

\subsubsection{Alternative Shrinking
techniques}~\label{subsubsec:shrink-techniques} As discussed in
\autoref{subsec:shrinking}, we may shrink a clause if
$\impunitclause{\formula}{a}{c}$, where $a \in \alpha_a$ and $c \in \alpha_c$.
Checking if $\impunitclause{\formula}{a}{c}$ can be expensive. A cheaper option is
to see if there is a clause $a \lor c$ in the original formula. While this may
miss some pairs, it is sufficient in certain cases, such as for the
pigeonhole benchmarks. Another alternative is to not shrink the clause
at all.

% Here is a complete list of flags discussed in 
