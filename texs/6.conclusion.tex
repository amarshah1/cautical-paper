\section{Conclusion and Future Work}~\label{sec:conclusion}


We answer the research questions posed at the start of this section. For
question 1, we conclude positively that \tool is able to provide short \pr
proofs for certain benchmark families. We match the best known result for The

However, it is currently unknown whether conditional autarkies can be used to
learn the shortest known \pr proofs for the mutilated
chessboard~\cite{mutilatedchessboard-pr} or Tseitin graph
formulae~\cite{sadical}. Additionally, there are benchmarks in the ??, ??, and
?? families that \prelearn solves and \tool does not.

For question 2, we can conclude positively that \tool is less sensitive to
encoding choices compared to other \pr learning techniques. We show that \tool
has $O(n^3)$ proofs for the pigeonhole principle even when the variables are
permuted, clauses are permuted, and literals are flipped. Additionally, \tool
does not consider any information about the encoding when propagating literals
(it does so randomly). In fact, we show that heuristics used to rank order
literals have an adverse effect on performance.