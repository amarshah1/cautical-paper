\section{Background}~\label{sec:background}

We begin with some SAT preliminaries. Variables $x_1, x_2, \dots$ take values
\emph{true} ($\top$) or \emph{false} ($\bot$). A literal $l$ is a variable
$x$ or its negation $\overline{x}$. A clause $C$ is a disjunction of literals,
e.g., $C = x_1 \lor \overline{x}_4 \lor x_6$. A clause may also be represented by the set of its literals:
e.g., the prior clause $C = \{x_1, \overline{x}_4, x_6\}$. This motivates denoting that a literal
$l$ occurs in clause $C$ via $l \in C$. A conjunctive normal form (CNF)
formula $\formula$ is a conjunction of (disjunctive) clauses. In this paper,
all formulae $\formula$ are CNF.

We use $var(\formula)$ to represent the set of variables occurring in formula $\formula$.
A partial assignment $\alpha : V \rightarrow \{\top, \bot\}$ on $\formula$
maps variables $V \subseteq var(\formula)$ to true or false. If $V
= var(\formula)$, then $\alpha$ is a total assignment. We abuse notation by
denoting $\alpha(l) = \alpha(x)$ if $l = x$ and $\alpha(l) = \neg \alpha(x)$ if
$l = \overline{x}$.

A formula restricted to a partial assignment $\formula|_\alpha$ is the formula resulting from
mapping variables in the domain of $\alpha$ to their assignments. We can
simplify such a formula by removing \emph{literals} assigned false and
\emph{clauses} including a literal assigned true (which are thus satisfied).
For example, with formula $\formula =
(x_1 \lor \overline{x}_4 \lor x_6) \land (x_2 \lor x_3) \land \overline{x}_5$
and assignment $\alpha$ mapping $x_1$ to $\bot$ and $x_2$ to $\top$,
we have $\formula|_\alpha = (\overline{x}_4 \lor x_6) \land \overline{x}_5$.

A clause $C = l_1 \lor \dots \lor l_m$ is \emph{blocked} by an assignment
mapping each $l \in C$ to false ($\bot$). An assignment $\alpha$
\emph{touches} a clause $C$ if there is a variable $x$ assigned by $\alpha$
such that $x \in C$ or $\overline{x} \in C$. We say $\alpha$ \emph{satisfies}
$C$ if there is some $l \in C$ with $\alpha(l) = \top$.

Assignment $\alpha$ \emph{satisfies} formula $\formula$ if it satisfies every
clause $C \in \formula$. If there exists such a satisfying assignment, then $\formula$ is
\emph{satisfiable}. The boolean satisfiability problem asks whether or not a given
formula $\formula$ is satisfiable.

\subsection{Redundant Clauses}~\label{subsec:redundant}

A clause $C$ is \emph{redundant} (or \emph{satisfiability-preserving}) with
respect to a formula $\formula$ if the formulae $\formula$ and $\formula \land
C$ are \emph{equisatisfiable}, i.e., $\formula$ is satisfiable if and only if
$\formula \land C$ is satisfiable.

In a \emph{clausal proof system}, each step adds or removes a redundant clause
$C$. The step may contain extra information, such as a boolean witness
justifying $C$'s redundancy. A list of redundant clauses ending with the empty
clause $\bot$ is a proof of unsatisfiability for formula $\formula$.

% If $\formula$ is unsatisfiable, then any clause $C$ will be redundant.
% However, when solving, we do not know whether $\formula$ is satisfiable a
% priori. Clausal proof systems can justify that a clause is redundant.

Adding redundant clauses may be helpful to a solver if they greatly
constrain the set of possible solutions, giving the solver
less area to search; however, the addition of these clauses
could negatively interact with solver heuristics, increasing solve time.

% \subsection{Reasoning Steps}~\label{subsec:reasoning-steps}

Clausal proof systems range in complexity, with one of the simplest derivation rules being resolution. 
Given two clauses, $C \lor x$ and $\overline{x} \lor D$,  \emph{resolution} produces the logically implied clause  $C
\lor D$ .

%
%\emph{Resolution} is the main proof step of clausal proof systems. Given two
%clauses in a formula, $C \lor x$ and $\overline{x} \lor D$, resolution gives $C
%\lor D$ as an equisatisfiable clause.
%
%\begin{equation*}
%    \inferrule*[Right=Resolution]{
%        C \lor x \\ \overline{x} \lor D
%    }{
%        C \lor D
%    }    
%\end{equation*}
%
\emph{Unit propagation} is a core reasoning technique in a SAT solver. Starting with empty assignment $\alpha$, if a formula $\formula$ contains a \emph{unit clause} $l$, i.e., a clause with only a single literal, then we set $\alpha(l) = \top$. Then, this unit is propagated: we consider the formula $\formula|_\alpha$ and continue this process until there are no unit clauses remaining. When unit propagation terminates with $\formula|_\alpha = \bot$, we say that it derives a conflict.
% todo: notation issue: I never defined \bot as the empty clause and I also am using \bot both as false and the empty clause.
%
%Unit propagation can be thought of as a proof step. Indeed, it can be thought of
%as a mechanized application of resolution.
%
%\begin{equation*}
%    % \inferrule*[Right=Unit-Pos]{
%    %     C \lor l \\ l
%    % }{
%    %     \bot
%    % }
%    % \qquad \qquad \qquad
%    \inferrule*[Right=Unit-Prop]{
%        C \lor l \\ \overline{l}
%    }{
%        C
%    }
%\end{equation*}

Given a formula $\formula$ and clause $C = l_1 \lor \dots \lor l_k$, we can say $\formula \impunit C$, read as ``$\formula$ implies $C$ via unit propagation,'' if $\formula \land \overline{l}_1 \land \dots \land \overline{l}_k$ derives a conflict. Here, $C$ is a simple example of a redundant clause w.r.t. $\formula$. 

% This can be justified in almost any proof system for propositional logic. However, this is not very useful as it rules out
For two formulae, we say $\formula \impunit \formula'$ if $\formula \impunit C$ for every clause $C \in \formula'$. 
% For literals $l_1$ and $l_2$, we sometimes notate $l_1 \impunit^\formula l_2$ to mean $\formula \vdash_1 \overline{l}_1 \land l_2$. Informally, we can think of this as saying ``$l_1$ implies $l_2$ via unit propagation on $\formula$,'' which means that .
% see definition here: https://www.cs.cmu.edu/~mheule/publications/prencode.pdf
% However, this is not very useful as it is quiet easy for the solver to figure this out. Instead, we must appeal to a more complicated notion of redundancy.

% The propagation redundant (\pr)

\subsection{Propagation Redundant}~\label{subsec:pr}

While resolution is complete for propositional logic, more powerful proof steps
can yield shorter proofs and faster runtimes. One such step is based on \pr
clauses.

\begin{definition}[Propagation Redundant (\pr) clauses]
    For formula $\formula$, we say that clause $C$ is propagation redundant
    (\pr) if there exists a \emph{witness} assignment $\omega$ implying the
    redundancy of $C$, as checkable in polynomial time by a solver. 
\end{definition}

% Such a clause $C$ must be redundant. Say there is a satisfying assignment
% $\alpha$ for $\formula$. Then if $\alpha$ is not a satisfying assignment for
% $\formula \land C$, then it must be that $\beta \subseteq \alpha$, i.e.,
% $\alpha$ extends $\beta$. However, since $\formula|_\beta \impunit
% \formula|_\omega$, it must be that any assignment that satisfies
% $\formula|_\beta$, will satisfy $\formula|_\omega$. 

% Then we can define $\alpha' (x) = \omega(x)$ if $x \in \omega$ and $\alpha'(x)
% = \alpha(x)$ otherwise. Thus, $\alpha'$ satisfies $\formula$ and $\alpha'$
% satisfies $C$.

% Intuitively, we can think of adding $C$ as the constraint that prunes all
% assignments that extend $\beta$. Since $\formula|_\beta \impunit
% \formula|_\omega$, it must be that any assignment that satisfies
% $\formula_\beta$, will satisfy $\formula_\omega$.

% Additionally, since $\omega$ satisfies $C$, removing the assignments that
% extend $\beta$, will not affect satisfiability. % todo : this needs ot be
% explained better.

% If $C$ is a \pr clause w.r.t $\formula$, then $\formula$ and $\formula \land
% C$ are equisatisfiable. Indeed, the clause defined in \autoref{thm:gbcequisat}
% is \pr with witness $\alpha_a$. 

Checking if a clause is redundant without hints is NP-complete~\cite{prclause}.
Thus, \pr introduces witnesses, which must be provided for proof checking.

\pr clauses subsume many classes of redundant clauses, including resolution
asymmetric tautologies (RATs)~\cite{rat}, blocked clauses~\cite{blockedclause},
set-blocked clauses~\cite{setblocked}, and globally-blocked
clauses~\cite{conditionalautarkies}.

\subsection{Autarkies and Globally Blocked Clauses}~\label{subsec:autarkies}

% A key realization in prior work~\cite{}

\begin{definition}[Autarky]
    A nonempty assignment $\alpha$ is an autarky for a formula $\formula$ if
    every clause $C \in \formula$ touched by $\alpha$ is satisfied.
\end{definition}

% An autarky $\alpha_a = a_1, \dots, a_m$ can be very useful as $a_1 \land \dots
% \land a_m$ is an equisatisfiable clause. However, these can be difficult to
% find.

In plain words, an autarky is an assignment that satisfies every clause it touches. Satisfying assignments and pure literals are examples of autarkies.

However, formulae often do not contain autarkies, and they are difficult to find
even when they do exist. Much easier to find is the following weakening to the
concept of an autarky. 

\begin{definition}[Conditional Autarky]
    A nonempty assignment $\alpha = \alpha_c \sqcup \alpha_a$ (disjoint union)
    is a conditional autarky for a formula $\formula$ if $\alpha_a$ is an
    autarky for $\formula|_{\alpha_c}$.
\end{definition}

Consider, as example, the formula with aptly named variables $\formula = (c \lor
a) \land (\overline{a} \lor x) \land (\overline{c} \lor \overline{a} \lor
\overline{y})$. We have a conditional autarky $\alpha = \alpha_c \sqcup \alpha_a
= c\;a$, with $\alpha_c = c$ and $\alpha_a = a$. Since $\formula|_{\alpha_c} =
(a \lor x) \land (a \lor \overline{y})$ and $a$ occurs here only positively, $a$
is an autarky. Thus, $\alpha$ is a conditional autarky on $\formula$.

% Specifically, we can think about searching for conditional autarkies by first
% looking for a partial assignment $\alpha_c$, and then finding an autarky
% $\alpha_a$ on the reduced formula $\formula|_{\alpha_c}$.

Conditional autarkies can be very useful for learning equisatisfiable clauses,
for instance, if $\alpha_c = c_1, \dots, c_n$ and $\alpha_a = a_1, \dots, a_m$,
we add clauses of the form:

\begin{equation*}
    [c_1 \land \dots \land c_n] \rightarrow [a_1 \land \dots \land a_m]
\end{equation*}

This results in $m$ different clauses as in the following theorem from Kiesl et
al.~\cite{conditionalautarkies}:

\begin{theorem}~\label{thm:gbcequisat} Let $\Gamma$ be a formula and $\alpha =
    \alpha_c \sqcup \alpha_a$ be a conditional autarky on $\Gamma$, with
    $\alpha_c = c_1, \dots, c_n$ and $\alpha_a = a_1, \dots, a_m$. Formula
    $\formula$ and $\formula \land \bigwedge_{1 \leq i \leq m} (\overline{c}_1
    \lor \dots \lor \overline{c}_n \lor a_i)$ are equisatisfiable.
\end{theorem}

That is, a formula $\formula$ is satisfiable if and only if $\formula \land
\bigwedge_{1 \leq i \leq m} (\overline{c}_1 \lor \dots \lor \overline{c}_n \lor
a_i)$ is satisfiable. A solver may thus add any of the $m$ clauses
$\overline{c}_1 \lor \dots \lor \overline{c}_n \lor a_i$ and preserve
satisfiability. Each of these clauses is a \pr clause with witness $\alpha_a$.
% Each of these clauses is a \emph{globally blocked clause}, however, this is
% not too important for our purposes, so we will not discuss it further.
% Additionally, we can see that this is a \pr clause:

% todo: this theorem works for us because of the algorithm we use to prove this,
% but it is not true in the general case.

% \begin{theorem}~\label{thm:totalassignmnet} For $1 \leq i \leq m$, the clause
%     $C = \overline{c}_1 \lor \dots \overline{c}_n \lor a_i$ is a \pr clause
%     w.r.t $\formula$ with witness $\alpha_a$ \end{theorem}

% \begin{proof} Since $a_i \in \alpha_a$, $\alpha_a$ satisfies $C$.
   
%    Define the assignment that blocks $C$ as $\beta = c_1, \dots, c_n,
% \overline{a}_i$. Now we want to show that: $\formula|_\beta \impunit
% \formula|_{\alpha_a}$. Take a clause $C \in \formula|_{\alpha_a}$. We know
% that \end{proof}




\subsection{Related Work}~\label{subsec:relatedwork}

The pigeonhole problem asks if $n+1$ pigeons can fit into $n$ holes without overlap. The
mutilated chessboard problem asks if a $2n \times 2n$ board with two diagonally
opposite corners removed can be tiled with $1 \times 2$ rectangular dominoes.
It is easy to convince a human that both of these problems are unsatisfiable. However, it has
been shown there are no polynomial-sized resolution proofs for either
problem~\cite{hakenpigeonhole,mutilatedchessboard-exponential}.
% todo: need to clarify if I should say that there are not polynomial-sized
% resolution proofs for these problems or if there are no sub-exponential-sized
% resolution proofs for these problems.

% todo: need to be consistent about whether I am using terminology "pigeonhole
% principle" or "pigeonhole problem"

% todo: I need to cite some paper that introduces pigeonhole and mutilated
% chessboard


 While extended resolution (\er) provides proofs of size $O(n^4)$ for the pigeonhole problems, it involves introducing new variables~\cite{er}. In general, the search space for new variables is infinite and thus tools like \glucoser based on \er do not scale well~\cite{glucoser}.

The \pr proof system remedies this by producing $O(n^3)$ proofs for the pigeonhole formula~\cite{prclauses} without learning any new variables. 
Later, this system was shown to additionally produce short proofs for mutilated chessboard problems~\cite{mutilatedchessboard-pr}. 
% todo we use O(n^_) where n is formula, but also where n is the number of variables.

Solvers that implement the \pr proof system typically use the satisfaction-driven clause learning (SDCL) framework~\cite{sdcl}, which extends conflict-driven clause learning (CDCL)~\cite{cdcl}. 
% Here, if unit propagation does not lead to a conflict, then they attempt to learn a \pr clause instead.
% \pr clause learning was first introduced in an extension of the solver \lingeling~\cite{prclause}.
After propagating an assignment, they check if the clause $C$ blocked by this assignment is \pr by creating a new SAT formula called the \emph{positive reduct}.
If the positive reduct is satisfiable, then $C$ is a \pr clause and is added. 
This was implemented in an extension of the solver \lingeling and was shown to scale well on pigeonhole benchmarks.

Later, two new variants of the positive reduct were proposed for more aggressive pruning of the search space \cite{sadical}. This allowed SDCL to solve other difficult problems such as Tseitin formulae~\cite{hardexamplesresolution}. This is implemented in a new SDCL solver \sadical.


\prelearn is a preprocessing technique for \pr clauses~\cite{prelearn}. It initially considers many possible clauses and queries \sadical to see which are \pr.

Our work differs as we do not use a \emph{positive reduct} to test if a clause
is \pr. Instead, our clauses are \pr by construction, as they come from a conditional
autarky. This has the potential downside that our clause may be large, and thus weak. To
remedy this, we apply a shrinking technique, reducing the size of a clause.
Additionally, prior techniques are sensitive to the encoding of the problem.
Minor changes such as literal and clause reordering can tank performance. We
provide two \emph{symmetry hardening} techniques to handle reordering. We
compare our implementation \tool to \sadical and \prelearn in
\autoref{sec:evaluation}.

Kiesl et al.~\cite{conditionalautarkies} first introduced conditional autarkies to identify globally blocked clauses. They aimed to eliminate globally blocked clauses from a formula. This technique allowed them to simulate circuit-simplification techniques. Our work differs from this as we add clauses instead of removing them, and we target a different class of benchmarks.
