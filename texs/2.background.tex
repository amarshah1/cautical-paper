\section{Background}~\label{sec:background}

We begin with some SAT preliminaries. Variables $x_1, x_2, ...$ can take values true ($\top$) or false ($\bot$). A literal $l$ can be a variables $x$ or its negation $\overline{x}$. A clause $C$ is a disjunction of literals, e.g. $x_1 \lor \overline{x_4} \lor x_6$. A formula $\phi$ is a conjunction of clauses. We notate $var(\phi)$ as the set of variables that occur in formula $\phi$.

For some $V \subseteq var(\phi)$, an assignment $\alpha : V \rightarrow \{\top, \bot\}$  maps some variables from a formula $\phi$ to \emph{true} or \emph{false}. If $V = var(\phi)$, we say $\alpha$ is a total assignment. We denote a formula restricted to an assignment $\phi|_\alpha$ which is the formula where every variable $x$ in the domain of $\alpha$ is assigned to $\alpha(x)$. Note that we can simplify such a formula by removing all literals that are assigned to false and all clauses with a literal assigned to true.

We say that an assignment $\alpha$ touches a clause $C$ if there is a variable $x$ in the domain of $\alpha$ such that $x$ or $\overline{x}$ occurs in $C$. We say $\alpha$ satisfies $C$ if $\alpha(x) = \top$ and $x \in C$ or $\alpha(x) = \bot$ and $\overline{x} \in C$.

A total assignemnt $\alpha$ satisfies formula $\phi$ if it satisfies every clause in $\phi$. If there is such a satisfying assignment, we say $\phi$ is \emph{satisfiable}. The boolean satisfiability problem asks whether a given formula $\phi$ is satisfiable.

\subsection{Autarkies and Globally Blocked Clauses}~\label{subsec:autarkies}

\begin{definition}[Autarky]
    A nonempty assignment $\alpha$ is an autarky for a formula $\phi$ if every clause $C \in \phi$ touched by $\alpha$ is satisfied.
\end{definition}

% An autarky $\alpha_a = a_1, ..., a_m$ can be very useful as $a_1 \land ... \land a_m$ is an equisatisfiable clause. However, these can be difficult to find.

\begin{definition}[Conditional Autarky]
    A nonempty assignment $\alpha = \alpha_c \sqcup \alpha_a$ (disjoint union) is a conditional autarky for a formula $\phi$ if $\alpha_a$ is an autarky for $\phi|_{\alpha_c}$.
\end{definition}

Conditional autarkies can be very useful for learning equisatisfiable clauses, for instance if $\alpha_c = c_1, ..., c_n$ and $\alpha_a = a_1, ..., a_m$, we add clauses of the form:

\begin{equation*}
    c_1 \land ... \land c_n \rightarrow a_1 \land ... \land a_m
\end{equation*}

We repeat the following theorem from Kiesel et al~\cite{conditionalautarkies}:

\begin{theorem}~\label{thm:gbcequisat}
    Formula $\phi$ and $\phi \land \bigwedge_{1 \leq i \leq m} (\overline{c_1} \lor ... \overline{c_n} \lor a_i)$ are equisatisfiable.
\end{theorem}

This means that $\phi$ is satisfiable if and only if $\phi \land \bigwedge_{1 \leq i \leq m} (\overline{c_1} \lor ... \overline{c_n} \lor a_i)$ is satisfiable. Thus, the solver can add any of the $m$ clauses $\overline{c_1} \lor ... \overline{c_n} \lor a_i$.

\subsection{Propagation Redundant}~\label{subsec:pr}

\emph{Unit propagation} is one of the core reasoning techniques of a SAT solver. If a formula $\phi$, contains a unit clause $l$, i.e. a clause with only a single literal, then it can be propagated, i.e. every clause with $l$ can be removed and every literal $\overline{l}$ can be removed. We repeat this process until there are no unit clauses remaining.

Given a formula $\phi$ and clause $C = l_1 \lor ... \lor l_k$, we can say $\phi \impunit C$, i.e. ``$\phi$ implies $C$ via unit propagation,'' if $\phi \land \overline{l_1} \land ... \land \overline{l_k}$ can be shown to be unsatisfiable via unit propagation. Additionaly for two formulas we can say $\phi \impunit \psi$ if for every clause $C \in \psi$, $\phi \impunit C$. For literals $l_1$ and $l_2$, we sometimes notate $l_1 \impunit^\phi l_2$, i.e. ``$l_1$ implies $l_2$ via unit propagation on $\phi$,'' which means that $\phi \land l_1 \land \overline{l_2}$ can be shown to be unsatisfiable only using unit propagation.  % see definition here: https://www.cs.cmu.edu/~mheule/publications/prencode.pdf

% The propagation redundant (\pr)

We say that clause $C = l_1 \lor ... \lor l_m$ is blocked by the partial assignment $\alpha$ that assigns each $l_i$ to false ($\bot$).

\begin{definition}[Propagation Redundant (\pr) clauses]
    For formula $\phi$, we say that clause $C$ (blocked by $\beta$) is propagation redundant (\pr) w.r.t. $\phi$ if there exists an assignment $\omega$ known as the witness such that $F|_\beta \impunit F_\omega$ and $\omega$ satisfies $C$
\end{definition}

If $C$ is a \pr clause w.r.t $\phi$, then $\phi$ and $\phi \land C$ are equisatisfiable. Indeed, the clause defined in \autoref{thm:gbcequisat} is \pr with witness $\alpha_a$. Checking if a clause is \pr without a witness is NP-complete~\cite{prclause}, so witnesses are typically provided when given a \pr proof to check.

\begin{theorem}~\label{thm:totalassignmnet}
    For $1 \leq i \leq m$, the clause $C = \overline{c_1} \lor ... \overline{c_n} \lor a_i$ is a \pr clause w.r.t $\phi$ with witness $\alpha_a$
\end{theorem}

\begin{proof}
   Since $a_i \in \alpha_a$, $\alpha_a$ satisfies $C$.
   
   Define the assignment that blocks $C$ as $\beta = c_1, ..., c_n, \overline{a_i}$. Now we want to show that: $F|_\beta \impunit F|_{\alpha_a}$. Take a clause $C \in F|_{\alpha_a}$. We know that
\end{proof}


\subsection{Related Work}~\label{subsec:relatedwork}

\pr clause learning was first introduced in an extension of the solver \lingeling~\cite{prclause}. After propagating an assignment, they would check if the clause $C$ that was blocked by this assignment was \pr. This was done by creating a new sat formula called the \emph{positive reduct}. If the positive reduct was satisfiable, then $C$ is a \pr clause and it was added. This approach was shown to scale well on pigeonhole benchmarks.

This was generalized by \sadical which provided two new variants of the positive reduct for more aggressive pruning of the search space \cite{sadical}. Finally, this was extended by \prelearn, which added unary and binary \pr clauses in a preprocessing step \cite{prelearn}. This was done by initially considering all possible clauses and querying \sadical to see which were \pr.

Our work differs as we do not use a \emph{positive reduct} to test is a clause is \pr. Instead, our clauses \pr by construction since they come from a conditional autarkies. This has the potential downside that our clause may be very large. To remedy this we apply a shrinking technique to reduce the size of a clause. Additionally, prior techniques are sensitive to the encoding of the problem. Minor changes such as literal and clause reordering can tank performance. We provide two \emph{symmetry hardening} techniques to handle reordering. We compare our implementation \tool to \sadical and \prelearn in \autoref{sec:evaluation}.